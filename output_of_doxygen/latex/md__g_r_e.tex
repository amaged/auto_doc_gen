\subsection*{cat /etc/network/interfaces}

\#\#\# 
\begin{DoxyCode}{0}
\DoxyCodeLine{auto lo}
\DoxyCodeLine{iface lo inet loopback}
\DoxyCodeLine{    address 10.0.0.41/32}
\DoxyCodeLine{    address fd00::41/128}
\DoxyCodeLine{}
\DoxyCodeLine{auto eth0}
\DoxyCodeLine{iface eth0 inet static}
\DoxyCodeLine{    address 10.255.0.1}
\DoxyCodeLine{    netmask 255.255.0.0}
\DoxyCodeLine{    gateway 10.255.0.3}
\DoxyCodeLine{auto eth1}
\DoxyCodeLine{iface eth1 inet static}
\DoxyCodeLine{    address 192.168.0.254}
\DoxyCodeLine{    netmask 255.255.0.0}
\DoxyCodeLine{}
\DoxyCodeLine{auto eth2}
\DoxyCodeLine{iface eth2}
\DoxyCodeLine{}
\DoxyCodeLine{auto eth3}
\DoxyCodeLine{iface eth3}
\end{DoxyCode}
 
\begin{DoxyCode}{0}
\DoxyCodeLine{Usage: cat [OPTION]... [FILE]...}
\DoxyCodeLine{Concatenate FILE(s), or standard input, to standard output.}
\DoxyCodeLine{}
\DoxyCodeLine{  -A, --show-all           equivalent to -vET}
\DoxyCodeLine{  -b, --number-nonblank    number nonempty output lines, overrides -n}
\DoxyCodeLine{  -e                       equivalent to -vE}
\DoxyCodeLine{  -E, --show-ends          display \$ at end of each line}
\DoxyCodeLine{  -n, --number             number all output lines}
\DoxyCodeLine{  -s, --squeeze-blank      suppress repeated empty output lines}
\DoxyCodeLine{  -t                       equivalent to -vT}
\DoxyCodeLine{  -T, --show-tabs          display TAB characters as \string^I}
\DoxyCodeLine{  -u                       (ignored)}
\DoxyCodeLine{  -v, --show-nonprinting   use \string^ and M- notation, except for LFD and TAB}
\DoxyCodeLine{      --help     display this help and exit}
\DoxyCodeLine{      --version  output version information and exit}
\DoxyCodeLine{}
\DoxyCodeLine{With no FILE, or when FILE is -, read standard input.}
\DoxyCodeLine{}
\DoxyCodeLine{Examples:}
\DoxyCodeLine{  cat f - g  Output f's contents, then standard input, then g's contents.}
\DoxyCodeLine{  cat        Copy standard input to standard output.}
\DoxyCodeLine{}
\DoxyCodeLine{GNU coreutils online help: <http://www.gnu.org/software/coreutils/>}
\DoxyCodeLine{Full documentation at: <http://www.gnu.org/software/coreutils/cat>}
\DoxyCodeLine{or available locally via: info '(coreutils) cat invocation'}
\DoxyCodeLine{}
\DoxyCodeLine{auto lo}
\DoxyCodeLine{iface lo inet loopback}
\DoxyCodeLine{    address 10.0.0.41/32}
\DoxyCodeLine{    address fd00::41/128}
\DoxyCodeLine{}
\DoxyCodeLine{auto eth0}
\DoxyCodeLine{iface eth0 inet static}
\DoxyCodeLine{    address 10.255.0.1}
\DoxyCodeLine{    netmask 255.255.0.0}
\DoxyCodeLine{    gateway 10.255.0.3}
\DoxyCodeLine{auto eth1}
\DoxyCodeLine{iface eth1 inet static}
\DoxyCodeLine{    address 192.168.0.254}
\DoxyCodeLine{    netmask 255.255.0.0}
\DoxyCodeLine{}
\DoxyCodeLine{auto eth2}
\DoxyCodeLine{iface eth2}
\DoxyCodeLine{}
\DoxyCodeLine{auto eth3}
\DoxyCodeLine{iface eth3}
\end{DoxyCode}
 \subsection*{ip tunnel show}

\#\#\# 
\begin{DoxyCode}{0}
\end{DoxyCode}
 
\begin{DoxyCode}{0}
\end{DoxyCode}
 