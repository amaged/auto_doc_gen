\subsection*{net show route static}

\#\#\# 
\begin{DoxyCode}{0}
\DoxyCodeLine{RIB entry for static}
\DoxyCodeLine{====================}
\end{DoxyCode}
 
\begin{DoxyCode}{0}
\DoxyCodeLine{The following commands contain keyword(s) 'static', 'route', 'show'}
\DoxyCodeLine{}
\DoxyCodeLine{    net show route (bgp|connected|kernel|ospf|ospf6|pim|rip|static|summary|supernets-only|table) [json]}
\DoxyCodeLine{    net show route vrf <text> (bgp|connected|kernel|ospf|ospf6|pim|rip|static|summary|supernets-only|table) [json]}
\end{DoxyCode}
 \subsection*{show ip rpf 230.\+0.\+0.\+0}

\#\#\# 
\begin{DoxyCode}{0}
\DoxyCodeLine{nohup: failed to run command ‘show’: No such file or directory}
\end{DoxyCode}
 
\begin{DoxyCode}{0}
\DoxyCodeLine{nohup: failed to run command ‘show’: No such file or directory}
\end{DoxyCode}
 \subsection*{ip route show}

\#\#\# 
\begin{DoxyCode}{0}
\DoxyCodeLine{default via 10.255.0.3 dev eth0  proto kernel }
\DoxyCodeLine{10.255.0.0/16 dev eth0  proto kernel  scope link  src 10.255.0.1 }
\DoxyCodeLine{172.17.0.0/16 dev docker0  proto kernel  scope link  src 172.17.0.1 dead linkdown }
\DoxyCodeLine{192.168.0.0/16 dev eth1  proto kernel  scope link  src 192.168.0.254 }
\end{DoxyCode}
 
\begin{DoxyCode}{0}
\DoxyCodeLine{Error: an inet prefix is expected rather than "--help".}
\DoxyCodeLine{}
\DoxyCodeLine{default via 10.255.0.3 dev eth0  proto kernel }
\DoxyCodeLine{10.255.0.0/16 dev eth0  proto kernel  scope link  src 10.255.0.1 }
\DoxyCodeLine{172.17.0.0/16 dev docker0  proto kernel  scope link  src 172.17.0.1 dead linkdown }
\DoxyCodeLine{192.168.0.0/16 dev eth1  proto kernel  scope link  src 192.168.0.254 }
\end{DoxyCode}
 \subsection*{sudo cl-\/resource-\/query}

\#\#\# 
\begin{DoxyCode}{0}
\DoxyCodeLine{route table mode is not available}
\end{DoxyCode}
 
\begin{DoxyCode}{0}
\DoxyCodeLine{usage: cl-resource-query [-h] [-k | -j]}
\DoxyCodeLine{}
\DoxyCodeLine{Cumulus Resource Reporting}
\DoxyCodeLine{}
\DoxyCodeLine{optional arguments:}
\DoxyCodeLine{  -h, --help       show this help message and exit}
\DoxyCodeLine{  -k, --key-value  Report key=value pairs}
\DoxyCodeLine{  -j, --json       Display JSON output}
\end{DoxyCode}
 \subsection*{cat /etc/cumulus/datapath/traffic.conf}

\#\#\# 
\begin{DoxyCode}{0}
\DoxyCodeLine{\#}
\DoxyCodeLine{\# /etc/cumulus/datapath/traffic.conf}
\DoxyCodeLine{\# Copyright 2014, 2015, 2016, 2017, Cumulus Networks, Inc.  All rights reserved.}
\DoxyCodeLine{\#}
\DoxyCodeLine{}
\DoxyCodeLine{\# packet header field used to determine the packet priority level}
\DoxyCodeLine{\# fields include \{802.1p, dscp\}}
\DoxyCodeLine{traffic.packet\_priority\_source\_set = [802.1p]}
\DoxyCodeLine{}
\DoxyCodeLine{\# packet priority source values assigned to each internal cos value}
\DoxyCodeLine{\# internal cos values \{cos\_0..cos\_7\}}
\DoxyCodeLine{\# (internal cos 3 has been reserved for CPU-generated traffic)}
\DoxyCodeLine{\#}
\DoxyCodeLine{\# 802.1p values = \{0..7\}}
\DoxyCodeLine{traffic.cos\_0.priority\_source.8021p = [0]}
\DoxyCodeLine{traffic.cos\_1.priority\_source.8021p = [1]}
\DoxyCodeLine{traffic.cos\_2.priority\_source.8021p = [2]}
\DoxyCodeLine{traffic.cos\_3.priority\_source.8021p = []}
\DoxyCodeLine{traffic.cos\_4.priority\_source.8021p = [3,4]}
\DoxyCodeLine{traffic.cos\_5.priority\_source.8021p = [5]}
\DoxyCodeLine{traffic.cos\_6.priority\_source.8021p = [6]}
\DoxyCodeLine{traffic.cos\_7.priority\_source.8021p = [7]}
\DoxyCodeLine{}
\DoxyCodeLine{\# dscp values = \{0..63\}}
\DoxyCodeLine{\#traffic.cos\_0.priority\_source.dscp = [0,1,2,3,4,5,6,7]}
\DoxyCodeLine{\#traffic.cos\_1.priority\_source.dscp = [8,9,10,11,12,13,14,15]}
\DoxyCodeLine{\#traffic.cos\_2.priority\_source.dscp = [16,17,18,19,20,21,22,23]}
\DoxyCodeLine{\#traffic.cos\_3.priority\_source.dscp = [24,25,26,27,28,29,30,31]}
\DoxyCodeLine{\#traffic.cos\_4.priority\_source.dscp = [32,33,34,35,36,37,38,39]}
\DoxyCodeLine{\#traffic.cos\_5.priority\_source.dscp = [40,41,42,43,44,45,46,47]}
\DoxyCodeLine{\#traffic.cos\_6.priority\_source.dscp = [48,49,50,51,52,53,54,55]}
\DoxyCodeLine{\#traffic.cos\_7.priority\_source.dscp = [56,57,58,59,60,61,62,63]}
\DoxyCodeLine{}
\DoxyCodeLine{\# remark packet priority value}
\DoxyCodeLine{\# fields include \{802.1p, dscp\}}
\DoxyCodeLine{traffic.packet\_priority\_remark\_set = []}
\DoxyCodeLine{}
\DoxyCodeLine{\# packet priority remark values assigned from each internal cos value}
\DoxyCodeLine{\# internal cos values \{cos\_0..cos\_7\}}
\DoxyCodeLine{\# (internal cos 3 has been reserved for CPU-generated traffic)}
\DoxyCodeLine{\#}
\DoxyCodeLine{\# 802.1p values = \{0..7\}}
\DoxyCodeLine{\#traffic.cos\_0.priority\_remark.8021p = [0]}
\DoxyCodeLine{\#traffic.cos\_1.priority\_remark.8021p = [1]}
\DoxyCodeLine{\#traffic.cos\_2.priority\_remark.8021p = [2]}
\DoxyCodeLine{\#traffic.cos\_3.priority\_remark.8021p = [3]}
\DoxyCodeLine{\#traffic.cos\_4.priority\_remark.8021p = [4]}
\DoxyCodeLine{\#traffic.cos\_5.priority\_remark.8021p = [5]}
\DoxyCodeLine{\#traffic.cos\_6.priority\_remark.8021p = [6]}
\DoxyCodeLine{\#traffic.cos\_7.priority\_remark.8021p = [7]}
\DoxyCodeLine{}
\DoxyCodeLine{\# dscp values = \{0..63\}}
\DoxyCodeLine{\#traffic.cos\_0.priority\_remark.dscp = [0]}
\DoxyCodeLine{\#traffic.cos\_1.priority\_remark.dscp = [8]}
\DoxyCodeLine{\#traffic.cos\_2.priority\_remark.dscp = [16]}
\DoxyCodeLine{\#traffic.cos\_3.priority\_remark.dscp = [24]}
\DoxyCodeLine{\#traffic.cos\_4.priority\_remark.dscp = [32]}
\DoxyCodeLine{\#traffic.cos\_5.priority\_remark.dscp = [40]}
\DoxyCodeLine{\#traffic.cos\_6.priority\_remark.dscp = [48]}
\DoxyCodeLine{\#traffic.cos\_7.priority\_remark.dscp = [56]}
\DoxyCodeLine{}
\DoxyCodeLine{\# source.port\_group\_list = [source\_port\_group]}
\DoxyCodeLine{\# source.source\_port\_group.packet\_priority\_source\_set = [dscp]}
\DoxyCodeLine{\# source.source\_port\_group.port\_set = swp1-swp4,swp6}
\DoxyCodeLine{\# source.source\_port\_group.cos\_0.priority\_source.dscp = [0,1,2,3,4,5,6,7]}
\DoxyCodeLine{\# source.source\_port\_group.cos\_1.priority\_source.dscp = [8,9,10,11,12,13,14,15]}
\DoxyCodeLine{\# source.source\_port\_group.cos\_2.priority\_source.dscp = [16,17,18,19,20,21,22,23]}
\DoxyCodeLine{\# source.source\_port\_group.cos\_3.priority\_source.dscp = [24,25,26,27,28,29,30,31]}
\DoxyCodeLine{\# source.source\_port\_group.cos\_4.priority\_source.dscp = [32,33,34,35,36,37,38,39]}
\DoxyCodeLine{\# source.source\_port\_group.cos\_5.priority\_source.dscp = [40,41,42,43,44,45,46,47]}
\DoxyCodeLine{\# source.source\_port\_group.cos\_6.priority\_source.dscp = [48,49,50,51,52,53,54,55]}
\DoxyCodeLine{\# source.source\_port\_group.cos\_7.priority\_source.dscp = [56,57,58,59,60,61,62,63]}
\DoxyCodeLine{}
\DoxyCodeLine{\# remark.port\_group\_list = [remark\_port\_group]}
\DoxyCodeLine{\# remark.remark\_port\_group.packet\_priority\_remark\_set = [dscp]}
\DoxyCodeLine{\# remark.remark\_port\_group.port\_set = swp1-swp4,swp6}
\DoxyCodeLine{\# remark.remark\_port\_group.cos\_0.priority\_remark.dscp = [0]}
\DoxyCodeLine{\# remark.remark\_port\_group.cos\_1.priority\_remark.dscp = [8]}
\DoxyCodeLine{\# remark.remark\_port\_group.cos\_2.priority\_remark.dscp = [16]}
\DoxyCodeLine{\# remark.remark\_port\_group.cos\_3.priority\_remark.dscp = [24]}
\DoxyCodeLine{\# remark.remark\_port\_group.cos\_4.priority\_remark.dscp = [32]}
\DoxyCodeLine{\# remark.remark\_port\_group.cos\_5.priority\_remark.dscp = [40]}
\DoxyCodeLine{\# remark.remark\_port\_group.cos\_6.priority\_remark.dscp = [48]}
\DoxyCodeLine{\# remark.remark\_port\_group.cos\_7.priority\_remark.dscp = [56]}
\DoxyCodeLine{}
\DoxyCodeLine{\# priority groups}
\DoxyCodeLine{traffic.priority\_group\_list = [control, service, bulk]}
\DoxyCodeLine{}
\DoxyCodeLine{\# internal cos values assigned to each priority group}
\DoxyCodeLine{\# each cos value should be assigned exactly once}
\DoxyCodeLine{\# internal cos values \{0..7\}}
\DoxyCodeLine{priority\_group.control.cos\_list = [7]}
\DoxyCodeLine{priority\_group.service.cos\_list = [2]}
\DoxyCodeLine{priority\_group.bulk.cos\_list = [0,1,3,4,5,6]}
\DoxyCodeLine{}
\DoxyCodeLine{\# to configure priority flow control on a group of ports:}
\DoxyCodeLine{\# -- assign cos value(s) to the cos list}
\DoxyCodeLine{\# -- add or replace a port group names in the port group list}
\DoxyCodeLine{\# -- for each port group in the list}
\DoxyCodeLine{\#    -- populate the port set, e.g.}
\DoxyCodeLine{\#       swp1-swp4,swp8,swp50s0-swp50s3}
\DoxyCodeLine{\#    -- set a PFC buffer size in bytes for each port in the group}
\DoxyCodeLine{\#    -- set the xoff byte limit (buffer limit that triggers PFC frames transmit to start)}
\DoxyCodeLine{\#    -- set the xon byte delta (buffer limit that triggers PFC frames transmit to stop)}
\DoxyCodeLine{\#    -- enable PFC frame transmit and/or PFC frame receive}
\DoxyCodeLine{}
\DoxyCodeLine{\# priority flow control}
\DoxyCodeLine{\# pfc.port\_group\_list = [pfc\_port\_group]}
\DoxyCodeLine{\# pfc.pfc\_port\_group.cos\_list = []}
\DoxyCodeLine{\# pfc.pfc\_port\_group.port\_set = swp1-swp4,swp6}
\DoxyCodeLine{\# pfc.pfc\_port\_group.port\_buffer\_bytes = 25000}
\DoxyCodeLine{\# pfc.pfc\_port\_group.xoff\_size = 10000}
\DoxyCodeLine{\# pfc.pfc\_port\_group.xon\_delta = 2000}
\DoxyCodeLine{\# pfc.pfc\_port\_group.tx\_enable = true}
\DoxyCodeLine{\# pfc.pfc\_port\_group.rx\_enable = true}
\DoxyCodeLine{}
\DoxyCodeLine{\# to configure pause on a group of ports:}
\DoxyCodeLine{\# -- add or replace port group names in the port group list}
\DoxyCodeLine{\# -- for each port group in the list}
\DoxyCodeLine{\#    -- populate the port set, e.g.}
\DoxyCodeLine{\#       swp1-swp4,swp8,swp50s0-swp50s3}
\DoxyCodeLine{\#    -- set a pause buffer size in bytes for each port}
\DoxyCodeLine{\#    -- set the xoff byte limit (buffer limit that triggers pause frames transmit to start)}
\DoxyCodeLine{\#    -- set the xon byte delta (buffer limit that triggers pause frames transmit to stop)}
\DoxyCodeLine{\#    -- enable pause frame transmit and/or pause frame receive}
\DoxyCodeLine{}
\DoxyCodeLine{\# link pause}
\DoxyCodeLine{\# link\_pause.port\_group\_list = [pause\_port\_group]}
\DoxyCodeLine{\# link\_pause.pause\_port\_group.port\_set = swp1-swp4,swp6}
\DoxyCodeLine{\# link\_pause.pause\_port\_group.port\_buffer\_bytes = 25000}
\DoxyCodeLine{\# link\_pause.pause\_port\_group.xoff\_size = 10000}
\DoxyCodeLine{\# link\_pause.pause\_port\_group.xon\_delta = 2000}
\DoxyCodeLine{\# link\_pause.pause\_port\_group.rx\_enable = true}
\DoxyCodeLine{\# link\_pause.pause\_port\_group.tx\_enable = true}
\DoxyCodeLine{}
\DoxyCodeLine{\# Explicit Congestion Notification}
\DoxyCodeLine{\# to configure ECN and RED on a group of ports:}
\DoxyCodeLine{\# -- add or replace port group names in the port group list}
\DoxyCodeLine{\# -- assign cos value(s) to the cos list}
\DoxyCodeLine{\# -- for each port group in the list}
\DoxyCodeLine{\#    -- populate the port set, e.g.}
\DoxyCodeLine{\#       swp1-swp4,swp8,swp50s0-swp50s3}
\DoxyCodeLine{\# -- to enable RED requires the latest traffic.conf}
\DoxyCodeLine{\# ecn\_red.port\_group\_list = [ecn\_red\_port\_group]}
\DoxyCodeLine{\# ecn\_red.ecn\_red\_port\_group.cos\_list = []}
\DoxyCodeLine{\# ecn\_red.ecn\_red\_port\_group.port\_set = swp1-swp4,swp6}
\DoxyCodeLine{\# ecn\_red.ecn\_red\_port\_group.ecn\_enable = true}
\DoxyCodeLine{\# ecn\_red.ecn\_red\_port\_group.red\_enable = false}
\DoxyCodeLine{\# ecn\_red.ecn\_red\_port\_group.min\_threshold\_bytes = 40000}
\DoxyCodeLine{\# ecn\_red.ecn\_red\_port\_group.max\_threshold\_bytes = 200000}
\DoxyCodeLine{\# ecn\_red.ecn\_red\_port\_group.probability = 100}
\DoxyCodeLine{}
\DoxyCodeLine{\# scheduling algorithm: algorithm values = \{dwrr\}}
\DoxyCodeLine{scheduling.algorithm = dwrr}
\DoxyCodeLine{}
\DoxyCodeLine{\# traffic group scheduling weight}
\DoxyCodeLine{\# weight values = \{0..127\}}
\DoxyCodeLine{\# '0' indicates strict priority}
\DoxyCodeLine{priority\_group.control.weight = 0}
\DoxyCodeLine{priority\_group.service.weight = 32}
\DoxyCodeLine{priority\_group.bulk.weight = 16}
\DoxyCodeLine{}
\DoxyCodeLine{\# To turn on/off Denial of service (DOS) prevention checks}
\DoxyCodeLine{dos\_enable = false}
\DoxyCodeLine{}
\DoxyCodeLine{\# Cut-through is disabled by default on all chips with the exception of}
\DoxyCodeLine{\# Spectrum.  On Spectrum cut-through cannot be disabled.}
\DoxyCodeLine{\#cut\_through\_enable = false}
\DoxyCodeLine{}
\DoxyCodeLine{\# Enable resilient hashing}
\DoxyCodeLine{\#resilient\_hash\_enable = FALSE}
\DoxyCodeLine{}
\DoxyCodeLine{\# Resilient hashing flowset entries per ECMP group}
\DoxyCodeLine{\# Valid values - 64, 128, 256, 512, 1024}
\DoxyCodeLine{\#resilient\_hash\_entries\_ecmp = 128}
\DoxyCodeLine{}
\DoxyCodeLine{\# Enable symmetric hashing}
\DoxyCodeLine{\#symmetric\_hash\_enable = TRUE}
\DoxyCodeLine{}
\DoxyCodeLine{\# Set sflow/sample ingress cpu packet rate and burst in packets/sec}
\DoxyCodeLine{\# Values: \{0..16384\}}
\DoxyCodeLine{\#sflow.rate = 16384}
\DoxyCodeLine{\#sflow.burst = 16384}
\DoxyCodeLine{}
\DoxyCodeLine{\#Specify the maximum number of paths per route entry.}
\DoxyCodeLine{\#  Maximum paths supported is 200.}
\DoxyCodeLine{\#  Default value 0 takes the number of physical ports as the max path size.}
\DoxyCodeLine{\#ecmp\_max\_paths = 0}
\DoxyCodeLine{}
\DoxyCodeLine{\#Specify the hash seed for Equal cost multipath entries}
\DoxyCodeLine{\# Default value 0}
\DoxyCodeLine{\# Value Rang: \{0..4294967295\}}
\DoxyCodeLine{\#ecmp\_hash\_seed = 42}
\DoxyCodeLine{}
\DoxyCodeLine{\# Specify the forwarding table resource allocation profile, applicable}
\DoxyCodeLine{\# only on platforms that support universal forwarding resources.}
\DoxyCodeLine{\#}
\DoxyCodeLine{\# /usr/cumulus/sbin/cl-resource-query reports the allocated table sizes}
\DoxyCodeLine{\# based on the profile setting.}
\DoxyCodeLine{\#}
\DoxyCodeLine{\#   Values: one of \{'default', 'l2-heavy', 'v4-lpm-heavy', 'v6-lpm-heavy',}
\DoxyCodeLine{\#                   'ipmc-heavy'\}}
\DoxyCodeLine{\#   Default value: 'default'}
\DoxyCodeLine{\#   Note: some devices may support more modes, please consult user}
\DoxyCodeLine{\#         guide for more details}
\DoxyCodeLine{\#}
\DoxyCodeLine{\#forwarding\_table.profile = default}
\end{DoxyCode}
 
\begin{DoxyCode}{0}
\DoxyCodeLine{Usage: cat [OPTION]... [FILE]...}
\DoxyCodeLine{Concatenate FILE(s), or standard input, to standard output.}
\DoxyCodeLine{}
\DoxyCodeLine{  -A, --show-all           equivalent to -vET}
\DoxyCodeLine{  -b, --number-nonblank    number nonempty output lines, overrides -n}
\DoxyCodeLine{  -e                       equivalent to -vE}
\DoxyCodeLine{  -E, --show-ends          display \$ at end of each line}
\DoxyCodeLine{  -n, --number             number all output lines}
\DoxyCodeLine{  -s, --squeeze-blank      suppress repeated empty output lines}
\DoxyCodeLine{  -t                       equivalent to -vT}
\DoxyCodeLine{  -T, --show-tabs          display TAB characters as \string^I}
\DoxyCodeLine{  -u                       (ignored)}
\DoxyCodeLine{  -v, --show-nonprinting   use \string^ and M- notation, except for LFD and TAB}
\DoxyCodeLine{      --help     display this help and exit}
\DoxyCodeLine{      --version  output version information and exit}
\DoxyCodeLine{}
\DoxyCodeLine{With no FILE, or when FILE is -, read standard input.}
\DoxyCodeLine{}
\DoxyCodeLine{Examples:}
\DoxyCodeLine{  cat f - g  Output f's contents, then standard input, then g's contents.}
\DoxyCodeLine{  cat        Copy standard input to standard output.}
\DoxyCodeLine{}
\DoxyCodeLine{GNU coreutils online help: <http://www.gnu.org/software/coreutils/>}
\DoxyCodeLine{Full documentation at: <http://www.gnu.org/software/coreutils/cat>}
\DoxyCodeLine{or available locally via: info '(coreutils) cat invocation'}
\DoxyCodeLine{}
\DoxyCodeLine{\#}
\DoxyCodeLine{\# /etc/cumulus/datapath/traffic.conf}
\DoxyCodeLine{\# Copyright 2014, 2015, 2016, 2017, Cumulus Networks, Inc.  All rights reserved.}
\DoxyCodeLine{\#}
\DoxyCodeLine{}
\DoxyCodeLine{\# packet header field used to determine the packet priority level}
\DoxyCodeLine{\# fields include \{802.1p, dscp\}}
\DoxyCodeLine{traffic.packet\_priority\_source\_set = [802.1p]}
\DoxyCodeLine{}
\DoxyCodeLine{\# packet priority source values assigned to each internal cos value}
\DoxyCodeLine{\# internal cos values \{cos\_0..cos\_7\}}
\DoxyCodeLine{\# (internal cos 3 has been reserved for CPU-generated traffic)}
\DoxyCodeLine{\#}
\DoxyCodeLine{\# 802.1p values = \{0..7\}}
\DoxyCodeLine{traffic.cos\_0.priority\_source.8021p = [0]}
\DoxyCodeLine{traffic.cos\_1.priority\_source.8021p = [1]}
\DoxyCodeLine{traffic.cos\_2.priority\_source.8021p = [2]}
\DoxyCodeLine{traffic.cos\_3.priority\_source.8021p = []}
\DoxyCodeLine{traffic.cos\_4.priority\_source.8021p = [3,4]}
\DoxyCodeLine{traffic.cos\_5.priority\_source.8021p = [5]}
\DoxyCodeLine{traffic.cos\_6.priority\_source.8021p = [6]}
\DoxyCodeLine{traffic.cos\_7.priority\_source.8021p = [7]}
\DoxyCodeLine{}
\DoxyCodeLine{\# dscp values = \{0..63\}}
\DoxyCodeLine{\#traffic.cos\_0.priority\_source.dscp = [0,1,2,3,4,5,6,7]}
\DoxyCodeLine{\#traffic.cos\_1.priority\_source.dscp = [8,9,10,11,12,13,14,15]}
\DoxyCodeLine{\#traffic.cos\_2.priority\_source.dscp = [16,17,18,19,20,21,22,23]}
\DoxyCodeLine{\#traffic.cos\_3.priority\_source.dscp = [24,25,26,27,28,29,30,31]}
\DoxyCodeLine{\#traffic.cos\_4.priority\_source.dscp = [32,33,34,35,36,37,38,39]}
\DoxyCodeLine{\#traffic.cos\_5.priority\_source.dscp = [40,41,42,43,44,45,46,47]}
\DoxyCodeLine{\#traffic.cos\_6.priority\_source.dscp = [48,49,50,51,52,53,54,55]}
\DoxyCodeLine{\#traffic.cos\_7.priority\_source.dscp = [56,57,58,59,60,61,62,63]}
\DoxyCodeLine{}
\DoxyCodeLine{\# remark packet priority value}
\DoxyCodeLine{\# fields include \{802.1p, dscp\}}
\DoxyCodeLine{traffic.packet\_priority\_remark\_set = []}
\DoxyCodeLine{}
\DoxyCodeLine{\# packet priority remark values assigned from each internal cos value}
\DoxyCodeLine{\# internal cos values \{cos\_0..cos\_7\}}
\DoxyCodeLine{\# (internal cos 3 has been reserved for CPU-generated traffic)}
\DoxyCodeLine{\#}
\DoxyCodeLine{\# 802.1p values = \{0..7\}}
\DoxyCodeLine{\#traffic.cos\_0.priority\_remark.8021p = [0]}
\DoxyCodeLine{\#traffic.cos\_1.priority\_remark.8021p = [1]}
\DoxyCodeLine{\#traffic.cos\_2.priority\_remark.8021p = [2]}
\DoxyCodeLine{\#traffic.cos\_3.priority\_remark.8021p = [3]}
\DoxyCodeLine{\#traffic.cos\_4.priority\_remark.8021p = [4]}
\DoxyCodeLine{\#traffic.cos\_5.priority\_remark.8021p = [5]}
\DoxyCodeLine{\#traffic.cos\_6.priority\_remark.8021p = [6]}
\DoxyCodeLine{\#traffic.cos\_7.priority\_remark.8021p = [7]}
\DoxyCodeLine{}
\DoxyCodeLine{\# dscp values = \{0..63\}}
\DoxyCodeLine{\#traffic.cos\_0.priority\_remark.dscp = [0]}
\DoxyCodeLine{\#traffic.cos\_1.priority\_remark.dscp = [8]}
\DoxyCodeLine{\#traffic.cos\_2.priority\_remark.dscp = [16]}
\DoxyCodeLine{\#traffic.cos\_3.priority\_remark.dscp = [24]}
\DoxyCodeLine{\#traffic.cos\_4.priority\_remark.dscp = [32]}
\DoxyCodeLine{\#traffic.cos\_5.priority\_remark.dscp = [40]}
\DoxyCodeLine{\#traffic.cos\_6.priority\_remark.dscp = [48]}
\DoxyCodeLine{\#traffic.cos\_7.priority\_remark.dscp = [56]}
\DoxyCodeLine{}
\DoxyCodeLine{\# source.port\_group\_list = [source\_port\_group]}
\DoxyCodeLine{\# source.source\_port\_group.packet\_priority\_source\_set = [dscp]}
\DoxyCodeLine{\# source.source\_port\_group.port\_set = swp1-swp4,swp6}
\DoxyCodeLine{\# source.source\_port\_group.cos\_0.priority\_source.dscp = [0,1,2,3,4,5,6,7]}
\DoxyCodeLine{\# source.source\_port\_group.cos\_1.priority\_source.dscp = [8,9,10,11,12,13,14,15]}
\DoxyCodeLine{\# source.source\_port\_group.cos\_2.priority\_source.dscp = [16,17,18,19,20,21,22,23]}
\DoxyCodeLine{\# source.source\_port\_group.cos\_3.priority\_source.dscp = [24,25,26,27,28,29,30,31]}
\DoxyCodeLine{\# source.source\_port\_group.cos\_4.priority\_source.dscp = [32,33,34,35,36,37,38,39]}
\DoxyCodeLine{\# source.source\_port\_group.cos\_5.priority\_source.dscp = [40,41,42,43,44,45,46,47]}
\DoxyCodeLine{\# source.source\_port\_group.cos\_6.priority\_source.dscp = [48,49,50,51,52,53,54,55]}
\DoxyCodeLine{\# source.source\_port\_group.cos\_7.priority\_source.dscp = [56,57,58,59,60,61,62,63]}
\DoxyCodeLine{}
\DoxyCodeLine{\# remark.port\_group\_list = [remark\_port\_group]}
\DoxyCodeLine{\# remark.remark\_port\_group.packet\_priority\_remark\_set = [dscp]}
\DoxyCodeLine{\# remark.remark\_port\_group.port\_set = swp1-swp4,swp6}
\DoxyCodeLine{\# remark.remark\_port\_group.cos\_0.priority\_remark.dscp = [0]}
\DoxyCodeLine{\# remark.remark\_port\_group.cos\_1.priority\_remark.dscp = [8]}
\DoxyCodeLine{\# remark.remark\_port\_group.cos\_2.priority\_remark.dscp = [16]}
\DoxyCodeLine{\# remark.remark\_port\_group.cos\_3.priority\_remark.dscp = [24]}
\DoxyCodeLine{\# remark.remark\_port\_group.cos\_4.priority\_remark.dscp = [32]}
\DoxyCodeLine{\# remark.remark\_port\_group.cos\_5.priority\_remark.dscp = [40]}
\DoxyCodeLine{\# remark.remark\_port\_group.cos\_6.priority\_remark.dscp = [48]}
\DoxyCodeLine{\# remark.remark\_port\_group.cos\_7.priority\_remark.dscp = [56]}
\DoxyCodeLine{}
\DoxyCodeLine{\# priority groups}
\DoxyCodeLine{traffic.priority\_group\_list = [control, service, bulk]}
\DoxyCodeLine{}
\DoxyCodeLine{\# internal cos values assigned to each priority group}
\DoxyCodeLine{\# each cos value should be assigned exactly once}
\DoxyCodeLine{\# internal cos values \{0..7\}}
\DoxyCodeLine{priority\_group.control.cos\_list = [7]}
\DoxyCodeLine{priority\_group.service.cos\_list = [2]}
\DoxyCodeLine{priority\_group.bulk.cos\_list = [0,1,3,4,5,6]}
\DoxyCodeLine{}
\DoxyCodeLine{\# to configure priority flow control on a group of ports:}
\DoxyCodeLine{\# -- assign cos value(s) to the cos list}
\DoxyCodeLine{\# -- add or replace a port group names in the port group list}
\DoxyCodeLine{\# -- for each port group in the list}
\DoxyCodeLine{\#    -- populate the port set, e.g.}
\DoxyCodeLine{\#       swp1-swp4,swp8,swp50s0-swp50s3}
\DoxyCodeLine{\#    -- set a PFC buffer size in bytes for each port in the group}
\DoxyCodeLine{\#    -- set the xoff byte limit (buffer limit that triggers PFC frames transmit to start)}
\DoxyCodeLine{\#    -- set the xon byte delta (buffer limit that triggers PFC frames transmit to stop)}
\DoxyCodeLine{\#    -- enable PFC frame transmit and/or PFC frame receive}
\DoxyCodeLine{}
\DoxyCodeLine{\# priority flow control}
\DoxyCodeLine{\# pfc.port\_group\_list = [pfc\_port\_group]}
\DoxyCodeLine{\# pfc.pfc\_port\_group.cos\_list = []}
\DoxyCodeLine{\# pfc.pfc\_port\_group.port\_set = swp1-swp4,swp6}
\DoxyCodeLine{\# pfc.pfc\_port\_group.port\_buffer\_bytes = 25000}
\DoxyCodeLine{\# pfc.pfc\_port\_group.xoff\_size = 10000}
\DoxyCodeLine{\# pfc.pfc\_port\_group.xon\_delta = 2000}
\DoxyCodeLine{\# pfc.pfc\_port\_group.tx\_enable = true}
\DoxyCodeLine{\# pfc.pfc\_port\_group.rx\_enable = true}
\DoxyCodeLine{}
\DoxyCodeLine{\# to configure pause on a group of ports:}
\DoxyCodeLine{\# -- add or replace port group names in the port group list}
\DoxyCodeLine{\# -- for each port group in the list}
\DoxyCodeLine{\#    -- populate the port set, e.g.}
\DoxyCodeLine{\#       swp1-swp4,swp8,swp50s0-swp50s3}
\DoxyCodeLine{\#    -- set a pause buffer size in bytes for each port}
\DoxyCodeLine{\#    -- set the xoff byte limit (buffer limit that triggers pause frames transmit to start)}
\DoxyCodeLine{\#    -- set the xon byte delta (buffer limit that triggers pause frames transmit to stop)}
\DoxyCodeLine{\#    -- enable pause frame transmit and/or pause frame receive}
\DoxyCodeLine{}
\DoxyCodeLine{\# link pause}
\DoxyCodeLine{\# link\_pause.port\_group\_list = [pause\_port\_group]}
\DoxyCodeLine{\# link\_pause.pause\_port\_group.port\_set = swp1-swp4,swp6}
\DoxyCodeLine{\# link\_pause.pause\_port\_group.port\_buffer\_bytes = 25000}
\DoxyCodeLine{\# link\_pause.pause\_port\_group.xoff\_size = 10000}
\DoxyCodeLine{\# link\_pause.pause\_port\_group.xon\_delta = 2000}
\DoxyCodeLine{\# link\_pause.pause\_port\_group.rx\_enable = true}
\DoxyCodeLine{\# link\_pause.pause\_port\_group.tx\_enable = true}
\DoxyCodeLine{}
\DoxyCodeLine{\# Explicit Congestion Notification}
\DoxyCodeLine{\# to configure ECN and RED on a group of ports:}
\DoxyCodeLine{\# -- add or replace port group names in the port group list}
\DoxyCodeLine{\# -- assign cos value(s) to the cos list}
\DoxyCodeLine{\# -- for each port group in the list}
\DoxyCodeLine{\#    -- populate the port set, e.g.}
\DoxyCodeLine{\#       swp1-swp4,swp8,swp50s0-swp50s3}
\DoxyCodeLine{\# -- to enable RED requires the latest traffic.conf}
\DoxyCodeLine{\# ecn\_red.port\_group\_list = [ecn\_red\_port\_group]}
\DoxyCodeLine{\# ecn\_red.ecn\_red\_port\_group.cos\_list = []}
\DoxyCodeLine{\# ecn\_red.ecn\_red\_port\_group.port\_set = swp1-swp4,swp6}
\DoxyCodeLine{\# ecn\_red.ecn\_red\_port\_group.ecn\_enable = true}
\DoxyCodeLine{\# ecn\_red.ecn\_red\_port\_group.red\_enable = false}
\DoxyCodeLine{\# ecn\_red.ecn\_red\_port\_group.min\_threshold\_bytes = 40000}
\DoxyCodeLine{\# ecn\_red.ecn\_red\_port\_group.max\_threshold\_bytes = 200000}
\DoxyCodeLine{\# ecn\_red.ecn\_red\_port\_group.probability = 100}
\DoxyCodeLine{}
\DoxyCodeLine{\# scheduling algorithm: algorithm values = \{dwrr\}}
\DoxyCodeLine{scheduling.algorithm = dwrr}
\DoxyCodeLine{}
\DoxyCodeLine{\# traffic group scheduling weight}
\DoxyCodeLine{\# weight values = \{0..127\}}
\DoxyCodeLine{\# '0' indicates strict priority}
\DoxyCodeLine{priority\_group.control.weight = 0}
\DoxyCodeLine{priority\_group.service.weight = 32}
\DoxyCodeLine{priority\_group.bulk.weight = 16}
\DoxyCodeLine{}
\DoxyCodeLine{\# To turn on/off Denial of service (DOS) prevention checks}
\DoxyCodeLine{dos\_enable = false}
\DoxyCodeLine{}
\DoxyCodeLine{\# Cut-through is disabled by default on all chips with the exception of}
\DoxyCodeLine{\# Spectrum.  On Spectrum cut-through cannot be disabled.}
\DoxyCodeLine{\#cut\_through\_enable = false}
\DoxyCodeLine{}
\DoxyCodeLine{\# Enable resilient hashing}
\DoxyCodeLine{\#resilient\_hash\_enable = FALSE}
\DoxyCodeLine{}
\DoxyCodeLine{\# Resilient hashing flowset entries per ECMP group}
\DoxyCodeLine{\# Valid values - 64, 128, 256, 512, 1024}
\DoxyCodeLine{\#resilient\_hash\_entries\_ecmp = 128}
\DoxyCodeLine{}
\DoxyCodeLine{\# Enable symmetric hashing}
\DoxyCodeLine{\#symmetric\_hash\_enable = TRUE}
\DoxyCodeLine{}
\DoxyCodeLine{\# Set sflow/sample ingress cpu packet rate and burst in packets/sec}
\DoxyCodeLine{\# Values: \{0..16384\}}
\DoxyCodeLine{\#sflow.rate = 16384}
\DoxyCodeLine{\#sflow.burst = 16384}
\DoxyCodeLine{}
\DoxyCodeLine{\#Specify the maximum number of paths per route entry.}
\DoxyCodeLine{\#  Maximum paths supported is 200.}
\DoxyCodeLine{\#  Default value 0 takes the number of physical ports as the max path size.}
\DoxyCodeLine{\#ecmp\_max\_paths = 0}
\DoxyCodeLine{}
\DoxyCodeLine{\#Specify the hash seed for Equal cost multipath entries}
\DoxyCodeLine{\# Default value 0}
\DoxyCodeLine{\# Value Rang: \{0..4294967295\}}
\DoxyCodeLine{\#ecmp\_hash\_seed = 42}
\DoxyCodeLine{}
\DoxyCodeLine{\# Specify the forwarding table resource allocation profile, applicable}
\DoxyCodeLine{\# only on platforms that support universal forwarding resources.}
\DoxyCodeLine{\#}
\DoxyCodeLine{\# /usr/cumulus/sbin/cl-resource-query reports the allocated table sizes}
\DoxyCodeLine{\# based on the profile setting.}
\DoxyCodeLine{\#}
\DoxyCodeLine{\#   Values: one of \{'default', 'l2-heavy', 'v4-lpm-heavy', 'v6-lpm-heavy',}
\DoxyCodeLine{\#                   'ipmc-heavy'\}}
\DoxyCodeLine{\#   Default value: 'default'}
\DoxyCodeLine{\#   Note: some devices may support more modes, please consult user}
\DoxyCodeLine{\#         guide for more details}
\DoxyCodeLine{\#}
\DoxyCodeLine{\#forwarding\_table.profile = default}
\end{DoxyCode}
 \subsection*{cat /usr/lib/python2.\+7/dist-\/packages/cumulus/\+\_\+\+\_\+chip\+\_\+config/mlx/datapath.\+conf}

\#\#\# 
\begin{DoxyCode}{0}
\DoxyCodeLine{\#}
\DoxyCodeLine{\# Default datapath configuration for Mellanox Spectrum chip}
\DoxyCodeLine{\# Copyright 2016, 2017, Cumulus Networks, Inc.  All rights reserved.}
\DoxyCodeLine{\#}
\DoxyCodeLine{}
\DoxyCodeLine{\# priority group ID assigned to each priority group}
\DoxyCodeLine{priority\_group.control.id = 7}
\DoxyCodeLine{priority\_group.service.id = 2}
\DoxyCodeLine{priority\_group.bulk.id = 0}
\DoxyCodeLine{}
\DoxyCodeLine{\# all priority groups share a service pool on Spectrum}
\DoxyCodeLine{\# service pools assigned to each priority group}
\DoxyCodeLine{priority\_group.control.service\_pool = 0}
\DoxyCodeLine{priority\_group.service.service\_pool = 0}
\DoxyCodeLine{priority\_group.bulk.service\_pool = 0}
\DoxyCodeLine{}
\DoxyCodeLine{\# --- ingress buffer space allocations ---}
\DoxyCodeLine{\#}
\DoxyCodeLine{\# total buffer}
\DoxyCodeLine{\#  - ingress minimum buffer allocations}
\DoxyCodeLine{\#  - ingress service pool buffer allocations}
\DoxyCodeLine{\#  - priority group ingress headroom allocations}
\DoxyCodeLine{\#  - ingress global headroom allocations}
\DoxyCodeLine{\#  = total ingress shared buffer size}
\DoxyCodeLine{}
\DoxyCodeLine{\# ingress service pool buffer allocation: percent of total buffer}
\DoxyCodeLine{\# If a service pool has no priority groups, the buffer is added}
\DoxyCodeLine{\# to the shared buffer space.}
\DoxyCodeLine{ingress\_service\_pool.0.percent = 100.0  \# all priority groups}
\DoxyCodeLine{}
\DoxyCodeLine{\# priority group minimum buffer allocation: percent of total buffer cells}
\DoxyCodeLine{\# priority group shared buffer allocation: percent of total shared buffer size}
\DoxyCodeLine{\# if a priority group has no packet priority values assigned to it, the buffers will not be allocated}
\DoxyCodeLine{}
\DoxyCodeLine{priority\_group.control.ingress\_buffer.min\_percent     =  4.0}
\DoxyCodeLine{priority\_group.control.ingress\_buffer.shared\_percent  = 20.0}
\DoxyCodeLine{}
\DoxyCodeLine{priority\_group.service.ingress\_buffer.min\_percent     =  2.0}
\DoxyCodeLine{priority\_group.service.ingress\_buffer.shared\_percent  =  3.0}
\DoxyCodeLine{}
\DoxyCodeLine{priority\_group.bulk.ingress\_buffer.min\_percent        = 18.0}
\DoxyCodeLine{priority\_group.bulk.ingress\_buffer.shared\_percent     = 15.0}
\DoxyCodeLine{}
\DoxyCodeLine{\# --- egress buffer space allocations ---}
\DoxyCodeLine{\#}
\DoxyCodeLine{\# total egress buffer}
\DoxyCodeLine{\#  - minimum buffer allocations}
\DoxyCodeLine{\#  = total service pool buffer size}
\DoxyCodeLine{\#}
\DoxyCodeLine{\# Service pool buffer allocation: percent of total}
\DoxyCodeLine{\# buffer size.}
\DoxyCodeLine{egress\_service\_pool.0.percent = 100.0   \# all priority groups, UC and MC}
\DoxyCodeLine{}
\DoxyCodeLine{\# Front panel port egress buffer limits enforced for each}
\DoxyCodeLine{\# priority group.}
\DoxyCodeLine{\# Unlimited egress buffers not supported on Spectrum.}
\DoxyCodeLine{priority\_group.control.unlimited\_egress\_buffer  = false}
\DoxyCodeLine{priority\_group.service.unlimited\_egress\_buffer  = false}
\DoxyCodeLine{priority\_group.bulk.unlimited\_egress\_buffer     = false}
\DoxyCodeLine{}
\DoxyCodeLine{\#}
\DoxyCodeLine{\# if a priority group has no cos values assigned to it, the buffers will not be allocated}
\DoxyCodeLine{\#}
\DoxyCodeLine{}
\DoxyCodeLine{\# priority group minimum buffer allocation: percent of total buffer}
\DoxyCodeLine{priority\_group.bulk.egress\_buffer.uc.min\_percent      =  6.0}
\DoxyCodeLine{priority\_group.service.egress\_buffer.uc.min\_percent   =  2.0}
\DoxyCodeLine{priority\_group.control.egress\_buffer.uc.min\_percent   =  4.0}
\DoxyCodeLine{}
\DoxyCodeLine{priority\_group.bulk.egress\_buffer.mc.min\_percent      =  5.0}
\DoxyCodeLine{priority\_group.service.egress\_buffer.mc.min\_percent   =  2.0}
\DoxyCodeLine{priority\_group.control.egress\_buffer.mc.min\_percent   =  3.0}
\DoxyCodeLine{}
\DoxyCodeLine{\# Priority group service pool buffer limits: percent of the}
\DoxyCodeLine{\# assigned service pool.}
\DoxyCodeLine{priority\_group.bulk.egress\_buffer.uc.sp\_percent       = 40.0}
\DoxyCodeLine{priority\_group.service.egress\_buffer.uc.sp\_percent    =  5.0}
\DoxyCodeLine{priority\_group.control.egress\_buffer.uc.sp\_percent    = 30.0}
\DoxyCodeLine{}
\DoxyCodeLine{priority\_group.bulk.egress\_buffer.mc.sp\_percent       = 40.0}
\DoxyCodeLine{priority\_group.service.egress\_buffer.mc.sp\_percent    =  5.0}
\DoxyCodeLine{priority\_group.control.egress\_buffer.mc.sp\_percent    = 30.0}
\DoxyCodeLine{}
\DoxyCodeLine{\# the per-port limit on multicast packets (applies to all switch priorities)}
\DoxyCodeLine{port.egress\_buffer.mc.sp\_percent = 12.5}
\DoxyCodeLine{}
\DoxyCodeLine{\# internal cos values mapped to egress queues}
\DoxyCodeLine{\# multicast queue: same as unicast queue}
\DoxyCodeLine{cos\_egr\_queue.cos\_0.uc  = 0}
\DoxyCodeLine{cos\_egr\_queue.cos\_0.cpu = 0}
\DoxyCodeLine{}
\DoxyCodeLine{cos\_egr\_queue.cos\_1.uc  = 1}
\DoxyCodeLine{cos\_egr\_queue.cos\_1.cpu = 1}
\DoxyCodeLine{}
\DoxyCodeLine{cos\_egr\_queue.cos\_2.uc  = 2}
\DoxyCodeLine{cos\_egr\_queue.cos\_2.cpu = 2}
\DoxyCodeLine{}
\DoxyCodeLine{cos\_egr\_queue.cos\_3.uc  = 3}
\DoxyCodeLine{cos\_egr\_queue.cos\_3.cpu = 3}
\DoxyCodeLine{}
\DoxyCodeLine{cos\_egr\_queue.cos\_4.uc  = 4}
\DoxyCodeLine{cos\_egr\_queue.cos\_4.cpu = 4}
\DoxyCodeLine{}
\DoxyCodeLine{cos\_egr\_queue.cos\_5.uc  = 5}
\DoxyCodeLine{cos\_egr\_queue.cos\_5.cpu = 5}
\DoxyCodeLine{}
\DoxyCodeLine{cos\_egr\_queue.cos\_6.uc  = 6}
\DoxyCodeLine{cos\_egr\_queue.cos\_6.cpu = 6}
\DoxyCodeLine{}
\DoxyCodeLine{cos\_egr\_queue.cos\_7.uc  = 7}
\DoxyCodeLine{cos\_egr\_queue.cos\_7.cpu = 7}
\DoxyCodeLine{}
\DoxyCodeLine{\#TCAM resource forwarding profile}
\DoxyCodeLine{\# Valid profiles -}
\DoxyCodeLine{\#    default, ipmc-heavy, acl-heavy, ipmc-max}
\DoxyCodeLine{tcam\_resource.profile = default}
\DoxyCodeLine{}
\DoxyCodeLine{\# Resilient hash timers: in milliseconds}
\DoxyCodeLine{\# resilient\_hash\_active\_timer = 120000}
\DoxyCodeLine{\# resilient\_hash\_max\_unbalanced\_timer = 4294967295}
\end{DoxyCode}
 
\begin{DoxyCode}{0}
\DoxyCodeLine{Usage: cat [OPTION]... [FILE]...}
\DoxyCodeLine{Concatenate FILE(s), or standard input, to standard output.}
\DoxyCodeLine{}
\DoxyCodeLine{  -A, --show-all           equivalent to -vET}
\DoxyCodeLine{  -b, --number-nonblank    number nonempty output lines, overrides -n}
\DoxyCodeLine{  -e                       equivalent to -vE}
\DoxyCodeLine{  -E, --show-ends          display \$ at end of each line}
\DoxyCodeLine{  -n, --number             number all output lines}
\DoxyCodeLine{  -s, --squeeze-blank      suppress repeated empty output lines}
\DoxyCodeLine{  -t                       equivalent to -vT}
\DoxyCodeLine{  -T, --show-tabs          display TAB characters as \string^I}
\DoxyCodeLine{  -u                       (ignored)}
\DoxyCodeLine{  -v, --show-nonprinting   use \string^ and M- notation, except for LFD and TAB}
\DoxyCodeLine{      --help     display this help and exit}
\DoxyCodeLine{      --version  output version information and exit}
\DoxyCodeLine{}
\DoxyCodeLine{With no FILE, or when FILE is -, read standard input.}
\DoxyCodeLine{}
\DoxyCodeLine{Examples:}
\DoxyCodeLine{  cat f - g  Output f's contents, then standard input, then g's contents.}
\DoxyCodeLine{  cat        Copy standard input to standard output.}
\DoxyCodeLine{}
\DoxyCodeLine{GNU coreutils online help: <http://www.gnu.org/software/coreutils/>}
\DoxyCodeLine{Full documentation at: <http://www.gnu.org/software/coreutils/cat>}
\DoxyCodeLine{or available locally via: info '(coreutils) cat invocation'}
\DoxyCodeLine{}
\DoxyCodeLine{\#}
\DoxyCodeLine{\# Default datapath configuration for Mellanox Spectrum chip}
\DoxyCodeLine{\# Copyright 2016, 2017, Cumulus Networks, Inc.  All rights reserved.}
\DoxyCodeLine{\#}
\DoxyCodeLine{}
\DoxyCodeLine{\# priority group ID assigned to each priority group}
\DoxyCodeLine{priority\_group.control.id = 7}
\DoxyCodeLine{priority\_group.service.id = 2}
\DoxyCodeLine{priority\_group.bulk.id = 0}
\DoxyCodeLine{}
\DoxyCodeLine{\# all priority groups share a service pool on Spectrum}
\DoxyCodeLine{\# service pools assigned to each priority group}
\DoxyCodeLine{priority\_group.control.service\_pool = 0}
\DoxyCodeLine{priority\_group.service.service\_pool = 0}
\DoxyCodeLine{priority\_group.bulk.service\_pool = 0}
\DoxyCodeLine{}
\DoxyCodeLine{\# --- ingress buffer space allocations ---}
\DoxyCodeLine{\#}
\DoxyCodeLine{\# total buffer}
\DoxyCodeLine{\#  - ingress minimum buffer allocations}
\DoxyCodeLine{\#  - ingress service pool buffer allocations}
\DoxyCodeLine{\#  - priority group ingress headroom allocations}
\DoxyCodeLine{\#  - ingress global headroom allocations}
\DoxyCodeLine{\#  = total ingress shared buffer size}
\DoxyCodeLine{}
\DoxyCodeLine{\# ingress service pool buffer allocation: percent of total buffer}
\DoxyCodeLine{\# If a service pool has no priority groups, the buffer is added}
\DoxyCodeLine{\# to the shared buffer space.}
\DoxyCodeLine{ingress\_service\_pool.0.percent = 100.0  \# all priority groups}
\DoxyCodeLine{}
\DoxyCodeLine{\# priority group minimum buffer allocation: percent of total buffer cells}
\DoxyCodeLine{\# priority group shared buffer allocation: percent of total shared buffer size}
\DoxyCodeLine{\# if a priority group has no packet priority values assigned to it, the buffers will not be allocated}
\DoxyCodeLine{}
\DoxyCodeLine{priority\_group.control.ingress\_buffer.min\_percent     =  4.0}
\DoxyCodeLine{priority\_group.control.ingress\_buffer.shared\_percent  = 20.0}
\DoxyCodeLine{}
\DoxyCodeLine{priority\_group.service.ingress\_buffer.min\_percent     =  2.0}
\DoxyCodeLine{priority\_group.service.ingress\_buffer.shared\_percent  =  3.0}
\DoxyCodeLine{}
\DoxyCodeLine{priority\_group.bulk.ingress\_buffer.min\_percent        = 18.0}
\DoxyCodeLine{priority\_group.bulk.ingress\_buffer.shared\_percent     = 15.0}
\DoxyCodeLine{}
\DoxyCodeLine{\# --- egress buffer space allocations ---}
\DoxyCodeLine{\#}
\DoxyCodeLine{\# total egress buffer}
\DoxyCodeLine{\#  - minimum buffer allocations}
\DoxyCodeLine{\#  = total service pool buffer size}
\DoxyCodeLine{\#}
\DoxyCodeLine{\# Service pool buffer allocation: percent of total}
\DoxyCodeLine{\# buffer size.}
\DoxyCodeLine{egress\_service\_pool.0.percent = 100.0   \# all priority groups, UC and MC}
\DoxyCodeLine{}
\DoxyCodeLine{\# Front panel port egress buffer limits enforced for each}
\DoxyCodeLine{\# priority group.}
\DoxyCodeLine{\# Unlimited egress buffers not supported on Spectrum.}
\DoxyCodeLine{priority\_group.control.unlimited\_egress\_buffer  = false}
\DoxyCodeLine{priority\_group.service.unlimited\_egress\_buffer  = false}
\DoxyCodeLine{priority\_group.bulk.unlimited\_egress\_buffer     = false}
\DoxyCodeLine{}
\DoxyCodeLine{\#}
\DoxyCodeLine{\# if a priority group has no cos values assigned to it, the buffers will not be allocated}
\DoxyCodeLine{\#}
\DoxyCodeLine{}
\DoxyCodeLine{\# priority group minimum buffer allocation: percent of total buffer}
\DoxyCodeLine{priority\_group.bulk.egress\_buffer.uc.min\_percent      =  6.0}
\DoxyCodeLine{priority\_group.service.egress\_buffer.uc.min\_percent   =  2.0}
\DoxyCodeLine{priority\_group.control.egress\_buffer.uc.min\_percent   =  4.0}
\DoxyCodeLine{}
\DoxyCodeLine{priority\_group.bulk.egress\_buffer.mc.min\_percent      =  5.0}
\DoxyCodeLine{priority\_group.service.egress\_buffer.mc.min\_percent   =  2.0}
\DoxyCodeLine{priority\_group.control.egress\_buffer.mc.min\_percent   =  3.0}
\DoxyCodeLine{}
\DoxyCodeLine{\# Priority group service pool buffer limits: percent of the}
\DoxyCodeLine{\# assigned service pool.}
\DoxyCodeLine{priority\_group.bulk.egress\_buffer.uc.sp\_percent       = 40.0}
\DoxyCodeLine{priority\_group.service.egress\_buffer.uc.sp\_percent    =  5.0}
\DoxyCodeLine{priority\_group.control.egress\_buffer.uc.sp\_percent    = 30.0}
\DoxyCodeLine{}
\DoxyCodeLine{priority\_group.bulk.egress\_buffer.mc.sp\_percent       = 40.0}
\DoxyCodeLine{priority\_group.service.egress\_buffer.mc.sp\_percent    =  5.0}
\DoxyCodeLine{priority\_group.control.egress\_buffer.mc.sp\_percent    = 30.0}
\DoxyCodeLine{}
\DoxyCodeLine{\# the per-port limit on multicast packets (applies to all switch priorities)}
\DoxyCodeLine{port.egress\_buffer.mc.sp\_percent = 12.5}
\DoxyCodeLine{}
\DoxyCodeLine{\# internal cos values mapped to egress queues}
\DoxyCodeLine{\# multicast queue: same as unicast queue}
\DoxyCodeLine{cos\_egr\_queue.cos\_0.uc  = 0}
\DoxyCodeLine{cos\_egr\_queue.cos\_0.cpu = 0}
\DoxyCodeLine{}
\DoxyCodeLine{cos\_egr\_queue.cos\_1.uc  = 1}
\DoxyCodeLine{cos\_egr\_queue.cos\_1.cpu = 1}
\DoxyCodeLine{}
\DoxyCodeLine{cos\_egr\_queue.cos\_2.uc  = 2}
\DoxyCodeLine{cos\_egr\_queue.cos\_2.cpu = 2}
\DoxyCodeLine{}
\DoxyCodeLine{cos\_egr\_queue.cos\_3.uc  = 3}
\DoxyCodeLine{cos\_egr\_queue.cos\_3.cpu = 3}
\DoxyCodeLine{}
\DoxyCodeLine{cos\_egr\_queue.cos\_4.uc  = 4}
\DoxyCodeLine{cos\_egr\_queue.cos\_4.cpu = 4}
\DoxyCodeLine{}
\DoxyCodeLine{cos\_egr\_queue.cos\_5.uc  = 5}
\DoxyCodeLine{cos\_egr\_queue.cos\_5.cpu = 5}
\DoxyCodeLine{}
\DoxyCodeLine{cos\_egr\_queue.cos\_6.uc  = 6}
\DoxyCodeLine{cos\_egr\_queue.cos\_6.cpu = 6}
\DoxyCodeLine{}
\DoxyCodeLine{cos\_egr\_queue.cos\_7.uc  = 7}
\DoxyCodeLine{cos\_egr\_queue.cos\_7.cpu = 7}
\DoxyCodeLine{}
\DoxyCodeLine{\#TCAM resource forwarding profile}
\DoxyCodeLine{\# Valid profiles -}
\DoxyCodeLine{\#    default, ipmc-heavy, acl-heavy, ipmc-max}
\DoxyCodeLine{tcam\_resource.profile = default}
\DoxyCodeLine{}
\DoxyCodeLine{\# Resilient hash timers: in milliseconds}
\DoxyCodeLine{\# resilient\_hash\_active\_timer = 120000}
\DoxyCodeLine{\# resilient\_hash\_max\_unbalanced\_timer = 4294967295}
\end{DoxyCode}
 \subsection*{cat /etc/cumulus/switchd.conf}

\#\#\# 
\begin{DoxyCode}{0}
\DoxyCodeLine{\# }
\DoxyCodeLine{\# /etc/cumulus/switchd.conf - switchd configuration file}
\DoxyCodeLine{\#}
\DoxyCodeLine{}
\DoxyCodeLine{\# Statistic poll interval (in msec)}
\DoxyCodeLine{\#stats.poll\_interval = 2000}
\DoxyCodeLine{}
\DoxyCodeLine{\# Buffer utilization poll interval (in msec), 0 means disable}
\DoxyCodeLine{\#buf\_util.poll\_interval = 0}
\DoxyCodeLine{}
\DoxyCodeLine{\# Buffer utilization measurement interval (in mins)}
\DoxyCodeLine{\#buf\_util.measure\_interval = 0}
\DoxyCodeLine{}
\DoxyCodeLine{\# Optimize ACL HW resources for better utilization}
\DoxyCodeLine{\#acl.optimize\_hw = FALSE}
\DoxyCodeLine{}
\DoxyCodeLine{\# Enable Flow based mirroring.}
\DoxyCodeLine{\#acl.flow\_based\_mirroring = TRUE}
\DoxyCodeLine{}
\DoxyCodeLine{\# Enable non atomic acl update}
\DoxyCodeLine{acl.non\_atomic\_update\_mode = FALSE}
\DoxyCodeLine{}
\DoxyCodeLine{\# Send ARPs for next hops}
\DoxyCodeLine{\#arp.next\_hops = TRUE}
\DoxyCodeLine{}
\DoxyCodeLine{\# Kernel routing table ID, range 1 - 2\string^31, default 254}
\DoxyCodeLine{\#route.table = 254}
\DoxyCodeLine{}
\DoxyCodeLine{\# Maximum hardware neighbor table occupancy (percent of hardware table size)}
\DoxyCodeLine{\#route.host\_max\_percent = 100}
\DoxyCodeLine{}
\DoxyCodeLine{\# Coalescing reduction factor for accumulating changes to reduce CPU load}
\DoxyCodeLine{\#coalescing.reducer = 1}
\DoxyCodeLine{}
\DoxyCodeLine{\# Coalescing time limit, in seconds}
\DoxyCodeLine{\#coalescing.timeout = 10}
\DoxyCodeLine{}
\DoxyCodeLine{\# Ignore routes that point to non-swp interfaces}
\DoxyCodeLine{ignore\_non\_swps = TRUE}
\DoxyCodeLine{}
\DoxyCodeLine{\# Disables restart after parity error}
\DoxyCodeLine{\#disable\_internal\_parity\_restart = TRUE}
\DoxyCodeLine{}
\DoxyCodeLine{\# Disables restart after an unrecoverable hardware error}
\DoxyCodeLine{\#disable\_internal\_hw\_err\_restart = FALSE}
\DoxyCodeLine{}
\DoxyCodeLine{\# Log messages using the given BACKEND=LEVEL,}
\DoxyCodeLine{\# space separate multiple BACKEND=LEVEL pairs.}
\DoxyCodeLine{\# BACKEND := \{stderr, file:filename, syslog, program:executable\},}
\DoxyCodeLine{\# LEVEL := \{CRIT, ERR, WARN, INFO, DEBUG\}}
\DoxyCodeLine{\# Prior to Cumulus Linux 2.5.4, file:/var/log/switchd.log=INFO was the default}
\DoxyCodeLine{logging = syslog=INFO}
\DoxyCodeLine{}
\DoxyCodeLine{\# Storm Control setting on a port, in pps, 0 means disable}
\DoxyCodeLine{\#interface.swp1.storm\_control.broadcast = 400}
\DoxyCodeLine{\#interface.swp1.storm\_control.multicast = 3000}
\DoxyCodeLine{}
\DoxyCodeLine{\# Maximum route limit}
\DoxyCodeLine{\#route.max\_routes = 32768}
\DoxyCodeLine{}
\DoxyCodeLine{\# Enable HW statistics }
\DoxyCodeLine{\# level specifies type of stats needed per-hw resource type.}
\DoxyCodeLine{\# LEVEL := \{NONE, BRIEF, DETAIL\}}
\DoxyCodeLine{\#stats.vlan.aggregate = BRIEF}
\DoxyCodeLine{\#stats.vxlan.aggregate = DETAIL}
\DoxyCodeLine{\#stats.vxlan.member = BRIEF}
\DoxyCodeLine{}
\DoxyCodeLine{\#stats.vlan.show\_internal\_vlans = FALSE}
\DoxyCodeLine{}
\DoxyCodeLine{\# Virtual devices hw-stat poll interval (in seconds)}
\DoxyCodeLine{\#stats.vdev\_hw\_poll\_interval = 5}
\DoxyCodeLine{}
\DoxyCodeLine{\# Internal VLAN range}
\DoxyCodeLine{\# minimum range size is 300}
\DoxyCodeLine{\#resv\_vlan\_range = 3000-3999}
\DoxyCodeLine{}
\DoxyCodeLine{\# Netlink}
\DoxyCodeLine{\# netlink socket buf size (90 * 1024 * 1024 = 90MB)}
\DoxyCodeLine{\#netlink.buf\_size=94371840}
\DoxyCodeLine{}
\DoxyCodeLine{\# delete routes on interfaces when carrier is down}
\DoxyCodeLine{\#route.delete\_dead\_routes = TRUE}
\DoxyCodeLine{}
\DoxyCodeLine{\# default TTL to use in vxlan header}
\DoxyCodeLine{\#vxlan.default\_ttl = 64}
\DoxyCodeLine{}
\DoxyCodeLine{\# bridge broadcast frame to cpu even if SVI is not enabled }
\DoxyCodeLine{\#bridge.broadcast\_frame\_to\_cpu = FALSE}
\DoxyCodeLine{}
\DoxyCodeLine{\#netlink libnl logger [0-5]}
\DoxyCodeLine{\#netlink.nl\_logger = 0}
\DoxyCodeLine{}
\DoxyCodeLine{\# router mac lookup per vlan}
\DoxyCodeLine{\#hal.bcm.per\_vlan\_router\_mac\_lookup = FALSE}
\DoxyCodeLine{}
\DoxyCodeLine{\#IGMP snooping optimized multicast forwarding}
\DoxyCodeLine{\#bridge.optimized\_mcast\_flood = FALSE}
\DoxyCodeLine{}
\DoxyCodeLine{\# default vxlan outer dscp action during encap}
\DoxyCodeLine{\# \{copy | set | derive\}}
\DoxyCodeLine{\# copy: only if inner packet is IP}
\DoxyCodeLine{\# set: to specific value}
\DoxyCodeLine{\# derive: from switch priority}
\DoxyCodeLine{\#vxlan.def\_encap\_dscp\_action = derive}
\DoxyCodeLine{}
\DoxyCodeLine{\# default vxlan encap dscp value, only applicable if action is 'set'}
\DoxyCodeLine{\#vxlan.def\_encap\_dscp\_value =}
\DoxyCodeLine{}
\DoxyCodeLine{\# default vxlan decap dscp/cos action}
\DoxyCodeLine{\# \{copy | preserve | derive\}}
\DoxyCodeLine{\# copy: only if inner packet is IP}
\DoxyCodeLine{\# preserve: inner dscp unchanged}
\DoxyCodeLine{\# derive: from switch priority}
\DoxyCodeLine{\#vxlan.def\_decap\_dscp\_action = derive}
\DoxyCodeLine{}
\DoxyCodeLine{\# Enable send unknown ipmc to CPU}
\DoxyCodeLine{\#ipmulticast.unknown\_ipmc\_to\_cpu = FALSE}
\end{DoxyCode}
 
\begin{DoxyCode}{0}
\DoxyCodeLine{Usage: cat [OPTION]... [FILE]...}
\DoxyCodeLine{Concatenate FILE(s), or standard input, to standard output.}
\DoxyCodeLine{}
\DoxyCodeLine{  -A, --show-all           equivalent to -vET}
\DoxyCodeLine{  -b, --number-nonblank    number nonempty output lines, overrides -n}
\DoxyCodeLine{  -e                       equivalent to -vE}
\DoxyCodeLine{  -E, --show-ends          display \$ at end of each line}
\DoxyCodeLine{  -n, --number             number all output lines}
\DoxyCodeLine{  -s, --squeeze-blank      suppress repeated empty output lines}
\DoxyCodeLine{  -t                       equivalent to -vT}
\DoxyCodeLine{  -T, --show-tabs          display TAB characters as \string^I}
\DoxyCodeLine{  -u                       (ignored)}
\DoxyCodeLine{  -v, --show-nonprinting   use \string^ and M- notation, except for LFD and TAB}
\DoxyCodeLine{      --help     display this help and exit}
\DoxyCodeLine{      --version  output version information and exit}
\DoxyCodeLine{}
\DoxyCodeLine{With no FILE, or when FILE is -, read standard input.}
\DoxyCodeLine{}
\DoxyCodeLine{Examples:}
\DoxyCodeLine{  cat f - g  Output f's contents, then standard input, then g's contents.}
\DoxyCodeLine{  cat        Copy standard input to standard output.}
\DoxyCodeLine{}
\DoxyCodeLine{GNU coreutils online help: <http://www.gnu.org/software/coreutils/>}
\DoxyCodeLine{Full documentation at: <http://www.gnu.org/software/coreutils/cat>}
\DoxyCodeLine{or available locally via: info '(coreutils) cat invocation'}
\DoxyCodeLine{}
\DoxyCodeLine{\# }
\DoxyCodeLine{\# /etc/cumulus/switchd.conf - switchd configuration file}
\DoxyCodeLine{\#}
\DoxyCodeLine{}
\DoxyCodeLine{\# Statistic poll interval (in msec)}
\DoxyCodeLine{\#stats.poll\_interval = 2000}
\DoxyCodeLine{}
\DoxyCodeLine{\# Buffer utilization poll interval (in msec), 0 means disable}
\DoxyCodeLine{\#buf\_util.poll\_interval = 0}
\DoxyCodeLine{}
\DoxyCodeLine{\# Buffer utilization measurement interval (in mins)}
\DoxyCodeLine{\#buf\_util.measure\_interval = 0}
\DoxyCodeLine{}
\DoxyCodeLine{\# Optimize ACL HW resources for better utilization}
\DoxyCodeLine{\#acl.optimize\_hw = FALSE}
\DoxyCodeLine{}
\DoxyCodeLine{\# Enable Flow based mirroring.}
\DoxyCodeLine{\#acl.flow\_based\_mirroring = TRUE}
\DoxyCodeLine{}
\DoxyCodeLine{\# Enable non atomic acl update}
\DoxyCodeLine{acl.non\_atomic\_update\_mode = FALSE}
\DoxyCodeLine{}
\DoxyCodeLine{\# Send ARPs for next hops}
\DoxyCodeLine{\#arp.next\_hops = TRUE}
\DoxyCodeLine{}
\DoxyCodeLine{\# Kernel routing table ID, range 1 - 2\string^31, default 254}
\DoxyCodeLine{\#route.table = 254}
\DoxyCodeLine{}
\DoxyCodeLine{\# Maximum hardware neighbor table occupancy (percent of hardware table size)}
\DoxyCodeLine{\#route.host\_max\_percent = 100}
\DoxyCodeLine{}
\DoxyCodeLine{\# Coalescing reduction factor for accumulating changes to reduce CPU load}
\DoxyCodeLine{\#coalescing.reducer = 1}
\DoxyCodeLine{}
\DoxyCodeLine{\# Coalescing time limit, in seconds}
\DoxyCodeLine{\#coalescing.timeout = 10}
\DoxyCodeLine{}
\DoxyCodeLine{\# Ignore routes that point to non-swp interfaces}
\DoxyCodeLine{ignore\_non\_swps = TRUE}
\DoxyCodeLine{}
\DoxyCodeLine{\# Disables restart after parity error}
\DoxyCodeLine{\#disable\_internal\_parity\_restart = TRUE}
\DoxyCodeLine{}
\DoxyCodeLine{\# Disables restart after an unrecoverable hardware error}
\DoxyCodeLine{\#disable\_internal\_hw\_err\_restart = FALSE}
\DoxyCodeLine{}
\DoxyCodeLine{\# Log messages using the given BACKEND=LEVEL,}
\DoxyCodeLine{\# space separate multiple BACKEND=LEVEL pairs.}
\DoxyCodeLine{\# BACKEND := \{stderr, file:filename, syslog, program:executable\},}
\DoxyCodeLine{\# LEVEL := \{CRIT, ERR, WARN, INFO, DEBUG\}}
\DoxyCodeLine{\# Prior to Cumulus Linux 2.5.4, file:/var/log/switchd.log=INFO was the default}
\DoxyCodeLine{logging = syslog=INFO}
\DoxyCodeLine{}
\DoxyCodeLine{\# Storm Control setting on a port, in pps, 0 means disable}
\DoxyCodeLine{\#interface.swp1.storm\_control.broadcast = 400}
\DoxyCodeLine{\#interface.swp1.storm\_control.multicast = 3000}
\DoxyCodeLine{}
\DoxyCodeLine{\# Maximum route limit}
\DoxyCodeLine{\#route.max\_routes = 32768}
\DoxyCodeLine{}
\DoxyCodeLine{\# Enable HW statistics }
\DoxyCodeLine{\# level specifies type of stats needed per-hw resource type.}
\DoxyCodeLine{\# LEVEL := \{NONE, BRIEF, DETAIL\}}
\DoxyCodeLine{\#stats.vlan.aggregate = BRIEF}
\DoxyCodeLine{\#stats.vxlan.aggregate = DETAIL}
\DoxyCodeLine{\#stats.vxlan.member = BRIEF}
\DoxyCodeLine{}
\DoxyCodeLine{\#stats.vlan.show\_internal\_vlans = FALSE}
\DoxyCodeLine{}
\DoxyCodeLine{\# Virtual devices hw-stat poll interval (in seconds)}
\DoxyCodeLine{\#stats.vdev\_hw\_poll\_interval = 5}
\DoxyCodeLine{}
\DoxyCodeLine{\# Internal VLAN range}
\DoxyCodeLine{\# minimum range size is 300}
\DoxyCodeLine{\#resv\_vlan\_range = 3000-3999}
\DoxyCodeLine{}
\DoxyCodeLine{\# Netlink}
\DoxyCodeLine{\# netlink socket buf size (90 * 1024 * 1024 = 90MB)}
\DoxyCodeLine{\#netlink.buf\_size=94371840}
\DoxyCodeLine{}
\DoxyCodeLine{\# delete routes on interfaces when carrier is down}
\DoxyCodeLine{\#route.delete\_dead\_routes = TRUE}
\DoxyCodeLine{}
\DoxyCodeLine{\# default TTL to use in vxlan header}
\DoxyCodeLine{\#vxlan.default\_ttl = 64}
\DoxyCodeLine{}
\DoxyCodeLine{\# bridge broadcast frame to cpu even if SVI is not enabled }
\DoxyCodeLine{\#bridge.broadcast\_frame\_to\_cpu = FALSE}
\DoxyCodeLine{}
\DoxyCodeLine{\#netlink libnl logger [0-5]}
\DoxyCodeLine{\#netlink.nl\_logger = 0}
\DoxyCodeLine{}
\DoxyCodeLine{\# router mac lookup per vlan}
\DoxyCodeLine{\#hal.bcm.per\_vlan\_router\_mac\_lookup = FALSE}
\DoxyCodeLine{}
\DoxyCodeLine{\#IGMP snooping optimized multicast forwarding}
\DoxyCodeLine{\#bridge.optimized\_mcast\_flood = FALSE}
\DoxyCodeLine{}
\DoxyCodeLine{\# default vxlan outer dscp action during encap}
\DoxyCodeLine{\# \{copy | set | derive\}}
\DoxyCodeLine{\# copy: only if inner packet is IP}
\DoxyCodeLine{\# set: to specific value}
\DoxyCodeLine{\# derive: from switch priority}
\DoxyCodeLine{\#vxlan.def\_encap\_dscp\_action = derive}
\DoxyCodeLine{}
\DoxyCodeLine{\# default vxlan encap dscp value, only applicable if action is 'set'}
\DoxyCodeLine{\#vxlan.def\_encap\_dscp\_value =}
\DoxyCodeLine{}
\DoxyCodeLine{\# default vxlan decap dscp/cos action}
\DoxyCodeLine{\# \{copy | preserve | derive\}}
\DoxyCodeLine{\# copy: only if inner packet is IP}
\DoxyCodeLine{\# preserve: inner dscp unchanged}
\DoxyCodeLine{\# derive: from switch priority}
\DoxyCodeLine{\#vxlan.def\_decap\_dscp\_action = derive}
\DoxyCodeLine{}
\DoxyCodeLine{\# Enable send unknown ipmc to CPU}
\DoxyCodeLine{\#ipmulticast.unknown\_ipmc\_to\_cpu = FALSE}
\end{DoxyCode}
 \subsection*{sudo systemctl -\/l status frr.\+service}

\#\#\# 
\begin{DoxyCode}{0}
\DoxyCodeLine{● frr.service - FRRouting}
\DoxyCodeLine{   Loaded: loaded (/lib/systemd/system/frr.service; enabled)}
\DoxyCodeLine{   Active: active (running) since Mon 2018-07-30 22:35:20 UTC; 1h 23min ago}
\DoxyCodeLine{  Process: 740 ExecStart=/usr/lib/frr/frr start (code=exited, status=0/SUCCESS)}
\DoxyCodeLine{   CGroup: /system.slice/frr.service}
\DoxyCodeLine{           ├─765 /usr/lib/frr/zebra -M snmp -s 90000000 --daemon -A 127.0.0.1}
\DoxyCodeLine{           └─771 /usr/lib/frr/watchfrr -d -r /usr/sbin/servicebBfrrbBrestartbB\%s -s /usr/sbin/servicebBfrrbBstartbB\%s -k /usr/sbin/servicebBfrrbBstopbB\%s -b bB zebra}
\DoxyCodeLine{}
\DoxyCodeLine{Jul 30 22:35:18 oob-mgmt-server zebra[764]: Setting netlink socket receive buffer size: 266240 -> 180000000}
\DoxyCodeLine{Jul 30 22:35:18 oob-mgmt-server zebra[765]: Initializing own label manager}
\DoxyCodeLine{Jul 30 22:35:18 oob-mgmt-server zebra[765]: zebra 4.0+cl3u2 starting: vty@2601}
\DoxyCodeLine{Jul 30 22:35:18 oob-mgmt-server frr[740]: Starting Frr daemons (prio:10):. zebra.}
\DoxyCodeLine{Jul 30 22:35:18 oob-mgmt-server watchfrr[771]: watchfrr 4.0+cl3u2 watching [zebra]}
\DoxyCodeLine{Jul 30 22:35:19 oob-mgmt-server watchfrr[771]: zebra state -> up : connect succeeded}
\DoxyCodeLine{Jul 30 22:35:19 oob-mgmt-server watchfrr[771]: Watchfrr: Notifying Systemd we are up and running}
\DoxyCodeLine{Jul 30 22:35:19 oob-mgmt-server frr[740]: Starting Frr monitor daemon: watchfrr.}
\DoxyCodeLine{Jul 30 22:35:20 oob-mgmt-server frr[740]: Exiting from the script}
\DoxyCodeLine{Jul 30 22:35:20 oob-mgmt-server systemd[1]: Started FRRouting.}
\end{DoxyCode}
 
\begin{DoxyCode}{0}
\DoxyCodeLine{systemctl [OPTIONS...] \{COMMAND\} ...}
\DoxyCodeLine{}
\DoxyCodeLine{Query or send control commands to the systemd manager.}
\DoxyCodeLine{}
\DoxyCodeLine{  -h --help           Show this help}
\DoxyCodeLine{     --version        Show package version}
\DoxyCodeLine{     --system         Connect to system manager}
\DoxyCodeLine{     --user           Connect to user service manager}
\DoxyCodeLine{  -H --host=[USER@]HOST}
\DoxyCodeLine{                      Operate on remote host}
\DoxyCodeLine{  -M --machine=CONTAINER}
\DoxyCodeLine{                      Operate on local container}
\DoxyCodeLine{  -t --type=TYPE      List only units of a particular type}
\DoxyCodeLine{     --state=STATE    List only units with particular LOAD or SUB or ACTIVE state}
\DoxyCodeLine{  -p --property=NAME  Show only properties by this name}
\DoxyCodeLine{  -a --all            Show all loaded units/properties, including dead/empty}
\DoxyCodeLine{                      ones. To list all units installed on the system, use}
\DoxyCodeLine{                      the 'list-unit-files' command instead.}
\DoxyCodeLine{  -l --full           Don't ellipsize unit names on output}
\DoxyCodeLine{  -r --recursive      Show unit list of host and local containers}
\DoxyCodeLine{     --reverse        Show reverse dependencies with 'list-dependencies'}
\DoxyCodeLine{     --job-mode=MODE  Specify how to deal with already queued jobs, when}
\DoxyCodeLine{                      queueing a new job}
\DoxyCodeLine{     --show-types     When showing sockets, explicitly show their type}
\DoxyCodeLine{  -i --ignore-inhibitors}
\DoxyCodeLine{                      When shutting down or sleeping, ignore inhibitors}
\DoxyCodeLine{     --kill-who=WHO   Who to send signal to}
\DoxyCodeLine{  -s --signal=SIGNAL  Which signal to send}
\DoxyCodeLine{  -q --quiet          Suppress output}
\DoxyCodeLine{     --no-block       Do not wait until operation finished}
\DoxyCodeLine{     --no-wall        Don't send wall message before halt/power-off/reboot}
\DoxyCodeLine{     --no-reload      When enabling/disabling unit files, don't reload daemon}
\DoxyCodeLine{                      configuration}
\DoxyCodeLine{     --no-legend      Do not print a legend (column headers and hints)}
\DoxyCodeLine{     --no-pager       Do not pipe output into a pager}
\DoxyCodeLine{     --no-ask-password}
\DoxyCodeLine{                      Do not ask for system passwords}
\DoxyCodeLine{     --global         Enable/disable unit files globally}
\DoxyCodeLine{     --runtime        Enable unit files only temporarily until next reboot}
\DoxyCodeLine{  -f --force          When enabling unit files, override existing symlinks}
\DoxyCodeLine{                      When shutting down, execute action immediately}
\DoxyCodeLine{     --preset-mode=   Specifies whether fully apply presets, or only enable,}
\DoxyCodeLine{                      or only disable}
\DoxyCodeLine{     --root=PATH      Enable unit files in the specified root directory}
\DoxyCodeLine{  -n --lines=INTEGER  Number of journal entries to show}
\DoxyCodeLine{  -o --output=STRING  Change journal output mode (short, short-monotonic,}
\DoxyCodeLine{                      verbose, export, json, json-pretty, json-sse, cat)}
\DoxyCodeLine{     --plain          Print unit dependencies as a list instead of a tree}
\DoxyCodeLine{}
\DoxyCodeLine{Unit Commands:}
\DoxyCodeLine{  list-units [PATTERN...]         List loaded units}
\DoxyCodeLine{  list-sockets [PATTERN...]       List loaded sockets ordered by address}
\DoxyCodeLine{  list-timers [PATTERN...]        List loaded timers ordered by next elapse}
\DoxyCodeLine{  start NAME...                   Start (activate) one or more units}
\DoxyCodeLine{  stop NAME...                    Stop (deactivate) one or more units}
\DoxyCodeLine{  reload NAME...                  Reload one or more units}
\DoxyCodeLine{  restart NAME...                 Start or restart one or more units}
\DoxyCodeLine{  try-restart NAME...             Restart one or more units if active}
\DoxyCodeLine{  reload-or-restart NAME...       Reload one or more units if possible,}
\DoxyCodeLine{                                  otherwise start or restart}
\DoxyCodeLine{  reload-or-try-restart NAME...   Reload one or more units if possible,}
\DoxyCodeLine{                                  otherwise restart if active}
\DoxyCodeLine{  isolate NAME                    Start one unit and stop all others}
\DoxyCodeLine{  kill NAME...                    Send signal to processes of a unit}
\DoxyCodeLine{  is-active PATTERN...            Check whether units are active}
\DoxyCodeLine{  is-failed PATTERN...            Check whether units are failed}
\DoxyCodeLine{  status [PATTERN...|PID...]      Show runtime status of one or more units}
\DoxyCodeLine{  show [PATTERN...|JOB...]        Show properties of one or more}
\DoxyCodeLine{                                  units/jobs or the manager}
\DoxyCodeLine{  cat PATTERN...                  Show files and drop-ins of one or more units}
\DoxyCodeLine{  set-property NAME ASSIGNMENT... Sets one or more properties of a unit}
\DoxyCodeLine{  help PATTERN...|PID...          Show manual for one or more units}
\DoxyCodeLine{  reset-failed [PATTERN...]       Reset failed state for all, one, or more}
\DoxyCodeLine{                                  units}
\DoxyCodeLine{  list-dependencies [NAME]        Recursively show units which are required}
\DoxyCodeLine{                                  or wanted by this unit or by which this}
\DoxyCodeLine{                                  unit is required or wanted}
\DoxyCodeLine{}
\DoxyCodeLine{Unit File Commands:}
\DoxyCodeLine{  list-unit-files [PATTERN...]    List installed unit files}
\DoxyCodeLine{  enable NAME...                  Enable one or more unit files}
\DoxyCodeLine{  disable NAME...                 Disable one or more unit files}
\DoxyCodeLine{  reenable NAME...                Reenable one or more unit files}
\DoxyCodeLine{  preset NAME...                  Enable/disable one or more unit files}
\DoxyCodeLine{                                  based on preset configuration}
\DoxyCodeLine{  preset-all                      Enable/disable all unit files based on}
\DoxyCodeLine{                                  preset configuration}
\DoxyCodeLine{  is-enabled NAME...              Check whether unit files are enabled}
\DoxyCodeLine{}
\DoxyCodeLine{  mask NAME...                    Mask one or more units}
\DoxyCodeLine{  unmask NAME...                  Unmask one or more units}
\DoxyCodeLine{  link PATH...                    Link one or more units files into}
\DoxyCodeLine{                                  the search path}
\DoxyCodeLine{  get-default                     Get the name of the default target}
\DoxyCodeLine{  set-default NAME                Set the default target}
\DoxyCodeLine{}
\DoxyCodeLine{Machine Commands:}
\DoxyCodeLine{  list-machines [PATTERN...]      List local containers and host}
\DoxyCodeLine{}
\DoxyCodeLine{Job Commands:}
\DoxyCodeLine{  list-jobs [PATTERN...]          List jobs}
\DoxyCodeLine{  cancel [JOB...]                 Cancel all, one, or more jobs}
\DoxyCodeLine{}
\DoxyCodeLine{Snapshot Commands:}
\DoxyCodeLine{  snapshot [NAME]                 Create a snapshot}
\DoxyCodeLine{  delete NAME...                  Remove one or more snapshots}
\DoxyCodeLine{}
\DoxyCodeLine{Environment Commands:}
\DoxyCodeLine{  show-environment                Dump environment}
\DoxyCodeLine{  set-environment NAME=VALUE...   Set one or more environment variables}
\DoxyCodeLine{  unset-environment NAME...       Unset one or more environment variables}
\DoxyCodeLine{  import-environment NAME...      Import all, one or more environment variables}
\DoxyCodeLine{}
\DoxyCodeLine{Manager Lifecycle Commands:}
\DoxyCodeLine{  daemon-reload                   Reload systemd manager configuration}
\DoxyCodeLine{  daemon-reexec                   Reexecute systemd manager}
\DoxyCodeLine{}
\DoxyCodeLine{System Commands:}
\DoxyCodeLine{  is-system-running               Check whether system is fully running}
\DoxyCodeLine{  default                         Enter system default mode}
\DoxyCodeLine{  rescue                          Enter system rescue mode}
\DoxyCodeLine{  emergency                       Enter system emergency mode}
\DoxyCodeLine{  halt                            Shut down and halt the system}
\DoxyCodeLine{  poweroff                        Shut down and power-off the system}
\DoxyCodeLine{  reboot [ARG]                    Shut down and reboot the system}
\DoxyCodeLine{  kexec                           Shut down and reboot the system with kexec}
\DoxyCodeLine{  exit                            Request user instance exit}
\DoxyCodeLine{  switch-root ROOT [INIT]         Change to a different root file system}
\DoxyCodeLine{  suspend                         Suspend the system}
\DoxyCodeLine{  hibernate                       Hibernate the system}
\DoxyCodeLine{  hybrid-sleep                    Hibernate and suspend the system}
\end{DoxyCode}
 \subsection*{cat /etc/frr/daemons}

\#\#\# 
\begin{DoxyCode}{0}
\DoxyCodeLine{zebra=yes}
\DoxyCodeLine{bgpd=no}
\DoxyCodeLine{ospfd=no}
\DoxyCodeLine{ospf6d=no}
\DoxyCodeLine{ripd=no}
\DoxyCodeLine{ripngd=no}
\DoxyCodeLine{isisd=no}
\DoxyCodeLine{pimd=no}
\DoxyCodeLine{ldpd=no}
\DoxyCodeLine{nhrpd=no}
\DoxyCodeLine{eigrpd=no}
\DoxyCodeLine{babeld=no}
\DoxyCodeLine{sharpd=no}
\DoxyCodeLine{pbrd=no}
\end{DoxyCode}
 
\begin{DoxyCode}{0}
\DoxyCodeLine{Usage: cat [OPTION]... [FILE]...}
\DoxyCodeLine{Concatenate FILE(s), or standard input, to standard output.}
\DoxyCodeLine{}
\DoxyCodeLine{  -A, --show-all           equivalent to -vET}
\DoxyCodeLine{  -b, --number-nonblank    number nonempty output lines, overrides -n}
\DoxyCodeLine{  -e                       equivalent to -vE}
\DoxyCodeLine{  -E, --show-ends          display \$ at end of each line}
\DoxyCodeLine{  -n, --number             number all output lines}
\DoxyCodeLine{  -s, --squeeze-blank      suppress repeated empty output lines}
\DoxyCodeLine{  -t                       equivalent to -vT}
\DoxyCodeLine{  -T, --show-tabs          display TAB characters as \string^I}
\DoxyCodeLine{  -u                       (ignored)}
\DoxyCodeLine{  -v, --show-nonprinting   use \string^ and M- notation, except for LFD and TAB}
\DoxyCodeLine{      --help     display this help and exit}
\DoxyCodeLine{      --version  output version information and exit}
\DoxyCodeLine{}
\DoxyCodeLine{With no FILE, or when FILE is -, read standard input.}
\DoxyCodeLine{}
\DoxyCodeLine{Examples:}
\DoxyCodeLine{  cat f - g  Output f's contents, then standard input, then g's contents.}
\DoxyCodeLine{  cat        Copy standard input to standard output.}
\DoxyCodeLine{}
\DoxyCodeLine{GNU coreutils online help: <http://www.gnu.org/software/coreutils/>}
\DoxyCodeLine{Full documentation at: <http://www.gnu.org/software/coreutils/cat>}
\DoxyCodeLine{or available locally via: info '(coreutils) cat invocation'}
\DoxyCodeLine{}
\DoxyCodeLine{zebra=yes}
\DoxyCodeLine{bgpd=no}
\DoxyCodeLine{ospfd=no}
\DoxyCodeLine{ospf6d=no}
\DoxyCodeLine{ripd=no}
\DoxyCodeLine{ripngd=no}
\DoxyCodeLine{isisd=no}
\DoxyCodeLine{pimd=no}
\DoxyCodeLine{ldpd=no}
\DoxyCodeLine{nhrpd=no}
\DoxyCodeLine{eigrpd=no}
\DoxyCodeLine{babeld=no}
\DoxyCodeLine{sharpd=no}
\DoxyCodeLine{pbrd=no}
\end{DoxyCode}
 \subsection*{cat /etc/frr/frr.conf}

\#\#\# 
\begin{DoxyCode}{0}
\DoxyCodeLine{frr version 4.0+cl3u2}
\DoxyCodeLine{frr defaults datacenter}
\DoxyCodeLine{hostname oob-mgmt-server}
\DoxyCodeLine{username cumulus nopassword}
\DoxyCodeLine{!}
\DoxyCodeLine{service integrated-vtysh-config}
\DoxyCodeLine{!}
\DoxyCodeLine{log syslog informational}
\DoxyCodeLine{!}
\DoxyCodeLine{interface eth2}
\DoxyCodeLine{ ipv6 nd ra-interval 10}
\DoxyCodeLine{ no ipv6 nd suppress-ra}
\DoxyCodeLine{!}
\DoxyCodeLine{interface eth3}
\DoxyCodeLine{ ipv6 nd ra-interval 10}
\DoxyCodeLine{ no ipv6 nd suppress-ra}
\DoxyCodeLine{!}
\DoxyCodeLine{line vty}
\DoxyCodeLine{!}
\end{DoxyCode}
 
\begin{DoxyCode}{0}
\DoxyCodeLine{Usage: cat [OPTION]... [FILE]...}
\DoxyCodeLine{Concatenate FILE(s), or standard input, to standard output.}
\DoxyCodeLine{}
\DoxyCodeLine{  -A, --show-all           equivalent to -vET}
\DoxyCodeLine{  -b, --number-nonblank    number nonempty output lines, overrides -n}
\DoxyCodeLine{  -e                       equivalent to -vE}
\DoxyCodeLine{  -E, --show-ends          display \$ at end of each line}
\DoxyCodeLine{  -n, --number             number all output lines}
\DoxyCodeLine{  -s, --squeeze-blank      suppress repeated empty output lines}
\DoxyCodeLine{  -t                       equivalent to -vT}
\DoxyCodeLine{  -T, --show-tabs          display TAB characters as \string^I}
\DoxyCodeLine{  -u                       (ignored)}
\DoxyCodeLine{  -v, --show-nonprinting   use \string^ and M- notation, except for LFD and TAB}
\DoxyCodeLine{      --help     display this help and exit}
\DoxyCodeLine{      --version  output version information and exit}
\DoxyCodeLine{}
\DoxyCodeLine{With no FILE, or when FILE is -, read standard input.}
\DoxyCodeLine{}
\DoxyCodeLine{Examples:}
\DoxyCodeLine{  cat f - g  Output f's contents, then standard input, then g's contents.}
\DoxyCodeLine{  cat        Copy standard input to standard output.}
\DoxyCodeLine{}
\DoxyCodeLine{GNU coreutils online help: <http://www.gnu.org/software/coreutils/>}
\DoxyCodeLine{Full documentation at: <http://www.gnu.org/software/coreutils/cat>}
\DoxyCodeLine{or available locally via: info '(coreutils) cat invocation'}
\DoxyCodeLine{}
\DoxyCodeLine{frr version 4.0+cl3u2}
\DoxyCodeLine{frr defaults datacenter}
\DoxyCodeLine{hostname oob-mgmt-server}
\DoxyCodeLine{username cumulus nopassword}
\DoxyCodeLine{!}
\DoxyCodeLine{service integrated-vtysh-config}
\DoxyCodeLine{!}
\DoxyCodeLine{log syslog informational}
\DoxyCodeLine{!}
\DoxyCodeLine{interface eth2}
\DoxyCodeLine{ ipv6 nd ra-interval 10}
\DoxyCodeLine{ no ipv6 nd suppress-ra}
\DoxyCodeLine{!}
\DoxyCodeLine{interface eth3}
\DoxyCodeLine{ ipv6 nd ra-interval 10}
\DoxyCodeLine{ no ipv6 nd suppress-ra}
\DoxyCodeLine{!}
\DoxyCodeLine{line vty}
\DoxyCodeLine{!}
\end{DoxyCode}
 \subsection*{net show configuration}

\#\#\# 
\begin{DoxyCode}{0}
\DoxyCodeLine{frr version 4.0+cl3u2}
\DoxyCodeLine{}
\DoxyCodeLine{frr defaults datacenter}
\DoxyCodeLine{}
\DoxyCodeLine{hostname oob-mgmt-server}
\DoxyCodeLine{}
\DoxyCodeLine{username cumulus nopassword}
\DoxyCodeLine{}
\DoxyCodeLine{service integrated-vtysh-config}
\DoxyCodeLine{}
\DoxyCodeLine{log syslog informational}
\DoxyCodeLine{}
\DoxyCodeLine{interface eth2}
\DoxyCodeLine{  ipv6 nd ra-interval 10}
\DoxyCodeLine{  no ipv6 nd suppress-ra}
\DoxyCodeLine{}
\DoxyCodeLine{interface eth3}
\DoxyCodeLine{  ipv6 nd ra-interval 10}
\DoxyCodeLine{  no ipv6 nd suppress-ra}
\DoxyCodeLine{}
\DoxyCodeLine{line vty}
\DoxyCodeLine{}
\DoxyCodeLine{snmp-server}
\DoxyCodeLine{  listening-address localhost}
\DoxyCodeLine{}
\DoxyCodeLine{dns}
\DoxyCodeLine{  }
\DoxyCodeLine{  nameserver}
\DoxyCodeLine{    8.8.8.8}
\DoxyCodeLine{}
\DoxyCodeLine{interface lo}
\DoxyCodeLine{  address 10.0.0.41/32}
\DoxyCodeLine{  address fd00::41/128}
\DoxyCodeLine{}
\DoxyCodeLine{interface eth0}
\DoxyCodeLine{  address 10.255.0.1}
\DoxyCodeLine{  netmask 255.255.0.0}
\DoxyCodeLine{  gateway 10.255.0.3}
\DoxyCodeLine{}
\DoxyCodeLine{interface eth1}
\DoxyCodeLine{  address 192.168.0.254}
\DoxyCodeLine{  netmask 255.255.0.0}
\DoxyCodeLine{}
\DoxyCodeLine{time}
\DoxyCodeLine{  }
\DoxyCodeLine{  zone}
\DoxyCodeLine{    Etc/UTC}
\DoxyCodeLine{  }
\DoxyCodeLine{  ntp}
\DoxyCodeLine{    }
\DoxyCodeLine{    servers}
\DoxyCodeLine{      0.cumulusnetworks.pool.ntp.org iburst}
\DoxyCodeLine{      1.cumulusnetworks.pool.ntp.org iburst}
\DoxyCodeLine{      2.cumulusnetworks.pool.ntp.org iburst}
\DoxyCodeLine{      3.cumulusnetworks.pool.ntp.org iburst}
\DoxyCodeLine{    }
\DoxyCodeLine{    source}
\DoxyCodeLine{      eth0}
\DoxyCodeLine{}
\DoxyCodeLine{dot1x}
\DoxyCodeLine{  mab-activation-delay 30}
\DoxyCodeLine{  eap-reauth-period 0}
\DoxyCodeLine{  }
\DoxyCodeLine{  radius}
\DoxyCodeLine{    accounting-port 1813}
\DoxyCodeLine{    authentication-port 1812}
\DoxyCodeLine{}
\DoxyCodeLine{}
\DoxyCodeLine{\# The above output is a summary of the configuration state of the switch.}
\DoxyCodeLine{\# Do not cut and paste this output into /etc/network/interfaces or any other}
\DoxyCodeLine{\# configuration file.  This output is intended to be used for troubleshooting}
\DoxyCodeLine{\# when you need to see a summary of configuration settings.}
\DoxyCodeLine{\#}
\DoxyCodeLine{\# Please use "net show configuration commands" for a configuration that}
\DoxyCodeLine{\# you can back up or copy and paste into a new device.}
\end{DoxyCode}
 
\begin{DoxyCode}{0}
\DoxyCodeLine{The following commands contain keyword(s) 'configuration', 'show'}
\DoxyCodeLine{}
\DoxyCodeLine{    net show configuration [commands|files|acl|bgp|multicast|ospf|ospf6]}
\DoxyCodeLine{    net show configuration dhcp [json]}
\DoxyCodeLine{    net show configuration dns}
\DoxyCodeLine{    net show configuration dot1x}
\DoxyCodeLine{    net show configuration interface [<interface>]}
\DoxyCodeLine{    net show configuration ptp}
\DoxyCodeLine{    net show configuration snmp-server}
\DoxyCodeLine{    net show configuration syslog}
\end{DoxyCode}
 \subsection*{net show debugs}

\#\#\# 
\begin{DoxyCode}{0}
\DoxyCodeLine{Zebra debugging status:}
\end{DoxyCode}
 
\begin{DoxyCode}{0}
\DoxyCodeLine{The following commands contain keyword(s) 'debugs', 'show'}
\DoxyCodeLine{}
\DoxyCodeLine{    net show debugs}
\end{DoxyCode}
 \subsection*{ps -\/ax $\vert$ grep ospf}

\#\#\# 
\begin{DoxyCode}{0}
\DoxyCodeLine{20641 ?        S      0:00 grep ospf}
\end{DoxyCode}
 
\begin{DoxyCode}{0}
\DoxyCodeLine{Usage: grep [OPTION]... PATTERN [FILE]...}
\DoxyCodeLine{Search for PATTERN in each FILE or standard input.}
\DoxyCodeLine{PATTERN is, by default, a basic regular expression (BRE).}
\DoxyCodeLine{Example: grep -i 'hello world' menu.h main.c}
\DoxyCodeLine{}
\DoxyCodeLine{Regexp selection and interpretation:}
\DoxyCodeLine{  -E, --extended-regexp     PATTERN is an extended regular expression (ERE)}
\DoxyCodeLine{  -F, --fixed-strings       PATTERN is a set of newline-separated fixed strings}
\DoxyCodeLine{  -G, --basic-regexp        PATTERN is a basic regular expression (BRE)}
\DoxyCodeLine{  -P, --perl-regexp         PATTERN is a Perl regular expression}
\DoxyCodeLine{  -e, --regexp=PATTERN      use PATTERN for matching}
\DoxyCodeLine{  -f, --file=FILE           obtain PATTERN from FILE}
\DoxyCodeLine{  -i, --ignore-case         ignore case distinctions}
\DoxyCodeLine{  -w, --word-regexp         force PATTERN to match only whole words}
\DoxyCodeLine{  -x, --line-regexp         force PATTERN to match only whole lines}
\DoxyCodeLine{  -z, --null-data           a data line ends in 0 byte, not newline}
\DoxyCodeLine{}
\DoxyCodeLine{Miscellaneous:}
\DoxyCodeLine{  -s, --no-messages         suppress error messages}
\DoxyCodeLine{  -v, --invert-match        select non-matching lines}
\DoxyCodeLine{  -V, --version             display version information and exit}
\DoxyCodeLine{      --help                display this help text and exit}
\DoxyCodeLine{}
\DoxyCodeLine{Output control:}
\DoxyCodeLine{  -m, --max-count=NUM       stop after NUM matches}
\DoxyCodeLine{  -b, --byte-offset         print the byte offset with output lines}
\DoxyCodeLine{  -n, --line-number         print line number with output lines}
\DoxyCodeLine{      --line-buffered       flush output on every line}
\DoxyCodeLine{  -H, --with-filename       print the file name for each match}
\DoxyCodeLine{  -h, --no-filename         suppress the file name prefix on output}
\DoxyCodeLine{      --label=LABEL         use LABEL as the standard input file name prefix}
\DoxyCodeLine{  -o, --only-matching       show only the part of a line matching PATTERN}
\DoxyCodeLine{  -q, --quiet, --silent     suppress all normal output}
\DoxyCodeLine{      --binary-files=TYPE   assume that binary files are TYPE;}
\DoxyCodeLine{                            TYPE is 'binary', 'text', or 'without-match'}
\DoxyCodeLine{  -a, --text                equivalent to --binary-files=text}
\DoxyCodeLine{  -I                        equivalent to --binary-files=without-match}
\DoxyCodeLine{  -d, --directories=ACTION  how to handle directories;}
\DoxyCodeLine{                            ACTION is 'read', 'recurse', or 'skip'}
\DoxyCodeLine{  -D, --devices=ACTION      how to handle devices, FIFOs and sockets;}
\DoxyCodeLine{                            ACTION is 'read' or 'skip'}
\DoxyCodeLine{  -r, --recursive           like --directories=recurse}
\DoxyCodeLine{  -R, --dereference-recursive  likewise, but follow all symlinks}
\DoxyCodeLine{      --include=FILE\_PATTERN  search only files that match FILE\_PATTERN}
\DoxyCodeLine{      --exclude=FILE\_PATTERN  skip files and directories matching FILE\_PATTERN}
\DoxyCodeLine{      --exclude-from=FILE   skip files matching any file pattern from FILE}
\DoxyCodeLine{      --exclude-dir=PATTERN  directories that match PATTERN will be skipped.}
\DoxyCodeLine{  -L, --files-without-match  print only names of FILEs containing no match}
\DoxyCodeLine{  -l, --files-with-matches  print only names of FILEs containing matches}
\DoxyCodeLine{  -c, --count               print only a count of matching lines per FILE}
\DoxyCodeLine{  -T, --initial-tab         make tabs line up (if needed)}
\DoxyCodeLine{  -Z, --null                print 0 byte after FILE name}
\DoxyCodeLine{}
\DoxyCodeLine{Context control:}
\DoxyCodeLine{  -B, --before-context=NUM  print NUM lines of leading context}
\DoxyCodeLine{  -A, --after-context=NUM   print NUM lines of trailing context}
\DoxyCodeLine{  -C, --context=NUM         print NUM lines of output context}
\DoxyCodeLine{  -NUM                      same as --context=NUM}
\DoxyCodeLine{      --color[=WHEN],}
\DoxyCodeLine{      --colour[=WHEN]       use markers to highlight the matching strings;}
\DoxyCodeLine{                            WHEN is 'always', 'never', or 'auto'}
\DoxyCodeLine{  -U, --binary              do not strip CR characters at EOL (MSDOS/Windows)}
\DoxyCodeLine{  -u, --unix-byte-offsets   report offsets as if CRs were not there}
\DoxyCodeLine{                            (MSDOS/Windows)}
\DoxyCodeLine{}
\DoxyCodeLine{'egrep' means 'grep -E'.  'fgrep' means 'grep -F'.}
\DoxyCodeLine{Direct invocation as either 'egrep' or 'fgrep' is deprecated.}
\DoxyCodeLine{When FILE is -, read standard input.  With no FILE, read . if a command-line}
\DoxyCodeLine{-r is given, - otherwise.  If fewer than two FILEs are given, assume -h.}
\DoxyCodeLine{Exit status is 0 if any line is selected, 1 otherwise;}
\DoxyCodeLine{if any error occurs and -q is not given, the exit status is 2.}
\DoxyCodeLine{}
\DoxyCodeLine{Report bugs to: bug-grep@gnu.org}
\DoxyCodeLine{GNU Grep home page: <http://www.gnu.org/software/grep/>}
\DoxyCodeLine{General help using GNU software: <http://www.gnu.org/gethelp/>}
\DoxyCodeLine{}
\DoxyCodeLine{20641 ?        S      0:00 grep ospf}
\end{DoxyCode}
 \subsection*{sudo tcpdump -\/v -\/i DP ip proto ospf}

\#\#\# 
\begin{DoxyCode}{0}
\DoxyCodeLine{tcpdump: \%DP: No such device exists}
\DoxyCodeLine{(SIOCGIFHWADDR: No such device)}
\end{DoxyCode}
 
\begin{DoxyCode}{0}
\DoxyCodeLine{tcpdump version 4.9.2}
\DoxyCodeLine{libpcap version 1.8.1}
\DoxyCodeLine{OpenSSL 1.0.1t  3 May 2016}
\DoxyCodeLine{Usage: tcpdump [-aAbdDefhHIJKlLnNOpqStuUvxX\#] [ -B size ] [ -c count ]}
\DoxyCodeLine{        [ -C file\_size ] [ -E algo:secret ] [ -F file ] [ -G seconds ]}
\DoxyCodeLine{        [ -i interface ] [ -j tstamptype ] [ -M secret ] [ --number ]}
\DoxyCodeLine{        [ -Q in|out|inout ]}
\DoxyCodeLine{        [ -r file ] [ -s snaplen ] [ --time-stamp-precision precision ]}
\DoxyCodeLine{        [ --immediate-mode ] [ -T type ] [ --version ] [ -V file ]}
\DoxyCodeLine{        [ -w file ] [ -W filecount ] [ -y datalinktype ] [ -z postrotate-command ]}
\DoxyCodeLine{        [ -Z user ] [ expression ]}
\end{DoxyCode}
 \subsection*{net show ospf6 neighbor \mbox{[}detail$\vert$drchoice\mbox{]}}

\#\#\# 
\begin{DoxyCode}{0}
\DoxyCodeLine{./run.commands: line 2572: drchoice]: command not found}
\DoxyCodeLine{ERROR: Command not found.}
\DoxyCodeLine{}
\DoxyCodeLine{    net show ospf6 neighbor [detail}
\DoxyCodeLine{                   \string^ Invalid value here.}
\DoxyCodeLine{}
\DoxyCodeLine{Use "net help KEYWORD(s)" to list all options that use KEYWORD(s).}
\end{DoxyCode}
 
\begin{DoxyCode}{0}
\DoxyCodeLine{./run.commands: line 2576: drchoice]: command not found}
\DoxyCodeLine{ERROR: Command not found.}
\DoxyCodeLine{}
\DoxyCodeLine{    net show ospf6 neighbor [detail}
\DoxyCodeLine{                   \string^ Invalid value here.}
\DoxyCodeLine{}
\DoxyCodeLine{Use "net help KEYWORD(s)" to list all options that use KEYWORD(s).}
\DoxyCodeLine{}
\DoxyCodeLine{./run.commands: line 2572: drchoice]: command not found}
\DoxyCodeLine{ERROR: Command not found.}
\DoxyCodeLine{}
\DoxyCodeLine{    net show ospf6 neighbor [detail}
\DoxyCodeLine{                   \string^ Invalid value here.}
\DoxyCodeLine{}
\DoxyCodeLine{Use "net help KEYWORD(s)" to list all options that use KEYWORD(s).}
\end{DoxyCode}
 \subsection*{net show route ospf6}

\#\#\# 
\begin{DoxyCode}{0}
\DoxyCodeLine{RIB entry for ospf6}
\DoxyCodeLine{===================}
\end{DoxyCode}
 
\begin{DoxyCode}{0}
\DoxyCodeLine{The following commands contain keyword(s) 'route', 'ospf6', 'show'}
\DoxyCodeLine{}
\DoxyCodeLine{    net show ospf6 database adv-router <ipv4> [detail|dump|internal]}
\DoxyCodeLine{    net show ospf6 linkstate router <ipv4>}
\DoxyCodeLine{    net show ospf6 route (intra-area|inter-area|external-1|external-2) [detail]}
\DoxyCodeLine{    net show ospf6 route <ipv6/prefixlen> longer [detail]}
\DoxyCodeLine{    net show ospf6 route <ipv6/prefixlen> match [detail]}
\DoxyCodeLine{    net show ospf6 route [<ipv6>|<ipv6/prefixlen>|detail|summary]}
\DoxyCodeLine{    net show route (bgp|connected|kernel|ospf|ospf6|pim|rip|static|summary|supernets-only|table) [json]}
\DoxyCodeLine{    net show route vrf <text> (bgp|connected|kernel|ospf|ospf6|pim|rip|static|summary|supernets-only|table) [json]}
\end{DoxyCode}
 \subsection*{net show ospf6 database detail}

\#\#\# 
\begin{DoxyCode}{0}
\DoxyCodeLine{ospf6d is not running}
\end{DoxyCode}
 
\begin{DoxyCode}{0}
\DoxyCodeLine{The following commands contain keyword(s) 'database', 'detail', 'ospf6', 'show'}
\DoxyCodeLine{}
\DoxyCodeLine{    net show ospf6 database [detail|dump|internal]}
\DoxyCodeLine{    net show ospf6 database adv-router <ipv4> [detail|dump|internal]}
\DoxyCodeLine{    net show ospf6 database linkstate-id <ipv4-id> [detail|dump|internal]}
\DoxyCodeLine{    net show ospf6 database self-originated [detail|dump|internal]}
\end{DoxyCode}
 \subsection*{net show bfd sessions}

\#\#\# 
\begin{DoxyCode}{0}
\DoxyCodeLine{ERROR: Command not found.}
\DoxyCodeLine{}
\DoxyCodeLine{    net show bfd sessions}
\DoxyCodeLine{             \string^ Invalid value here.}
\DoxyCodeLine{}
\DoxyCodeLine{Use "net help KEYWORD(s)" to list all options that use KEYWORD(s).}
\end{DoxyCode}
 
\begin{DoxyCode}{0}
\DoxyCodeLine{ERROR: There are no commands with keyword(s) 'sessions', 'show', 'bfd'}
\end{DoxyCode}
 \subsection*{net show bfd sessions detail}

\#\#\# 
\begin{DoxyCode}{0}
\DoxyCodeLine{ERROR: Command not found.}
\DoxyCodeLine{}
\DoxyCodeLine{    net show bfd sessions detail}
\DoxyCodeLine{             \string^ Invalid value here.}
\DoxyCodeLine{}
\DoxyCodeLine{Use "net help KEYWORD(s)" to list all options that use KEYWORD(s).}
\end{DoxyCode}
 
\begin{DoxyCode}{0}
\DoxyCodeLine{ERROR: There are no commands with keyword(s) 'sessions', 'detail', 'show', 'bfd'}
\end{DoxyCode}
 