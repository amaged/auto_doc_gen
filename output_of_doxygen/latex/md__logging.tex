\subsection*{net example syslog}

\#\#\# 
\begin{DoxyCode}{0}
\DoxyCodeLine{Scenario}
\DoxyCodeLine{========}
\DoxyCodeLine{}
\DoxyCodeLine{  +-------------+        +-------------+}
\DoxyCodeLine{  |  syslog     |        |  syslog     |}
\DoxyCodeLine{  |  server1    |        |  server2    |}
\DoxyCodeLine{  |  1.1.1.1    |        |  2:: (IPv6) |}
\DoxyCodeLine{  | udp port 43 |        | tcp port 45 |}
\DoxyCodeLine{  +-------------+        +-------------+}
\DoxyCodeLine{        | swp20                 | swp25}
\DoxyCodeLine{        |                       |}
\DoxyCodeLine{        |                       |}
\DoxyCodeLine{        | swp2                  |}
\DoxyCodeLine{  +-------------+               |}
\DoxyCodeLine{  |   switch1   |---------------+}
\DoxyCodeLine{  +-------------+ swp1}
\DoxyCodeLine{}
\DoxyCodeLine{}
\DoxyCodeLine{WARNING: Improper syslog configuration can disrupt}
\DoxyCodeLine{logging for other processes.}
\DoxyCodeLine{}
\DoxyCodeLine{}
\DoxyCodeLine{net commands}
\DoxyCodeLine{============}
\DoxyCodeLine{switch1\# net add syslog host ipv4 1.1.1.1 port udp 43}
\DoxyCodeLine{switch1\# net add syslog host ipv6 2:: port tcp 45}
\DoxyCodeLine{switch1\# net pending}
\DoxyCodeLine{switch1\# net commit}
\DoxyCodeLine{}
\DoxyCodeLine{UDP is the default transport protocol.  This first}
\DoxyCodeLine{command above could just be:}
\DoxyCodeLine{}
\DoxyCodeLine{switch1\# net add syslog host ipv4 1.1.1.1 port 43}
\DoxyCodeLine{}
\DoxyCodeLine{}
\DoxyCodeLine{Verification}
\DoxyCodeLine{============}
\DoxyCodeLine{switch1\# net show configuration syslog}
\end{DoxyCode}
 
\begin{DoxyCode}{0}
\DoxyCodeLine{The following commands contain keyword(s) 'example', 'syslog'}
\DoxyCodeLine{}
\DoxyCodeLine{    net example syslog}
\end{DoxyCode}
 \subsection*{net example snmp-\/server}

\#\#\# 
\begin{DoxyCode}{0}
\DoxyCodeLine{Scenario}
\DoxyCodeLine{========}
\DoxyCodeLine{}
\DoxyCodeLine{  +------------+}
\DoxyCodeLine{  |  Network   |}
\DoxyCodeLine{  | Management |}
\DoxyCodeLine{  |  Station   |}
\DoxyCodeLine{  | 10.1.1.2   |}
\DoxyCodeLine{  +------------+}
\DoxyCodeLine{        |}
\DoxyCodeLine{        |}
\DoxyCodeLine{        |}
\DoxyCodeLine{        |}
\DoxyCodeLine{  +-------------+}
\DoxyCodeLine{  | 10.1.1.1    |}
\DoxyCodeLine{  |  switch1    |}
\DoxyCodeLine{  | SNMP Agent  |}
\DoxyCodeLine{  |Cumulus Linux|}
\DoxyCodeLine{  +-------------+}
\DoxyCodeLine{}
\DoxyCodeLine{Network Management Station IP is  10.1.1.2/24}
\DoxyCodeLine{Cumulus Linux SWITCH server IP is 10.1.1.1/24}
\DoxyCodeLine{}
\DoxyCodeLine{The following commands start the snmpd daemon listening on a specific}
\DoxyCodeLine{IP address using the command listening-address.  They also restrict}
\DoxyCodeLine{access to a specific MIB tree OID using the viewname command from a}
\DoxyCodeLine{specific NMS system using the readonly-community access keyword.}
\DoxyCodeLine{In addition, they allow access only with a specific readonly-community}
\DoxyCodeLine{password.}
\DoxyCodeLine{}
\DoxyCodeLine{net commands}
\DoxyCodeLine{============}
\DoxyCodeLine{switch1\# net add snmp-server listening-address localhost 10.1.1.1}
\DoxyCodeLine{switch1\# net add snmp-server viewname mysystemview included .1.3.6.1.2.1.1}
\DoxyCodeLine{switch1\# net add snmp-server readonly-community mysecretpassword access 10.1.1.2 view mysystemview}
\DoxyCodeLine{switch1\# net pending}
\DoxyCodeLine{switch1\# net commit}
\DoxyCodeLine{}
\DoxyCodeLine{}
\DoxyCodeLine{Verification}
\DoxyCodeLine{============}
\DoxyCodeLine{switch1\# net show snmp-server status}
\DoxyCodeLine{switch1\# net show config snmp-server}
\DoxyCodeLine{switch1\# net show config commands}
\end{DoxyCode}
 
\begin{DoxyCode}{0}
\DoxyCodeLine{The following commands contain keyword(s) 'snmp-server', 'example'}
\DoxyCodeLine{}
\DoxyCodeLine{    net example snmp-server}
\end{DoxyCode}
 \subsection*{cat /etc/rsyslog.d/11-\/remotesyslog.\+conf}

\#\#\# 
\begin{DoxyCode}{0}
\DoxyCodeLine{\# This file was automatically generated by NCLU.}
\end{DoxyCode}
 
\begin{DoxyCode}{0}
\DoxyCodeLine{Usage: cat [OPTION]... [FILE]...}
\DoxyCodeLine{Concatenate FILE(s), or standard input, to standard output.}
\DoxyCodeLine{}
\DoxyCodeLine{  -A, --show-all           equivalent to -vET}
\DoxyCodeLine{  -b, --number-nonblank    number nonempty output lines, overrides -n}
\DoxyCodeLine{  -e                       equivalent to -vE}
\DoxyCodeLine{  -E, --show-ends          display \$ at end of each line}
\DoxyCodeLine{  -n, --number             number all output lines}
\DoxyCodeLine{  -s, --squeeze-blank      suppress repeated empty output lines}
\DoxyCodeLine{  -t                       equivalent to -vT}
\DoxyCodeLine{  -T, --show-tabs          display TAB characters as \string^I}
\DoxyCodeLine{  -u                       (ignored)}
\DoxyCodeLine{  -v, --show-nonprinting   use \string^ and M- notation, except for LFD and TAB}
\DoxyCodeLine{      --help     display this help and exit}
\DoxyCodeLine{      --version  output version information and exit}
\DoxyCodeLine{}
\DoxyCodeLine{With no FILE, or when FILE is -, read standard input.}
\DoxyCodeLine{}
\DoxyCodeLine{Examples:}
\DoxyCodeLine{  cat f - g  Output f's contents, then standard input, then g's contents.}
\DoxyCodeLine{  cat        Copy standard input to standard output.}
\DoxyCodeLine{}
\DoxyCodeLine{GNU coreutils online help: <http://www.gnu.org/software/coreutils/>}
\DoxyCodeLine{Full documentation at: <http://www.gnu.org/software/coreutils/cat>}
\DoxyCodeLine{or available locally via: info '(coreutils) cat invocation'}
\DoxyCodeLine{}
\DoxyCodeLine{\# This file was automatically generated by NCLU.}
\end{DoxyCode}
 \subsection*{cat /etc/rsyslog.conf}

\#\#\# 
\begin{DoxyCode}{0}
\DoxyCodeLine{\#  /etc/rsyslog.conf    Configuration file for rsyslog.}
\DoxyCodeLine{\#}
\DoxyCodeLine{\#           For more information see}
\DoxyCodeLine{\#           /usr/share/doc/rsyslog-doc/html/rsyslog\_conf.html}
\DoxyCodeLine{}
\DoxyCodeLine{}
\DoxyCodeLine{\#\#\#\#\#\#\#\#\#\#\#\#\#\#\#\#\#}
\DoxyCodeLine{\#\#\#\# MODULES \#\#\#\#}
\DoxyCodeLine{\#\#\#\#\#\#\#\#\#\#\#\#\#\#\#\#\#}
\DoxyCodeLine{}
\DoxyCodeLine{\$ModLoad imuxsock \# provides support for local system logging}
\DoxyCodeLine{\$ModLoad imklog   \# provides kernel logging support}
\DoxyCodeLine{\#\$ModLoad immark  \# provides --MARK-- message capability}
\DoxyCodeLine{}
\DoxyCodeLine{\# provides UDP syslog reception}
\DoxyCodeLine{\#\$ModLoad imudp}
\DoxyCodeLine{\#\$UDPServerRun 514}
\DoxyCodeLine{}
\DoxyCodeLine{\# provides TCP syslog reception}
\DoxyCodeLine{\#\$ModLoad imtcp}
\DoxyCodeLine{\#\$InputTCPServerRun 514}
\DoxyCodeLine{}
\DoxyCodeLine{\# To forward messages to a remote host with Management VRF:}
\DoxyCodeLine{\#action(type="omfwd" Target="name or ip here" Device="mgmt" Port="514" Protocol="udp")}
\DoxyCodeLine{}
\DoxyCodeLine{\#\#\#\#\#\#\#\#\#\#\#\#\#\#\#\#\#\#\#\#\#\#\#\#\#\#\#}
\DoxyCodeLine{\#\#\#\# GLOBAL DIRECTIVES \#\#\#\#}
\DoxyCodeLine{\#\#\#\#\#\#\#\#\#\#\#\#\#\#\#\#\#\#\#\#\#\#\#\#\#\#\#}
\DoxyCodeLine{}
\DoxyCodeLine{\#}
\DoxyCodeLine{\# Use traditional timestamp format.}
\DoxyCodeLine{\# To enable high precision timestamps, comment out the following line.}
\DoxyCodeLine{\#}
\DoxyCodeLine{\$ActionFileDefaultTemplate RSYSLOG\_TraditionalFileFormat}
\DoxyCodeLine{}
\DoxyCodeLine{\#}
\DoxyCodeLine{\# Set the default permissions for all log files.}
\DoxyCodeLine{\#}
\DoxyCodeLine{\$FileOwner logger}
\DoxyCodeLine{\$FileGroup logging}
\DoxyCodeLine{\$FileCreateMode 0666}
\DoxyCodeLine{\$DirCreateMode 0777}
\DoxyCodeLine{\$Umask 0022}
\DoxyCodeLine{}
\DoxyCodeLine{\#}
\DoxyCodeLine{\# Where to place spool and state files}
\DoxyCodeLine{\#}
\DoxyCodeLine{\$WorkDirectory /var/spool/rsyslog}
\DoxyCodeLine{}
\DoxyCodeLine{\#}
\DoxyCodeLine{\# Include all config files in /etc/rsyslog.d/}
\DoxyCodeLine{\#}
\DoxyCodeLine{\$IncludeConfig /etc/rsyslog.d/*.conf}
\DoxyCodeLine{}
\DoxyCodeLine{}
\DoxyCodeLine{\#\#\#\#\#\#\#\#\#\#\#\#\#\#\#}
\DoxyCodeLine{\#\#\#\# RULES \#\#\#\#}
\DoxyCodeLine{\#\#\#\#\#\#\#\#\#\#\#\#\#\#\#}
\DoxyCodeLine{}
\DoxyCodeLine{\# Cumulus Linux program specific logfiles are configured by files in}
\DoxyCodeLine{\# the /etc/rsyslog.d/ directory.  Precision timestamping is one of those}
\DoxyCodeLine{\# rules.}
\DoxyCodeLine{}
\DoxyCodeLine{\# Cumulus departs from the traditional UNIX set of log files, and logs}
\DoxyCodeLine{\# all standard messages to a single file, syslog.}
\DoxyCodeLine{}
\DoxyCodeLine{\# Edit /etc/rsyslog.d/99-syslog.conf to change this behavior, if you wish to}
\DoxyCodeLine{\# add your own rules below this line.}
\DoxyCodeLine{}
\DoxyCodeLine{\# Other Cumulus Linux logs are controlled by files in /etc/rsyslog.d/}
\DoxyCodeLine{}
\DoxyCodeLine{\# It is simpler and avoids merging configuration changes, if you add your}
\DoxyCodeLine{\# own rules in new files in the /etc/rsyslog.d directory}
\DoxyCodeLine{}
\DoxyCodeLine{\#}
\DoxyCodeLine{\# Emergencies are sent to everybody logged in.}
\DoxyCodeLine{\#}
\DoxyCodeLine{*.emerg             :omusrmsg:*}
\end{DoxyCode}
 
\begin{DoxyCode}{0}
\DoxyCodeLine{Usage: cat [OPTION]... [FILE]...}
\DoxyCodeLine{Concatenate FILE(s), or standard input, to standard output.}
\DoxyCodeLine{}
\DoxyCodeLine{  -A, --show-all           equivalent to -vET}
\DoxyCodeLine{  -b, --number-nonblank    number nonempty output lines, overrides -n}
\DoxyCodeLine{  -e                       equivalent to -vE}
\DoxyCodeLine{  -E, --show-ends          display \$ at end of each line}
\DoxyCodeLine{  -n, --number             number all output lines}
\DoxyCodeLine{  -s, --squeeze-blank      suppress repeated empty output lines}
\DoxyCodeLine{  -t                       equivalent to -vT}
\DoxyCodeLine{  -T, --show-tabs          display TAB characters as \string^I}
\DoxyCodeLine{  -u                       (ignored)}
\DoxyCodeLine{  -v, --show-nonprinting   use \string^ and M- notation, except for LFD and TAB}
\DoxyCodeLine{      --help     display this help and exit}
\DoxyCodeLine{      --version  output version information and exit}
\DoxyCodeLine{}
\DoxyCodeLine{With no FILE, or when FILE is -, read standard input.}
\DoxyCodeLine{}
\DoxyCodeLine{Examples:}
\DoxyCodeLine{  cat f - g  Output f's contents, then standard input, then g's contents.}
\DoxyCodeLine{  cat        Copy standard input to standard output.}
\DoxyCodeLine{}
\DoxyCodeLine{GNU coreutils online help: <http://www.gnu.org/software/coreutils/>}
\DoxyCodeLine{Full documentation at: <http://www.gnu.org/software/coreutils/cat>}
\DoxyCodeLine{or available locally via: info '(coreutils) cat invocation'}
\DoxyCodeLine{}
\DoxyCodeLine{\#  /etc/rsyslog.conf    Configuration file for rsyslog.}
\DoxyCodeLine{\#}
\DoxyCodeLine{\#           For more information see}
\DoxyCodeLine{\#           /usr/share/doc/rsyslog-doc/html/rsyslog\_conf.html}
\DoxyCodeLine{}
\DoxyCodeLine{}
\DoxyCodeLine{\#\#\#\#\#\#\#\#\#\#\#\#\#\#\#\#\#}
\DoxyCodeLine{\#\#\#\# MODULES \#\#\#\#}
\DoxyCodeLine{\#\#\#\#\#\#\#\#\#\#\#\#\#\#\#\#\#}
\DoxyCodeLine{}
\DoxyCodeLine{\$ModLoad imuxsock \# provides support for local system logging}
\DoxyCodeLine{\$ModLoad imklog   \# provides kernel logging support}
\DoxyCodeLine{\#\$ModLoad immark  \# provides --MARK-- message capability}
\DoxyCodeLine{}
\DoxyCodeLine{\# provides UDP syslog reception}
\DoxyCodeLine{\#\$ModLoad imudp}
\DoxyCodeLine{\#\$UDPServerRun 514}
\DoxyCodeLine{}
\DoxyCodeLine{\# provides TCP syslog reception}
\DoxyCodeLine{\#\$ModLoad imtcp}
\DoxyCodeLine{\#\$InputTCPServerRun 514}
\DoxyCodeLine{}
\DoxyCodeLine{\# To forward messages to a remote host with Management VRF:}
\DoxyCodeLine{\#action(type="omfwd" Target="name or ip here" Device="mgmt" Port="514" Protocol="udp")}
\DoxyCodeLine{}
\DoxyCodeLine{\#\#\#\#\#\#\#\#\#\#\#\#\#\#\#\#\#\#\#\#\#\#\#\#\#\#\#}
\DoxyCodeLine{\#\#\#\# GLOBAL DIRECTIVES \#\#\#\#}
\DoxyCodeLine{\#\#\#\#\#\#\#\#\#\#\#\#\#\#\#\#\#\#\#\#\#\#\#\#\#\#\#}
\DoxyCodeLine{}
\DoxyCodeLine{\#}
\DoxyCodeLine{\# Use traditional timestamp format.}
\DoxyCodeLine{\# To enable high precision timestamps, comment out the following line.}
\DoxyCodeLine{\#}
\DoxyCodeLine{\$ActionFileDefaultTemplate RSYSLOG\_TraditionalFileFormat}
\DoxyCodeLine{}
\DoxyCodeLine{\#}
\DoxyCodeLine{\# Set the default permissions for all log files.}
\DoxyCodeLine{\#}
\DoxyCodeLine{\$FileOwner logger}
\DoxyCodeLine{\$FileGroup logging}
\DoxyCodeLine{\$FileCreateMode 0666}
\DoxyCodeLine{\$DirCreateMode 0777}
\DoxyCodeLine{\$Umask 0022}
\DoxyCodeLine{}
\DoxyCodeLine{\#}
\DoxyCodeLine{\# Where to place spool and state files}
\DoxyCodeLine{\#}
\DoxyCodeLine{\$WorkDirectory /var/spool/rsyslog}
\DoxyCodeLine{}
\DoxyCodeLine{\#}
\DoxyCodeLine{\# Include all config files in /etc/rsyslog.d/}
\DoxyCodeLine{\#}
\DoxyCodeLine{\$IncludeConfig /etc/rsyslog.d/*.conf}
\DoxyCodeLine{}
\DoxyCodeLine{}
\DoxyCodeLine{\#\#\#\#\#\#\#\#\#\#\#\#\#\#\#}
\DoxyCodeLine{\#\#\#\# RULES \#\#\#\#}
\DoxyCodeLine{\#\#\#\#\#\#\#\#\#\#\#\#\#\#\#}
\DoxyCodeLine{}
\DoxyCodeLine{\# Cumulus Linux program specific logfiles are configured by files in}
\DoxyCodeLine{\# the /etc/rsyslog.d/ directory.  Precision timestamping is one of those}
\DoxyCodeLine{\# rules.}
\DoxyCodeLine{}
\DoxyCodeLine{\# Cumulus departs from the traditional UNIX set of log files, and logs}
\DoxyCodeLine{\# all standard messages to a single file, syslog.}
\DoxyCodeLine{}
\DoxyCodeLine{\# Edit /etc/rsyslog.d/99-syslog.conf to change this behavior, if you wish to}
\DoxyCodeLine{\# add your own rules below this line.}
\DoxyCodeLine{}
\DoxyCodeLine{\# Other Cumulus Linux logs are controlled by files in /etc/rsyslog.d/}
\DoxyCodeLine{}
\DoxyCodeLine{\# It is simpler and avoids merging configuration changes, if you add your}
\DoxyCodeLine{\# own rules in new files in the /etc/rsyslog.d directory}
\DoxyCodeLine{}
\DoxyCodeLine{\#}
\DoxyCodeLine{\# Emergencies are sent to everybody logged in.}
\DoxyCodeLine{\#}
\DoxyCodeLine{*.emerg             :omusrmsg:*}
\end{DoxyCode}
 \subsection*{sudo systemctl status rsyslog.\+service}

\#\#\# 
\begin{DoxyCode}{0}
\DoxyCodeLine{● rsyslog.service - System Logging Service}
\DoxyCodeLine{   Loaded: loaded (/lib/systemd/system/rsyslog.service; enabled)}
\DoxyCodeLine{   Active: active (running) since Mon 2018-07-30 22:34:56 UTC; 1h 23min ago}
\DoxyCodeLine{     Docs: man:rsyslogd(8)}
\DoxyCodeLine{           http://www.rsyslog.com/doc/}
\DoxyCodeLine{ Main PID: 410 (rsyslogd)}
\DoxyCodeLine{   CGroup: /system.slice/rsyslog.service}
\DoxyCodeLine{           └─410 /usr/sbin/rsyslogd -n}
\DoxyCodeLine{}
\DoxyCodeLine{Jul 30 22:34:56 oob-mgmt-server rsyslogd[410]: rsyslogd: warning: ~ action is deprecated, consider using the 'stop' statement instead [try http://www.rsyslog.com/e/2307 ]}
\DoxyCodeLine{Jul 30 22:34:56 oob-mgmt-server systemd[1]: Started System Logging Service.}
\end{DoxyCode}
 
\begin{DoxyCode}{0}
\DoxyCodeLine{systemctl [OPTIONS...] \{COMMAND\} ...}
\DoxyCodeLine{}
\DoxyCodeLine{Query or send control commands to the systemd manager.}
\DoxyCodeLine{}
\DoxyCodeLine{  -h --help           Show this help}
\DoxyCodeLine{     --version        Show package version}
\DoxyCodeLine{     --system         Connect to system manager}
\DoxyCodeLine{     --user           Connect to user service manager}
\DoxyCodeLine{  -H --host=[USER@]HOST}
\DoxyCodeLine{                      Operate on remote host}
\DoxyCodeLine{  -M --machine=CONTAINER}
\DoxyCodeLine{                      Operate on local container}
\DoxyCodeLine{  -t --type=TYPE      List only units of a particular type}
\DoxyCodeLine{     --state=STATE    List only units with particular LOAD or SUB or ACTIVE state}
\DoxyCodeLine{  -p --property=NAME  Show only properties by this name}
\DoxyCodeLine{  -a --all            Show all loaded units/properties, including dead/empty}
\DoxyCodeLine{                      ones. To list all units installed on the system, use}
\DoxyCodeLine{                      the 'list-unit-files' command instead.}
\DoxyCodeLine{  -l --full           Don't ellipsize unit names on output}
\DoxyCodeLine{  -r --recursive      Show unit list of host and local containers}
\DoxyCodeLine{     --reverse        Show reverse dependencies with 'list-dependencies'}
\DoxyCodeLine{     --job-mode=MODE  Specify how to deal with already queued jobs, when}
\DoxyCodeLine{                      queueing a new job}
\DoxyCodeLine{     --show-types     When showing sockets, explicitly show their type}
\DoxyCodeLine{  -i --ignore-inhibitors}
\DoxyCodeLine{                      When shutting down or sleeping, ignore inhibitors}
\DoxyCodeLine{     --kill-who=WHO   Who to send signal to}
\DoxyCodeLine{  -s --signal=SIGNAL  Which signal to send}
\DoxyCodeLine{  -q --quiet          Suppress output}
\DoxyCodeLine{     --no-block       Do not wait until operation finished}
\DoxyCodeLine{     --no-wall        Don't send wall message before halt/power-off/reboot}
\DoxyCodeLine{     --no-reload      When enabling/disabling unit files, don't reload daemon}
\DoxyCodeLine{                      configuration}
\DoxyCodeLine{     --no-legend      Do not print a legend (column headers and hints)}
\DoxyCodeLine{     --no-pager       Do not pipe output into a pager}
\DoxyCodeLine{     --no-ask-password}
\DoxyCodeLine{                      Do not ask for system passwords}
\DoxyCodeLine{     --global         Enable/disable unit files globally}
\DoxyCodeLine{     --runtime        Enable unit files only temporarily until next reboot}
\DoxyCodeLine{  -f --force          When enabling unit files, override existing symlinks}
\DoxyCodeLine{                      When shutting down, execute action immediately}
\DoxyCodeLine{     --preset-mode=   Specifies whether fully apply presets, or only enable,}
\DoxyCodeLine{                      or only disable}
\DoxyCodeLine{     --root=PATH      Enable unit files in the specified root directory}
\DoxyCodeLine{  -n --lines=INTEGER  Number of journal entries to show}
\DoxyCodeLine{  -o --output=STRING  Change journal output mode (short, short-monotonic,}
\DoxyCodeLine{                      verbose, export, json, json-pretty, json-sse, cat)}
\DoxyCodeLine{     --plain          Print unit dependencies as a list instead of a tree}
\DoxyCodeLine{}
\DoxyCodeLine{Unit Commands:}
\DoxyCodeLine{  list-units [PATTERN...]         List loaded units}
\DoxyCodeLine{  list-sockets [PATTERN...]       List loaded sockets ordered by address}
\DoxyCodeLine{  list-timers [PATTERN...]        List loaded timers ordered by next elapse}
\DoxyCodeLine{  start NAME...                   Start (activate) one or more units}
\DoxyCodeLine{  stop NAME...                    Stop (deactivate) one or more units}
\DoxyCodeLine{  reload NAME...                  Reload one or more units}
\DoxyCodeLine{  restart NAME...                 Start or restart one or more units}
\DoxyCodeLine{  try-restart NAME...             Restart one or more units if active}
\DoxyCodeLine{  reload-or-restart NAME...       Reload one or more units if possible,}
\DoxyCodeLine{                                  otherwise start or restart}
\DoxyCodeLine{  reload-or-try-restart NAME...   Reload one or more units if possible,}
\DoxyCodeLine{                                  otherwise restart if active}
\DoxyCodeLine{  isolate NAME                    Start one unit and stop all others}
\DoxyCodeLine{  kill NAME...                    Send signal to processes of a unit}
\DoxyCodeLine{  is-active PATTERN...            Check whether units are active}
\DoxyCodeLine{  is-failed PATTERN...            Check whether units are failed}
\DoxyCodeLine{  status [PATTERN...|PID...]      Show runtime status of one or more units}
\DoxyCodeLine{  show [PATTERN...|JOB...]        Show properties of one or more}
\DoxyCodeLine{                                  units/jobs or the manager}
\DoxyCodeLine{  cat PATTERN...                  Show files and drop-ins of one or more units}
\DoxyCodeLine{  set-property NAME ASSIGNMENT... Sets one or more properties of a unit}
\DoxyCodeLine{  help PATTERN...|PID...          Show manual for one or more units}
\DoxyCodeLine{  reset-failed [PATTERN...]       Reset failed state for all, one, or more}
\DoxyCodeLine{                                  units}
\DoxyCodeLine{  list-dependencies [NAME]        Recursively show units which are required}
\DoxyCodeLine{                                  or wanted by this unit or by which this}
\DoxyCodeLine{                                  unit is required or wanted}
\DoxyCodeLine{}
\DoxyCodeLine{Unit File Commands:}
\DoxyCodeLine{  list-unit-files [PATTERN...]    List installed unit files}
\DoxyCodeLine{  enable NAME...                  Enable one or more unit files}
\DoxyCodeLine{  disable NAME...                 Disable one or more unit files}
\DoxyCodeLine{  reenable NAME...                Reenable one or more unit files}
\DoxyCodeLine{  preset NAME...                  Enable/disable one or more unit files}
\DoxyCodeLine{                                  based on preset configuration}
\DoxyCodeLine{  preset-all                      Enable/disable all unit files based on}
\DoxyCodeLine{                                  preset configuration}
\DoxyCodeLine{  is-enabled NAME...              Check whether unit files are enabled}
\DoxyCodeLine{}
\DoxyCodeLine{  mask NAME...                    Mask one or more units}
\DoxyCodeLine{  unmask NAME...                  Unmask one or more units}
\DoxyCodeLine{  link PATH...                    Link one or more units files into}
\DoxyCodeLine{                                  the search path}
\DoxyCodeLine{  get-default                     Get the name of the default target}
\DoxyCodeLine{  set-default NAME                Set the default target}
\DoxyCodeLine{}
\DoxyCodeLine{Machine Commands:}
\DoxyCodeLine{  list-machines [PATTERN...]      List local containers and host}
\DoxyCodeLine{}
\DoxyCodeLine{Job Commands:}
\DoxyCodeLine{  list-jobs [PATTERN...]          List jobs}
\DoxyCodeLine{  cancel [JOB...]                 Cancel all, one, or more jobs}
\DoxyCodeLine{}
\DoxyCodeLine{Snapshot Commands:}
\DoxyCodeLine{  snapshot [NAME]                 Create a snapshot}
\DoxyCodeLine{  delete NAME...                  Remove one or more snapshots}
\DoxyCodeLine{}
\DoxyCodeLine{Environment Commands:}
\DoxyCodeLine{  show-environment                Dump environment}
\DoxyCodeLine{  set-environment NAME=VALUE...   Set one or more environment variables}
\DoxyCodeLine{  unset-environment NAME...       Unset one or more environment variables}
\DoxyCodeLine{  import-environment NAME...      Import all, one or more environment variables}
\DoxyCodeLine{}
\DoxyCodeLine{Manager Lifecycle Commands:}
\DoxyCodeLine{  daemon-reload                   Reload systemd manager configuration}
\DoxyCodeLine{  daemon-reexec                   Reexecute systemd manager}
\DoxyCodeLine{}
\DoxyCodeLine{System Commands:}
\DoxyCodeLine{  is-system-running               Check whether system is fully running}
\DoxyCodeLine{  default                         Enter system default mode}
\DoxyCodeLine{  rescue                          Enter system rescue mode}
\DoxyCodeLine{  emergency                       Enter system emergency mode}
\DoxyCodeLine{  halt                            Shut down and halt the system}
\DoxyCodeLine{  poweroff                        Shut down and power-off the system}
\DoxyCodeLine{  reboot [ARG]                    Shut down and reboot the system}
\DoxyCodeLine{  kexec                           Shut down and reboot the system with kexec}
\DoxyCodeLine{  exit                            Request user instance exit}
\DoxyCodeLine{  switch-root ROOT [INIT]         Change to a different root file system}
\DoxyCodeLine{  suspend                         Suspend the system}
\DoxyCodeLine{  hibernate                       Hibernate the system}
\DoxyCodeLine{  hybrid-sleep                    Hibernate and suspend the system}
\end{DoxyCode}
 