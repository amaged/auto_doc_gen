\subsection*{net show mroute}

\#\#\# 
\begin{DoxyCode}{0}
\DoxyCodeLine{pimd is not running}
\end{DoxyCode}
 
\begin{DoxyCode}{0}
\DoxyCodeLine{The following commands contain keyword(s) 'mroute', 'show'}
\DoxyCodeLine{}
\DoxyCodeLine{    net show mroute [count|json]}
\DoxyCodeLine{    net show mroute vrf <text> [count|json]}
\end{DoxyCode}
 \subsection*{net show pim local-\/membership}

\#\#\# 
\begin{DoxyCode}{0}
\DoxyCodeLine{pimd is not running}
\end{DoxyCode}
 
\begin{DoxyCode}{0}
\DoxyCodeLine{The following commands contain keyword(s) 'local-membership', 'show', 'pim'}
\DoxyCodeLine{}
\DoxyCodeLine{    net show pim (join|local-membership|rpf|rp-info|upstream|upstream-join-desired|upstream-rpf) [json]}
\DoxyCodeLine{    net show pim vrf <text> (join|local-membership|rpf|rp-info|upstream|upstream-join-desired|upstream-rpf) [json]}
\end{DoxyCode}
 \subsection*{net show pim state}

\#\#\# 
\begin{DoxyCode}{0}
\DoxyCodeLine{pimd is not running}
\end{DoxyCode}
 
\begin{DoxyCode}{0}
\DoxyCodeLine{The following commands contain keyword(s) 'state', 'show', 'pim'}
\DoxyCodeLine{}
\DoxyCodeLine{    net show pim state <ipv4-source> <ipv4-mcast-group> [json]}
\DoxyCodeLine{    net show pim state <ipv4-source> [json]}
\DoxyCodeLine{    net show pim state [json]}
\DoxyCodeLine{    net show pim vrf <text> state <ipv4-source> <ipv4-mcast-group> [json]}
\DoxyCodeLine{    net show pim vrf <text> state <ipv4-source> [json]}
\DoxyCodeLine{    net show pim vrf <text> state [json]}
\end{DoxyCode}
 \subsection*{net show pim upstream}

\#\#\# 
\begin{DoxyCode}{0}
\DoxyCodeLine{pimd is not running}
\end{DoxyCode}
 
\begin{DoxyCode}{0}
\DoxyCodeLine{The following commands contain keyword(s) 'upstream', 'show', 'pim'}
\DoxyCodeLine{}
\DoxyCodeLine{    net show pim (join|local-membership|rpf|rp-info|upstream|upstream-join-desired|upstream-rpf) [json]}
\DoxyCodeLine{    net show pim vrf <text> (join|local-membership|rpf|rp-info|upstream|upstream-join-desired|upstream-rpf) [json]}
\end{DoxyCode}
 \subsection*{net show pim upstream-\/join-\/desired}

\#\#\# 
\begin{DoxyCode}{0}
\DoxyCodeLine{pimd is not running}
\end{DoxyCode}
 
\begin{DoxyCode}{0}
\DoxyCodeLine{The following commands contain keyword(s) 'upstream-join-desired', 'show', 'pim'}
\DoxyCodeLine{}
\DoxyCodeLine{    net show pim (join|local-membership|rpf|rp-info|upstream|upstream-join-desired|upstream-rpf) [json]}
\DoxyCodeLine{    net show pim vrf <text> (join|local-membership|rpf|rp-info|upstream|upstream-join-desired|upstream-rpf) [json]}
\end{DoxyCode}
 \subsection*{net show igmp groups}

\#\#\# 
\begin{DoxyCode}{0}
\DoxyCodeLine{pimd is not running}
\end{DoxyCode}
 
\begin{DoxyCode}{0}
\DoxyCodeLine{The following commands contain keyword(s) 'groups', 'show', 'igmp'}
\DoxyCodeLine{}
\DoxyCodeLine{    net show igmp groups [json]}
\DoxyCodeLine{    net show igmp vrf <text> groups [json]}
\end{DoxyCode}
 \subsection*{net show igmp sources}

\#\#\# 
\begin{DoxyCode}{0}
\DoxyCodeLine{pimd is not running}
\end{DoxyCode}
 
\begin{DoxyCode}{0}
\DoxyCodeLine{The following commands contain keyword(s) 'sources', 'show', 'igmp'}
\DoxyCodeLine{}
\DoxyCodeLine{    net show igmp (join|sources)}
\DoxyCodeLine{    net show igmp vrf <text> (join|sources)}
\end{DoxyCode}
 \subsection*{cat /etc/network/interfaces}

\#\#\# 
\begin{DoxyCode}{0}
\DoxyCodeLine{auto lo}
\DoxyCodeLine{iface lo inet loopback}
\DoxyCodeLine{    address 10.0.0.41/32}
\DoxyCodeLine{    address fd00::41/128}
\DoxyCodeLine{}
\DoxyCodeLine{auto eth0}
\DoxyCodeLine{iface eth0 inet static}
\DoxyCodeLine{    address 10.255.0.1}
\DoxyCodeLine{    netmask 255.255.0.0}
\DoxyCodeLine{    gateway 10.255.0.3}
\DoxyCodeLine{auto eth1}
\DoxyCodeLine{iface eth1 inet static}
\DoxyCodeLine{    address 192.168.0.254}
\DoxyCodeLine{    netmask 255.255.0.0}
\DoxyCodeLine{}
\DoxyCodeLine{auto eth2}
\DoxyCodeLine{iface eth2}
\DoxyCodeLine{}
\DoxyCodeLine{auto eth3}
\DoxyCodeLine{iface eth3}
\end{DoxyCode}
 
\begin{DoxyCode}{0}
\DoxyCodeLine{Usage: cat [OPTION]... [FILE]...}
\DoxyCodeLine{Concatenate FILE(s), or standard input, to standard output.}
\DoxyCodeLine{}
\DoxyCodeLine{  -A, --show-all           equivalent to -vET}
\DoxyCodeLine{  -b, --number-nonblank    number nonempty output lines, overrides -n}
\DoxyCodeLine{  -e                       equivalent to -vE}
\DoxyCodeLine{  -E, --show-ends          display \$ at end of each line}
\DoxyCodeLine{  -n, --number             number all output lines}
\DoxyCodeLine{  -s, --squeeze-blank      suppress repeated empty output lines}
\DoxyCodeLine{  -t                       equivalent to -vT}
\DoxyCodeLine{  -T, --show-tabs          display TAB characters as \string^I}
\DoxyCodeLine{  -u                       (ignored)}
\DoxyCodeLine{  -v, --show-nonprinting   use \string^ and M- notation, except for LFD and TAB}
\DoxyCodeLine{      --help     display this help and exit}
\DoxyCodeLine{      --version  output version information and exit}
\DoxyCodeLine{}
\DoxyCodeLine{With no FILE, or when FILE is -, read standard input.}
\DoxyCodeLine{}
\DoxyCodeLine{Examples:}
\DoxyCodeLine{  cat f - g  Output f's contents, then standard input, then g's contents.}
\DoxyCodeLine{  cat        Copy standard input to standard output.}
\DoxyCodeLine{}
\DoxyCodeLine{GNU coreutils online help: <http://www.gnu.org/software/coreutils/>}
\DoxyCodeLine{Full documentation at: <http://www.gnu.org/software/coreutils/cat>}
\DoxyCodeLine{or available locally via: info '(coreutils) cat invocation'}
\DoxyCodeLine{}
\DoxyCodeLine{auto lo}
\DoxyCodeLine{iface lo inet loopback}
\DoxyCodeLine{    address 10.0.0.41/32}
\DoxyCodeLine{    address fd00::41/128}
\DoxyCodeLine{}
\DoxyCodeLine{auto eth0}
\DoxyCodeLine{iface eth0 inet static}
\DoxyCodeLine{    address 10.255.0.1}
\DoxyCodeLine{    netmask 255.255.0.0}
\DoxyCodeLine{    gateway 10.255.0.3}
\DoxyCodeLine{auto eth1}
\DoxyCodeLine{iface eth1 inet static}
\DoxyCodeLine{    address 192.168.0.254}
\DoxyCodeLine{    netmask 255.255.0.0}
\DoxyCodeLine{}
\DoxyCodeLine{auto eth2}
\DoxyCodeLine{iface eth2}
\DoxyCodeLine{}
\DoxyCodeLine{auto eth3}
\DoxyCodeLine{iface eth3}
\end{DoxyCode}
 \subsection*{cat /etc/frr/frr.conf}

\#\#\# 
\begin{DoxyCode}{0}
\DoxyCodeLine{frr version 4.0+cl3u2}
\DoxyCodeLine{frr defaults datacenter}
\DoxyCodeLine{hostname oob-mgmt-server}
\DoxyCodeLine{username cumulus nopassword}
\DoxyCodeLine{!}
\DoxyCodeLine{service integrated-vtysh-config}
\DoxyCodeLine{!}
\DoxyCodeLine{log syslog informational}
\DoxyCodeLine{!}
\DoxyCodeLine{interface eth2}
\DoxyCodeLine{ ipv6 nd ra-interval 10}
\DoxyCodeLine{ no ipv6 nd suppress-ra}
\DoxyCodeLine{!}
\DoxyCodeLine{interface eth3}
\DoxyCodeLine{ ipv6 nd ra-interval 10}
\DoxyCodeLine{ no ipv6 nd suppress-ra}
\DoxyCodeLine{!}
\DoxyCodeLine{line vty}
\DoxyCodeLine{!}
\end{DoxyCode}
 
\begin{DoxyCode}{0}
\DoxyCodeLine{Usage: cat [OPTION]... [FILE]...}
\DoxyCodeLine{Concatenate FILE(s), or standard input, to standard output.}
\DoxyCodeLine{}
\DoxyCodeLine{  -A, --show-all           equivalent to -vET}
\DoxyCodeLine{  -b, --number-nonblank    number nonempty output lines, overrides -n}
\DoxyCodeLine{  -e                       equivalent to -vE}
\DoxyCodeLine{  -E, --show-ends          display \$ at end of each line}
\DoxyCodeLine{  -n, --number             number all output lines}
\DoxyCodeLine{  -s, --squeeze-blank      suppress repeated empty output lines}
\DoxyCodeLine{  -t                       equivalent to -vT}
\DoxyCodeLine{  -T, --show-tabs          display TAB characters as \string^I}
\DoxyCodeLine{  -u                       (ignored)}
\DoxyCodeLine{  -v, --show-nonprinting   use \string^ and M- notation, except for LFD and TAB}
\DoxyCodeLine{      --help     display this help and exit}
\DoxyCodeLine{      --version  output version information and exit}
\DoxyCodeLine{}
\DoxyCodeLine{With no FILE, or when FILE is -, read standard input.}
\DoxyCodeLine{}
\DoxyCodeLine{Examples:}
\DoxyCodeLine{  cat f - g  Output f's contents, then standard input, then g's contents.}
\DoxyCodeLine{  cat        Copy standard input to standard output.}
\DoxyCodeLine{}
\DoxyCodeLine{GNU coreutils online help: <http://www.gnu.org/software/coreutils/>}
\DoxyCodeLine{Full documentation at: <http://www.gnu.org/software/coreutils/cat>}
\DoxyCodeLine{or available locally via: info '(coreutils) cat invocation'}
\DoxyCodeLine{}
\DoxyCodeLine{frr version 4.0+cl3u2}
\DoxyCodeLine{frr defaults datacenter}
\DoxyCodeLine{hostname oob-mgmt-server}
\DoxyCodeLine{username cumulus nopassword}
\DoxyCodeLine{!}
\DoxyCodeLine{service integrated-vtysh-config}
\DoxyCodeLine{!}
\DoxyCodeLine{log syslog informational}
\DoxyCodeLine{!}
\DoxyCodeLine{interface eth2}
\DoxyCodeLine{ ipv6 nd ra-interval 10}
\DoxyCodeLine{ no ipv6 nd suppress-ra}
\DoxyCodeLine{!}
\DoxyCodeLine{interface eth3}
\DoxyCodeLine{ ipv6 nd ra-interval 10}
\DoxyCodeLine{ no ipv6 nd suppress-ra}
\DoxyCodeLine{!}
\DoxyCodeLine{line vty}
\DoxyCodeLine{!}
\end{DoxyCode}
 \subsection*{net show mroute vrf blue}

\#\#\# 
\begin{DoxyCode}{0}
\DoxyCodeLine{pimd is not running}
\end{DoxyCode}
 
\begin{DoxyCode}{0}
\DoxyCodeLine{ERROR: There are no commands with keyword(s) 'mroute', 'show', 'blue', 'vrf'}
\end{DoxyCode}
 \subsection*{cl-\/resource-\/query $\vert$ grep Mcast}

\#\#\# 
\begin{DoxyCode}{0}
\end{DoxyCode}
 
\begin{DoxyCode}{0}
\DoxyCodeLine{Usage: grep [OPTION]... PATTERN [FILE]...}
\DoxyCodeLine{Search for PATTERN in each FILE or standard input.}
\DoxyCodeLine{PATTERN is, by default, a basic regular expression (BRE).}
\DoxyCodeLine{Example: grep -i 'hello world' menu.h main.c}
\DoxyCodeLine{}
\DoxyCodeLine{Regexp selection and interpretation:}
\DoxyCodeLine{  -E, --extended-regexp     PATTERN is an extended regular expression (ERE)}
\DoxyCodeLine{  -F, --fixed-strings       PATTERN is a set of newline-separated fixed strings}
\DoxyCodeLine{  -G, --basic-regexp        PATTERN is a basic regular expression (BRE)}
\DoxyCodeLine{  -P, --perl-regexp         PATTERN is a Perl regular expression}
\DoxyCodeLine{  -e, --regexp=PATTERN      use PATTERN for matching}
\DoxyCodeLine{  -f, --file=FILE           obtain PATTERN from FILE}
\DoxyCodeLine{  -i, --ignore-case         ignore case distinctions}
\DoxyCodeLine{  -w, --word-regexp         force PATTERN to match only whole words}
\DoxyCodeLine{  -x, --line-regexp         force PATTERN to match only whole lines}
\DoxyCodeLine{  -z, --null-data           a data line ends in 0 byte, not newline}
\DoxyCodeLine{}
\DoxyCodeLine{Miscellaneous:}
\DoxyCodeLine{  -s, --no-messages         suppress error messages}
\DoxyCodeLine{  -v, --invert-match        select non-matching lines}
\DoxyCodeLine{  -V, --version             display version information and exit}
\DoxyCodeLine{      --help                display this help text and exit}
\DoxyCodeLine{}
\DoxyCodeLine{Output control:}
\DoxyCodeLine{  -m, --max-count=NUM       stop after NUM matches}
\DoxyCodeLine{  -b, --byte-offset         print the byte offset with output lines}
\DoxyCodeLine{  -n, --line-number         print line number with output lines}
\DoxyCodeLine{      --line-buffered       flush output on every line}
\DoxyCodeLine{  -H, --with-filename       print the file name for each match}
\DoxyCodeLine{  -h, --no-filename         suppress the file name prefix on output}
\DoxyCodeLine{      --label=LABEL         use LABEL as the standard input file name prefix}
\DoxyCodeLine{  -o, --only-matching       show only the part of a line matching PATTERN}
\DoxyCodeLine{  -q, --quiet, --silent     suppress all normal output}
\DoxyCodeLine{      --binary-files=TYPE   assume that binary files are TYPE;}
\DoxyCodeLine{                            TYPE is 'binary', 'text', or 'without-match'}
\DoxyCodeLine{  -a, --text                equivalent to --binary-files=text}
\DoxyCodeLine{  -I                        equivalent to --binary-files=without-match}
\DoxyCodeLine{  -d, --directories=ACTION  how to handle directories;}
\DoxyCodeLine{                            ACTION is 'read', 'recurse', or 'skip'}
\DoxyCodeLine{  -D, --devices=ACTION      how to handle devices, FIFOs and sockets;}
\DoxyCodeLine{                            ACTION is 'read' or 'skip'}
\DoxyCodeLine{  -r, --recursive           like --directories=recurse}
\DoxyCodeLine{  -R, --dereference-recursive  likewise, but follow all symlinks}
\DoxyCodeLine{      --include=FILE\_PATTERN  search only files that match FILE\_PATTERN}
\DoxyCodeLine{      --exclude=FILE\_PATTERN  skip files and directories matching FILE\_PATTERN}
\DoxyCodeLine{      --exclude-from=FILE   skip files matching any file pattern from FILE}
\DoxyCodeLine{      --exclude-dir=PATTERN  directories that match PATTERN will be skipped.}
\DoxyCodeLine{  -L, --files-without-match  print only names of FILEs containing no match}
\DoxyCodeLine{  -l, --files-with-matches  print only names of FILEs containing matches}
\DoxyCodeLine{  -c, --count               print only a count of matching lines per FILE}
\DoxyCodeLine{  -T, --initial-tab         make tabs line up (if needed)}
\DoxyCodeLine{  -Z, --null                print 0 byte after FILE name}
\DoxyCodeLine{}
\DoxyCodeLine{Context control:}
\DoxyCodeLine{  -B, --before-context=NUM  print NUM lines of leading context}
\DoxyCodeLine{  -A, --after-context=NUM   print NUM lines of trailing context}
\DoxyCodeLine{  -C, --context=NUM         print NUM lines of output context}
\DoxyCodeLine{  -NUM                      same as --context=NUM}
\DoxyCodeLine{      --color[=WHEN],}
\DoxyCodeLine{      --colour[=WHEN]       use markers to highlight the matching strings;}
\DoxyCodeLine{                            WHEN is 'always', 'never', or 'auto'}
\DoxyCodeLine{  -U, --binary              do not strip CR characters at EOL (MSDOS/Windows)}
\DoxyCodeLine{  -u, --unix-byte-offsets   report offsets as if CRs were not there}
\DoxyCodeLine{                            (MSDOS/Windows)}
\DoxyCodeLine{}
\DoxyCodeLine{'egrep' means 'grep -E'.  'fgrep' means 'grep -F'.}
\DoxyCodeLine{Direct invocation as either 'egrep' or 'fgrep' is deprecated.}
\DoxyCodeLine{When FILE is -, read standard input.  With no FILE, read . if a command-line}
\DoxyCodeLine{-r is given, - otherwise.  If fewer than two FILEs are given, assume -h.}
\DoxyCodeLine{Exit status is 0 if any line is selected, 1 otherwise;}
\DoxyCodeLine{if any error occurs and -q is not given, the exit status is 2.}
\DoxyCodeLine{}
\DoxyCodeLine{Report bugs to: bug-grep@gnu.org}
\DoxyCodeLine{GNU Grep home page: <http://www.gnu.org/software/grep/>}
\DoxyCodeLine{General help using GNU software: <http://www.gnu.org/gethelp/>}
\DoxyCodeLine{close failed in file object destructor:}
\DoxyCodeLine{sys.excepthook is missing}
\DoxyCodeLine{lost sys.stderr}
\end{DoxyCode}
 \subsection*{net show msdp sa}

\#\#\# 
\begin{DoxyCode}{0}
\DoxyCodeLine{pimd is not running}
\end{DoxyCode}
 
\begin{DoxyCode}{0}
\DoxyCodeLine{The following commands contain keyword(s) 'show', 'msdp', 'sa'}
\DoxyCodeLine{}
\DoxyCodeLine{    net show msdp sa <ipv4> <ipv4-mcast-group> [json]}
\DoxyCodeLine{    net show msdp sa <ipv4> [json]}
\DoxyCodeLine{    net show msdp sa [detail] [json]}
\DoxyCodeLine{    net show msdp vrf <text> sa <ipv4> <ipv4-mcast-group> [json]}
\DoxyCodeLine{    net show msdp vrf <text> sa <ipv4> [json]}
\DoxyCodeLine{    net show msdp vrf <text> sa [detail] [json]}
\end{DoxyCode}
 