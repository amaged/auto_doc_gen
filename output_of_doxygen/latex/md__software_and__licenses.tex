\subsection*{cat /etc/lsb-\/release}

\#\#\# Check all the information you need 
\begin{DoxyCode}{0}
\DoxyCodeLine{DISTRIB\_ID="Cumulus Linux"}
\DoxyCodeLine{DISTRIB\_RELEASE=3.6.1}
\DoxyCodeLine{DISTRIB\_DESCRIPTION="Cumulus Linux 3.6.1"}
\end{DoxyCode}
 
\begin{DoxyCode}{0}
\DoxyCodeLine{Usage: cat [OPTION]... [FILE]...}
\DoxyCodeLine{Concatenate FILE(s), or standard input, to standard output.}
\DoxyCodeLine{}
\DoxyCodeLine{  -A, --show-all           equivalent to -vET}
\DoxyCodeLine{  -b, --number-nonblank    number nonempty output lines, overrides -n}
\DoxyCodeLine{  -e                       equivalent to -vE}
\DoxyCodeLine{  -E, --show-ends          display \$ at end of each line}
\DoxyCodeLine{  -n, --number             number all output lines}
\DoxyCodeLine{  -s, --squeeze-blank      suppress repeated empty output lines}
\DoxyCodeLine{  -t                       equivalent to -vT}
\DoxyCodeLine{  -T, --show-tabs          display TAB characters as \string^I}
\DoxyCodeLine{  -u                       (ignored)}
\DoxyCodeLine{  -v, --show-nonprinting   use \string^ and M- notation, except for LFD and TAB}
\DoxyCodeLine{      --help     display this help and exit}
\DoxyCodeLine{      --version  output version information and exit}
\DoxyCodeLine{}
\DoxyCodeLine{With no FILE, or when FILE is -, read standard input.}
\DoxyCodeLine{}
\DoxyCodeLine{Examples:}
\DoxyCodeLine{  cat f - g  Output f's contents, then standard input, then g's contents.}
\DoxyCodeLine{  cat        Copy standard input to standard output.}
\DoxyCodeLine{}
\DoxyCodeLine{GNU coreutils online help: <http://www.gnu.org/software/coreutils/>}
\DoxyCodeLine{Full documentation at: <http://www.gnu.org/software/coreutils/cat>}
\DoxyCodeLine{or available locally via: info '(coreutils) cat invocation'}
\DoxyCodeLine{}
\DoxyCodeLine{DISTRIB\_ID="Cumulus Linux"}
\DoxyCodeLine{DISTRIB\_RELEASE=3.6.1}
\DoxyCodeLine{DISTRIB\_DESCRIPTION="Cumulus Linux 3.6.1"}
\end{DoxyCode}
 \subsection*{cat /etc/os-\/release}

\#\#\# Check all the info ~\newline
 Another Line to check the info 
\begin{DoxyCode}{0}
\DoxyCodeLine{NAME="Cumulus Linux"}
\DoxyCodeLine{VERSION\_ID=3.6.1}
\DoxyCodeLine{VERSION="Cumulus Linux 3.6.1"}
\DoxyCodeLine{PRETTY\_NAME="Cumulus Linux"}
\DoxyCodeLine{ID=cumulus-linux}
\DoxyCodeLine{ID\_LIKE=debian}
\DoxyCodeLine{CPE\_NAME=cpe:/o:cumulusnetworks:cumulus\_linux:3.6.1}
\DoxyCodeLine{HOME\_URL="http://www.cumulusnetworks.com/"}
\DoxyCodeLine{SUPPORT\_URL="http://support.cumulusnetworks.com/"}
\end{DoxyCode}
 
\begin{DoxyCode}{0}
\DoxyCodeLine{Usage: cat [OPTION]... [FILE]...}
\DoxyCodeLine{Concatenate FILE(s), or standard input, to standard output.}
\DoxyCodeLine{}
\DoxyCodeLine{  -A, --show-all           equivalent to -vET}
\DoxyCodeLine{  -b, --number-nonblank    number nonempty output lines, overrides -n}
\DoxyCodeLine{  -e                       equivalent to -vE}
\DoxyCodeLine{  -E, --show-ends          display \$ at end of each line}
\DoxyCodeLine{  -n, --number             number all output lines}
\DoxyCodeLine{  -s, --squeeze-blank      suppress repeated empty output lines}
\DoxyCodeLine{  -t                       equivalent to -vT}
\DoxyCodeLine{  -T, --show-tabs          display TAB characters as \string^I}
\DoxyCodeLine{  -u                       (ignored)}
\DoxyCodeLine{  -v, --show-nonprinting   use \string^ and M- notation, except for LFD and TAB}
\DoxyCodeLine{      --help     display this help and exit}
\DoxyCodeLine{      --version  output version information and exit}
\DoxyCodeLine{}
\DoxyCodeLine{With no FILE, or when FILE is -, read standard input.}
\DoxyCodeLine{}
\DoxyCodeLine{Examples:}
\DoxyCodeLine{  cat f - g  Output f's contents, then standard input, then g's contents.}
\DoxyCodeLine{  cat        Copy standard input to standard output.}
\DoxyCodeLine{}
\DoxyCodeLine{GNU coreutils online help: <http://www.gnu.org/software/coreutils/>}
\DoxyCodeLine{Full documentation at: <http://www.gnu.org/software/coreutils/cat>}
\DoxyCodeLine{or available locally via: info '(coreutils) cat invocation'}
\DoxyCodeLine{}
\DoxyCodeLine{NAME="Cumulus Linux"}
\DoxyCodeLine{VERSION\_ID=3.6.1}
\DoxyCodeLine{VERSION="Cumulus Linux 3.6.1"}
\DoxyCodeLine{PRETTY\_NAME="Cumulus Linux"}
\DoxyCodeLine{ID=cumulus-linux}
\DoxyCodeLine{ID\_LIKE=debian}
\DoxyCodeLine{CPE\_NAME=cpe:/o:cumulusnetworks:cumulus\_linux:3.6.1}
\DoxyCodeLine{HOME\_URL="http://www.cumulusnetworks.com/"}
\DoxyCodeLine{SUPPORT\_URL="http://support.cumulusnetworks.com/"}
\end{DoxyCode}
 \subsection*{onie-\/select -\/d}

\#\#\# Check info /n Just trying to find a bug with the Forward slash here \+:) 
\begin{DoxyCode}{0}
\DoxyCodeLine{Boot mode: Cumulus Linux}
\end{DoxyCode}
 
\begin{DoxyCode}{0}
\DoxyCodeLine{Illegal option --}
\DoxyCodeLine{Illegal option -e}
\DoxyCodeLine{Illegal option -l}
\DoxyCodeLine{Illegal option --}
\DoxyCodeLine{Failure: Unexpected command option.}
\DoxyCodeLine{usage: onie-select [-h] [-v] [-d] [-f] [-r] [-i] [-k] [-p] [-c] [-n]}
\DoxyCodeLine{Use -h for more help.}
\DoxyCodeLine{}
\DoxyCodeLine{Boot mode: Cumulus Linux}
\end{DoxyCode}
 \subsection*{decode-\/syseeprom}

\#\#\# $\ast$ These additional info can be combined with blah blah ~\newline
 to generate blah blah for your system blah. 
\begin{DoxyCode}{0}
\DoxyCodeLine{TlvInfo Header:}
\DoxyCodeLine{   Id String:    TlvInfo}
\DoxyCodeLine{   Version:      1}
\DoxyCodeLine{   Total Length: 69}
\DoxyCodeLine{TLV Name             Code Len Value}
\DoxyCodeLine{-------------------- ---- --- -----}
\DoxyCodeLine{Vendor Name          0x2D  16 Cumulus Networks}
\DoxyCodeLine{Product Name         0x21   2 VX}
\DoxyCodeLine{Device Version       0x26   1 3}
\DoxyCodeLine{Part Number          0x22   5 3.5.3}
\DoxyCodeLine{MAC Addresses        0x2A   2 4}
\DoxyCodeLine{Base MAC Address     0x24   6 44:38:39:00:01:00}
\DoxyCodeLine{Serial Number        0x23  17 44:38:39:00:01:00}
\DoxyCodeLine{CRC-32               0xFE   4 0x19447E39}
\DoxyCodeLine{(checksum valid)}
\end{DoxyCode}
 
\begin{DoxyCode}{0}
\DoxyCodeLine{Usage: /usr/cumulus/bin/decode-syseeprom [-a][-r][-s [args]][-t <target>][-e][-m]}
\DoxyCodeLine{}
\DoxyCodeLine{Options:}
\DoxyCodeLine{  -h, --help    show this help message and exit}
\DoxyCodeLine{  -a            print the base mac address for switch interfaces}
\DoxyCodeLine{  -r            print the number of macs allocated for switch interfaces}
\DoxyCodeLine{  -s            set the eeprom content if the eeprom is writable. args can be}
\DoxyCodeLine{                supplied in command line in a comma separated list of the form}
\DoxyCodeLine{                '<field>=<value>, ...'.  ',' and '=' are illegal characters in}
\DoxyCodeLine{                field names and values. Fields that are not specified will}
\DoxyCodeLine{                default to their current values.  If args are supplied in}
\DoxyCodeLine{                command line, they will be written without confirmation.  If}
\DoxyCodeLine{                args is empty, the values will be prompted interactively.}
\DoxyCodeLine{  -j, --json    Display JSON output}
\DoxyCodeLine{  -t TARGET     select the target eeprom (board) for read or write operation,}
\DoxyCodeLine{                or select 'all' for read operation, default is 'board'}
\DoxyCodeLine{  -e, --serial  print device serial number}
\DoxyCodeLine{  -m            print the base mac address for management interfaces}
\DoxyCodeLine{  --init        clear and initialize board eeprom cache}
\end{DoxyCode}
 \subsection*{dmesg}

\#\#\# you know what it is 
\begin{DoxyCode}{0}
\DoxyCodeLine{[    0.000000] Initializing cgroup subsys cpuset}
\DoxyCodeLine{[    0.000000] Initializing cgroup subsys cpu}
\DoxyCodeLine{[    0.000000] Initializing cgroup subsys cpuacct}
\DoxyCodeLine{[    0.000000] Linux version 4.1.0-cl-7-amd64 (dev-support@cumulusnetworks.com) (gcc version 4.9.2 (Debian 4.9.2-10+deb8u1) ) \#1 SMP Debian 4.1.33-1+cl3u13 (2018-04-25)}
\DoxyCodeLine{[    0.000000] Command line: BOOT\_IMAGE=/boot/vmlinuz-4.1.0-cl-7-amd64 root=UUID=77e5d5f6-9568-4559-8513-aaf6e20d1b5f ro cl\_platform=cumulus\_vx console=ttyS0,115200n8 console=tty0 quiet}
\DoxyCodeLine{[    0.000000] e820: BIOS-provided physical RAM map:}
\DoxyCodeLine{[    0.000000] BIOS-e820: [mem 0x0000000000000000-0x000000000009fbff] usable}
\DoxyCodeLine{[    0.000000] BIOS-e820: [mem 0x000000000009fc00-0x000000000009ffff] reserved}
\DoxyCodeLine{[    0.000000] BIOS-e820: [mem 0x00000000000f0000-0x00000000000fffff] reserved}
\DoxyCodeLine{[    0.000000] BIOS-e820: [mem 0x0000000000100000-0x00000000bffddfff] usable}
\DoxyCodeLine{[    0.000000] BIOS-e820: [mem 0x00000000bffde000-0x00000000bfffffff] reserved}
\DoxyCodeLine{[    0.000000] BIOS-e820: [mem 0x00000000feffc000-0x00000000feffffff] reserved}
\DoxyCodeLine{[    0.000000] BIOS-e820: [mem 0x00000000fffc0000-0x00000000ffffffff] reserved}
\DoxyCodeLine{[    0.000000] BIOS-e820: [mem 0x0000000100000000-0x00000002bfffffff] usable}
\DoxyCodeLine{[    0.000000] NX (Execute Disable) protection: active}
\DoxyCodeLine{[    0.000000] SMBIOS 2.8 present.}
\DoxyCodeLine{[    0.000000] DMI: Bochs Bochs, BIOS Bochs }
\DoxyCodeLine{[    0.000000] e820: update [mem 0x00000000-0x00000fff] usable ==> reserved}
\DoxyCodeLine{[    0.000000] e820: remove [mem 0x000a0000-0x000fffff] usable}
\DoxyCodeLine{[    0.000000] e820: last\_pfn = 0x2c0000 max\_arch\_pfn = 0x400000000}
\DoxyCodeLine{[    0.000000] MTRR default type: write-back}
\DoxyCodeLine{[    0.000000] MTRR fixed ranges enabled:}
\DoxyCodeLine{[    0.000000]   00000-9FFFF write-back}
\DoxyCodeLine{[    0.000000]   A0000-BFFFF uncachable}
\DoxyCodeLine{[    0.000000]   C0000-FFFFF write-protect}
\DoxyCodeLine{[    0.000000] MTRR variable ranges enabled:}
\DoxyCodeLine{[    0.000000]   0 base 00C0000000 mask FFC0000000 uncachable}
\DoxyCodeLine{[    0.000000]   1 disabled}
\DoxyCodeLine{[    0.000000]   2 disabled}
\DoxyCodeLine{[    0.000000]   3 disabled}
\DoxyCodeLine{[    0.000000]   4 disabled}
\DoxyCodeLine{[    0.000000]   5 disabled}
\DoxyCodeLine{[    0.000000]   6 disabled}
\DoxyCodeLine{[    0.000000]   7 disabled}
\DoxyCodeLine{[    0.000000] e820: last\_pfn = 0xbffde max\_arch\_pfn = 0x400000000}
\DoxyCodeLine{[    0.000000] found SMP MP-table at [mem 0x000f6a60-0x000f6a6f] mapped at [ffff8800000f6a60]}
\DoxyCodeLine{[    0.000000] Base memory trampoline at [ffff880000099000] 99000 size 24576}
\DoxyCodeLine{[    0.000000] init\_memory\_mapping: [mem 0x00000000-0x000fffff]}
\DoxyCodeLine{[    0.000000]  [mem 0x00000000-0x000fffff] page 4k}
\DoxyCodeLine{[    0.000000] BRK [0x02014000, 0x02014fff] PGTABLE}
\DoxyCodeLine{[    0.000000] BRK [0x02015000, 0x02015fff] PGTABLE}
\DoxyCodeLine{[    0.000000] BRK [0x02016000, 0x02016fff] PGTABLE}
\DoxyCodeLine{[    0.000000] init\_memory\_mapping: [mem 0x2bfe00000-0x2bfffffff]}
\DoxyCodeLine{[    0.000000]  [mem 0x2bfe00000-0x2bfffffff] page 2M}
\DoxyCodeLine{[    0.000000] BRK [0x02017000, 0x02017fff] PGTABLE}
\DoxyCodeLine{[    0.000000] init\_memory\_mapping: [mem 0x2a0000000-0x2bfdfffff]}
\DoxyCodeLine{[    0.000000]  [mem 0x2a0000000-0x2bfdfffff] page 2M}
\DoxyCodeLine{[    0.000000] init\_memory\_mapping: [mem 0x00100000-0xbffddfff]}
\DoxyCodeLine{[    0.000000]  [mem 0x00100000-0x001fffff] page 4k}
\DoxyCodeLine{[    0.000000]  [mem 0x00200000-0xbfdfffff] page 2M}
\DoxyCodeLine{[    0.000000]  [mem 0xbfe00000-0xbffddfff] page 4k}
\DoxyCodeLine{[    0.000000] init\_memory\_mapping: [mem 0x100000000-0x29fffffff]}
\DoxyCodeLine{[    0.000000]  [mem 0x100000000-0x29fffffff] page 2M}
\DoxyCodeLine{[    0.000000] BRK [0x02018000, 0x02018fff] PGTABLE}
\DoxyCodeLine{[    0.000000] BRK [0x02019000, 0x02019fff] PGTABLE}
\DoxyCodeLine{[    0.000000] RAMDISK: [mem 0x37518000-0x37a83fff]}
\DoxyCodeLine{[    0.000000] ACPI: Early table checksum verification disabled}
\DoxyCodeLine{[    0.000000] ACPI: RSDP 0x00000000000F6860 000014 (v00 BOCHS )}
\DoxyCodeLine{[    0.000000] ACPI: RSDT 0x00000000BFFE18D9 000034 (v01 BOCHS  BXPCRSDT 00000001 BXPC 00000001)}
\DoxyCodeLine{[    0.000000] ACPI: FACP 0x00000000BFFE0E65 000074 (v01 BOCHS  BXPCFACP 00000001 BXPC 00000001)}
\DoxyCodeLine{[    0.000000] ACPI: DSDT 0x00000000BFFE0040 000E25 (v01 BOCHS  BXPCDSDT 00000001 BXPC 00000001)}
\DoxyCodeLine{[    0.000000] ACPI: FACS 0x00000000BFFE0000 000040}
\DoxyCodeLine{[    0.000000] ACPI: SSDT 0x00000000BFFE0ED9 000948 (v01 BOCHS  BXPCSSDT 00000001 BXPC 00000001)}
\DoxyCodeLine{[    0.000000] ACPI: APIC 0x00000000BFFE1821 000080 (v01 BOCHS  BXPCAPIC 00000001 BXPC 00000001)}
\DoxyCodeLine{[    0.000000] ACPI: HPET 0x00000000BFFE18A1 000038 (v01 BOCHS  BXPCHPET 00000001 BXPC 00000001)}
\DoxyCodeLine{[    0.000000] ACPI: Local APIC address 0xfee00000}
\DoxyCodeLine{[    0.000000]  [ffffea0000000000-ffffea000affffff] PMD -> [ffff8802b5600000-ffff8802bf5fffff] on node 0}
\DoxyCodeLine{[    0.000000] Zone ranges:}
\DoxyCodeLine{[    0.000000]   DMA      [mem 0x0000000000001000-0x0000000000ffffff]}
\DoxyCodeLine{[    0.000000]   DMA32    [mem 0x0000000001000000-0x00000000ffffffff]}
\DoxyCodeLine{[    0.000000]   Normal   [mem 0x0000000100000000-0x00000002bfffffff]}
\DoxyCodeLine{[    0.000000] Movable zone start for each node}
\DoxyCodeLine{[    0.000000] Early memory node ranges}
\DoxyCodeLine{[    0.000000]   node   0: [mem 0x0000000000001000-0x000000000009efff]}
\DoxyCodeLine{[    0.000000]   node   0: [mem 0x0000000000100000-0x00000000bffddfff]}
\DoxyCodeLine{[    0.000000]   node   0: [mem 0x0000000100000000-0x00000002bfffffff]}
\DoxyCodeLine{[    0.000000] Initmem setup node 0 [mem 0x0000000000001000-0x00000002bfffffff]}
\DoxyCodeLine{[    0.000000] On node 0 totalpages: 2621308}
\DoxyCodeLine{[    0.000000]   DMA zone: 64 pages used for memmap}
\DoxyCodeLine{[    0.000000]   DMA zone: 21 pages reserved}
\DoxyCodeLine{[    0.000000]   DMA zone: 3998 pages, LIFO batch:0}
\DoxyCodeLine{[    0.000000]   DMA32 zone: 12224 pages used for memmap}
\DoxyCodeLine{[    0.000000]   DMA32 zone: 782302 pages, LIFO batch:31}
\DoxyCodeLine{[    0.000000]   Normal zone: 28672 pages used for memmap}
\DoxyCodeLine{[    0.000000]   Normal zone: 1835008 pages, LIFO batch:31}
\DoxyCodeLine{[    0.000000] ACPI: PM-Timer IO Port: 0x608}
\DoxyCodeLine{[    0.000000] ACPI: Local APIC address 0xfee00000}
\DoxyCodeLine{[    0.000000] ACPI: LAPIC\_NMI (acpi\_id[0xff] dfl dfl lint[0x1])}
\DoxyCodeLine{[    0.000000] IOAPIC[0]: apic\_id 0, version 17, address 0xfec00000, GSI 0-23}
\DoxyCodeLine{[    0.000000] ACPI: INT\_SRC\_OVR (bus 0 bus\_irq 0 global\_irq 2 dfl dfl)}
\DoxyCodeLine{[    0.000000] ACPI: INT\_SRC\_OVR (bus 0 bus\_irq 5 global\_irq 5 high level)}
\DoxyCodeLine{[    0.000000] ACPI: INT\_SRC\_OVR (bus 0 bus\_irq 9 global\_irq 9 high level)}
\DoxyCodeLine{[    0.000000] ACPI: INT\_SRC\_OVR (bus 0 bus\_irq 10 global\_irq 10 high level)}
\DoxyCodeLine{[    0.000000] ACPI: INT\_SRC\_OVR (bus 0 bus\_irq 11 global\_irq 11 high level)}
\DoxyCodeLine{[    0.000000] ACPI: IRQ0 used by override.}
\DoxyCodeLine{[    0.000000] ACPI: IRQ5 used by override.}
\DoxyCodeLine{[    0.000000] ACPI: IRQ9 used by override.}
\DoxyCodeLine{[    0.000000] ACPI: IRQ10 used by override.}
\DoxyCodeLine{[    0.000000] ACPI: IRQ11 used by override.}
\DoxyCodeLine{[    0.000000] Using ACPI (MADT) for SMP configuration information}
\DoxyCodeLine{[    0.000000] ACPI: HPET id: 0x8086a201 base: 0xfed00000}
\DoxyCodeLine{[    0.000000] smpboot: Allowing 2 CPUs, 0 hotplug CPUs}
\DoxyCodeLine{[    0.000000] e820: [mem 0xc0000000-0xfeffbfff] available for PCI devices}
\DoxyCodeLine{[    0.000000] clocksource refined-jiffies: mask: 0xffffffff max\_cycles: 0xffffffff, max\_idle\_ns: 1910969940391419 ns}
\DoxyCodeLine{[    0.000000] eda-debug:eda\_init(): size \& alignment came from: compiled in defaults}
\DoxyCodeLine{[    0.000000] eda: dma\_vaddr: 0xffff8800bbf00000, dma\_paddr: 0x00000000bbf00000, size: 0x04000000, alignment: 0x00100000}
\DoxyCodeLine{[    0.000000] setup\_percpu: NR\_CPUS:256 nr\_cpumask\_bits:256 nr\_cpu\_ids:2 nr\_node\_ids:1}
\DoxyCodeLine{[    0.000000] PERCPU: Embedded 33 pages/cpu @ffff8802bfc00000 s94936 r8192 d32040 u1048576}
\DoxyCodeLine{[    0.000000] pcpu-alloc: s94936 r8192 d32040 u1048576 alloc=1*2097152}
\DoxyCodeLine{[    0.000000] pcpu-alloc: [0] 0 1 }
\DoxyCodeLine{[    0.000000] Built 1 zonelists in Zone order, mobility grouping on.  Total pages: 2580327}
\DoxyCodeLine{[    0.000000] Kernel command line: BOOT\_IMAGE=/boot/vmlinuz-4.1.0-cl-7-amd64 root=UUID=77e5d5f6-9568-4559-8513-aaf6e20d1b5f ro cl\_platform=cumulus\_vx console=ttyS0,115200n8 console=tty0 quiet}
\DoxyCodeLine{[    0.000000] PID hash table entries: 4096 (order: 3, 32768 bytes)}
\DoxyCodeLine{[    0.000000] Dentry cache hash table entries: 2097152 (order: 12, 16777216 bytes)}
\DoxyCodeLine{[    0.000000] Inode-cache hash table entries: 1048576 (order: 11, 8388608 bytes)}
\DoxyCodeLine{[    0.000000] Memory: 10142664K/10485232K available (7568K kernel code, 1287K rwdata, 2444K rodata, 1212K init, 1640K bss, 342568K reserved, 0K cma-reserved)}
\DoxyCodeLine{[    0.000000] Hierarchical RCU implementation.}
\DoxyCodeLine{[    0.000000]  RCU dyntick-idle grace-period acceleration is enabled.}
\DoxyCodeLine{[    0.000000]  Additional per-CPU info printed with stalls.}
\DoxyCodeLine{[    0.000000]  RCU restricting CPUs from NR\_CPUS=256 to nr\_cpu\_ids=2.}
\DoxyCodeLine{[    0.000000] RCU: Adjusting geometry for rcu\_fanout\_leaf=16, nr\_cpu\_ids=2}
\DoxyCodeLine{[    0.000000] NR\_IRQS:16640 nr\_irqs:440 16}
\DoxyCodeLine{[    0.000000] Console: colour VGA+ 80x25}
\DoxyCodeLine{[    0.000000] console [tty0] enabled}
\DoxyCodeLine{[    0.000000] console [ttyS0] enabled}
\DoxyCodeLine{[    0.000000] clocksource hpet: mask: 0xffffffff max\_cycles: 0xffffffff, max\_idle\_ns: 19112604467 ns}
\DoxyCodeLine{[    0.000000] hpet clockevent registered}
\DoxyCodeLine{[    0.000000] tsc: Fast TSC calibration failed}
\DoxyCodeLine{[    0.000000] tsc: Unable to calibrate against PIT}
\DoxyCodeLine{[    0.000000] tsc: HPET/PMTIMER calibration failed}
\DoxyCodeLine{[    0.000000] tsc: Marking TSC unstable due to could not calculate TSC khz}
\DoxyCodeLine{[    0.006000] Calibrating delay loop... 2000.89 BogoMIPS (lpj=1000448)}
\DoxyCodeLine{[    0.012000] pid\_max: default: 32768 minimum: 301}
\DoxyCodeLine{[    0.013000] ACPI: Core revision 20150410}
\DoxyCodeLine{[    0.018000] ACPI: All ACPI Tables successfully acquired}
\DoxyCodeLine{[    0.020000] Mount-cache hash table entries: 32768 (order: 6, 262144 bytes)}
\DoxyCodeLine{[    0.020000] Mountpoint-cache hash table entries: 32768 (order: 6, 262144 bytes)}
\DoxyCodeLine{[    0.036000] Initializing cgroup subsys blkio}
\DoxyCodeLine{[    0.036000] Initializing cgroup subsys memory}
\DoxyCodeLine{[    0.036000] Initializing cgroup subsys devices}
\DoxyCodeLine{[    0.036000] Initializing cgroup subsys freezer}
\DoxyCodeLine{[    0.036000] Initializing cgroup subsys net\_cls}
\DoxyCodeLine{[    0.037000] Initializing cgroup subsys perf\_event}
\DoxyCodeLine{[    0.037000] Initializing cgroup subsys net\_prio}
\DoxyCodeLine{[    0.037000] Initializing cgroup subsys hugetlb}
\DoxyCodeLine{[    0.037000] Initializing cgroup subsys l3mdev}
\DoxyCodeLine{[    0.039000] mce: CPU supports 10 MCE banks}
\DoxyCodeLine{[    0.041000] Last level iTLB entries: 4KB 0, 2MB 0, 4MB 0}
\DoxyCodeLine{[    0.041000] Last level dTLB entries: 4KB 0, 2MB 0, 4MB 0, 1GB 0}
\DoxyCodeLine{[    0.084000] Freeing SMP alternatives memory: 28K (ffffffff81e72000 - ffffffff81e79000)}
\DoxyCodeLine{[    0.084000] ftrace: allocating 26938 entries in 106 pages}
\DoxyCodeLine{[    0.103000] x2apic: IRQ remapping doesn't support X2APIC mode}
\DoxyCodeLine{[    0.125000] ..TIMER: vector=0x30 apic1=0 pin1=2 apic2=-1 pin2=-1}
\DoxyCodeLine{[    0.135000] APIC calibration not consistent with PM-Timer: 143ms instead of 100ms}
\DoxyCodeLine{[    0.135000] APIC delta adjusted to PM-Timer: 6250841 (8986617)}
\DoxyCodeLine{[    0.135000] smpboot: CPU0: Intel(R) Core(TM)2 Duo CPU     T7700  @ 2.40GHz (fam: 06, model: 0f, stepping: 0b)}
\DoxyCodeLine{[    0.135000] Performance Events: unsupported p6 CPU model 15 no PMU driver, software events only.}
\DoxyCodeLine{[    0.143000] NMI watchdog: disabled (cpu0): hardware events not enabled}
\DoxyCodeLine{[    0.143000] NMI watchdog: Shutting down hard lockup detector on all cpus}
\DoxyCodeLine{[    0.146000] x86: Booting SMP configuration:}
\DoxyCodeLine{[    0.146000] .... node  \#0, CPUs:      \#1}
\DoxyCodeLine{[    0.192000] x86: Booted up 1 node, 2 CPUs}
\DoxyCodeLine{[    0.192000] smpboot: Total of 2 processors activated (4206.59 BogoMIPS)}
\DoxyCodeLine{[    0.204000] devtmpfs: initialized}
\DoxyCodeLine{[    0.211000] clocksource jiffies: mask: 0xffffffff max\_cycles: 0xffffffff, max\_idle\_ns: 1911260446275000 ns}
\DoxyCodeLine{[    0.220000] NET: Registered protocol family 16}
\DoxyCodeLine{[    0.231000] cpuidle: using governor ladder}
\DoxyCodeLine{[    0.238000] cpuidle: using governor menu}
\DoxyCodeLine{[    0.238000] ACPI: bus type PCI registered}
\DoxyCodeLine{[    0.238000] acpiphp: ACPI Hot Plug PCI Controller Driver version: 0.5}
\DoxyCodeLine{[    0.240000] PCI: Using configuration type 1 for base access}
\DoxyCodeLine{[    0.279000] ACPI: Added \_OSI(Module Device)}
\DoxyCodeLine{[    0.279000] ACPI: Added \_OSI(Processor Device)}
\DoxyCodeLine{[    0.279000] ACPI: Added \_OSI(3.0 \_SCP Extensions)}
\DoxyCodeLine{[    0.279000] ACPI: Added \_OSI(Processor Aggregator Device)}
\DoxyCodeLine{[    0.283000] ACPI: Interpreter enabled}
\DoxyCodeLine{[    0.283000] ACPI: (supports S0 S5)}
\DoxyCodeLine{[    0.283000] ACPI: Using IOAPIC for interrupt routing}
\DoxyCodeLine{[    0.283000] PCI: Using host bridge windows from ACPI; if necessary, use "pci=nocrs" and report a bug}
\DoxyCodeLine{[    0.300000] ACPI: PCI Root Bridge [PCI0] (domain 0000 [bus 00-ff])}
\DoxyCodeLine{[    0.300000] acpi PNP0A03:00: \_OSC: OS supports [ASPM ClockPM Segments MSI]}
\DoxyCodeLine{[    0.300000] acpi PNP0A03:00: \_OSC failed (AE\_NOT\_FOUND); disabling ASPM}
\DoxyCodeLine{[    0.300000] acpi PNP0A03:00: fail to add MMCONFIG information, can't access extended PCI configuration space under this bridge.}
\DoxyCodeLine{[    0.301000] acpiphp: Slot [3] registered}
\DoxyCodeLine{[    0.302000] acpiphp: Slot [4] registered}
\DoxyCodeLine{[    0.302000] acpiphp: Slot [5] registered}
\DoxyCodeLine{[    0.302000] acpiphp: Slot [6] registered}
\DoxyCodeLine{[    0.302000] acpiphp: Slot [7] registered}
\DoxyCodeLine{[    0.303000] acpiphp: Slot [8] registered}
\DoxyCodeLine{[    0.303000] acpiphp: Slot [9] registered}
\DoxyCodeLine{[    0.303000] acpiphp: Slot [10] registered}
\DoxyCodeLine{[    0.303000] acpiphp: Slot [11] registered}
\DoxyCodeLine{[    0.303000] acpiphp: Slot [12] registered}
\DoxyCodeLine{[    0.304000] acpiphp: Slot [13] registered}
\DoxyCodeLine{[    0.304000] acpiphp: Slot [14] registered}
\DoxyCodeLine{[    0.304000] acpiphp: Slot [15] registered}
\DoxyCodeLine{[    0.304000] acpiphp: Slot [16] registered}
\DoxyCodeLine{[    0.305000] acpiphp: Slot [17] registered}
\DoxyCodeLine{[    0.305000] acpiphp: Slot [18] registered}
\DoxyCodeLine{[    0.305000] acpiphp: Slot [19] registered}
\DoxyCodeLine{[    0.305000] acpiphp: Slot [20] registered}
\DoxyCodeLine{[    0.305000] acpiphp: Slot [21] registered}
\DoxyCodeLine{[    0.306000] acpiphp: Slot [22] registered}
\DoxyCodeLine{[    0.306000] acpiphp: Slot [23] registered}
\DoxyCodeLine{[    0.306000] acpiphp: Slot [24] registered}
\DoxyCodeLine{[    0.306000] acpiphp: Slot [25] registered}
\DoxyCodeLine{[    0.307000] acpiphp: Slot [26] registered}
\DoxyCodeLine{[    0.307000] acpiphp: Slot [27] registered}
\DoxyCodeLine{[    0.307000] acpiphp: Slot [28] registered}
\DoxyCodeLine{[    0.307000] acpiphp: Slot [29] registered}
\DoxyCodeLine{[    0.308000] acpiphp: Slot [30] registered}
\DoxyCodeLine{[    0.308000] acpiphp: Slot [31] registered}
\DoxyCodeLine{[    0.308000] PCI host bridge to bus 0000:00}
\DoxyCodeLine{[    0.308000] pci\_bus 0000:00: root bus resource [bus 00-ff]}
\DoxyCodeLine{[    0.308000] pci\_bus 0000:00: root bus resource [io  0x0000-0x0cf7 window]}
\DoxyCodeLine{[    0.308000] pci\_bus 0000:00: root bus resource [io  0x0d00-0xadff window]}
\DoxyCodeLine{[    0.308000] pci\_bus 0000:00: root bus resource [io  0xae0f-0xaeff window]}
\DoxyCodeLine{[    0.308000] pci\_bus 0000:00: root bus resource [io  0xaf20-0xafdf window]}
\DoxyCodeLine{[    0.308000] pci\_bus 0000:00: root bus resource [io  0xafe4-0xffff window]}
\DoxyCodeLine{[    0.308000] pci\_bus 0000:00: root bus resource [mem 0x000a0000-0x000bffff window]}
\DoxyCodeLine{[    0.308000] pci\_bus 0000:00: root bus resource [mem 0xc0000000-0xfebfffff window]}
\DoxyCodeLine{[    0.309000] pci 0000:00:00.0: [8086:1237] type 00 class 0x060000}
\DoxyCodeLine{[    0.313000] pci 0000:00:01.0: [8086:7000] type 00 class 0x060100}
\DoxyCodeLine{[    0.319000] pci 0000:00:01.1: [8086:7010] type 00 class 0x010180}
\DoxyCodeLine{[    0.331000] pci 0000:00:01.1: reg 0x20: [io  0xc0c0-0xc0cf]}
\DoxyCodeLine{[    0.335000] pci 0000:00:01.1: legacy IDE quirk: reg 0x10: [io  0x01f0-0x01f7]}
\DoxyCodeLine{[    0.335000] pci 0000:00:01.1: legacy IDE quirk: reg 0x14: [io  0x03f6]}
\DoxyCodeLine{[    0.335000] pci 0000:00:01.1: legacy IDE quirk: reg 0x18: [io  0x0170-0x0177]}
\DoxyCodeLine{[    0.335000] pci 0000:00:01.1: legacy IDE quirk: reg 0x1c: [io  0x0376]}
\DoxyCodeLine{[    0.337000] pci 0000:00:01.3: [8086:7113] type 00 class 0x068000}
\DoxyCodeLine{[    0.343000] pci 0000:00:01.3: quirk: [io  0x0600-0x063f] claimed by PIIX4 ACPI}
\DoxyCodeLine{[    0.343000] pci 0000:00:01.3: quirk: [io  0x0700-0x070f] claimed by PIIX4 SMB}
\DoxyCodeLine{[    0.345000] pci 0000:00:02.0: [1013:00b8] type 00 class 0x030000}
\DoxyCodeLine{[    0.356000] pci 0000:00:02.0: reg 0x10: [mem 0xfc000000-0xfdffffff pref]}
\DoxyCodeLine{[    0.367000] pci 0000:00:02.0: reg 0x14: [mem 0xfebf0000-0xfebf0fff]}
\DoxyCodeLine{[    0.412000] pci 0000:00:02.0: reg 0x30: [mem 0xfebe0000-0xfebeffff pref]}
\DoxyCodeLine{[    0.420000] pci 0000:00:03.0: [1af4:1000] type 00 class 0x020000}
\DoxyCodeLine{[    0.424000] pci 0000:00:03.0: reg 0x10: [io  0xc040-0xc05f]}
\DoxyCodeLine{[    0.428000] pci 0000:00:03.0: reg 0x14: [mem 0xfebf1000-0xfebf1fff]}
\DoxyCodeLine{[    0.440000] pci 0000:00:04.0: [1af4:1000] type 00 class 0x020000}
\DoxyCodeLine{[    0.443000] pci 0000:00:04.0: reg 0x10: [io  0xc060-0xc07f]}
\DoxyCodeLine{[    0.446000] pci 0000:00:04.0: reg 0x14: [mem 0xfebf2000-0xfebf2fff]}
\DoxyCodeLine{[    0.470000] pci 0000:00:05.0: [1af4:1000] type 00 class 0x020000}
\DoxyCodeLine{[    0.473000] pci 0000:00:05.0: reg 0x10: [io  0xc080-0xc09f]}
\DoxyCodeLine{[    0.477000] pci 0000:00:05.0: reg 0x14: [mem 0xfebf3000-0xfebf3fff]}
\DoxyCodeLine{[    0.489000] pci 0000:00:06.0: [1af4:1000] type 00 class 0x020000}
\DoxyCodeLine{[    0.492000] pci 0000:00:06.0: reg 0x10: [io  0xc0a0-0xc0bf]}
\DoxyCodeLine{[    0.493000] pci 0000:00:06.0: reg 0x14: [mem 0xfebf4000-0xfebf4fff]}
\DoxyCodeLine{[    0.505000] pci 0000:00:07.0: [1af4:1001] type 00 class 0x010000}
\DoxyCodeLine{[    0.509000] pci 0000:00:07.0: reg 0x10: [io  0xc000-0xc03f]}
\DoxyCodeLine{[    0.512000] pci 0000:00:07.0: reg 0x14: [mem 0xfebf5000-0xfebf5fff]}
\DoxyCodeLine{[    0.526000] pci\_bus 0000:00: on NUMA node 0}
\DoxyCodeLine{[    0.528000] ACPI: PCI Interrupt Link [LNKA] (IRQs 5 *10 11)}
\DoxyCodeLine{[    0.530000] ACPI: PCI Interrupt Link [LNKB] (IRQs 5 *10 11)}
\DoxyCodeLine{[    0.530000] ACPI: PCI Interrupt Link [LNKC] (IRQs 5 10 *11)}
\DoxyCodeLine{[    0.531000] ACPI: PCI Interrupt Link [LNKD] (IRQs 5 10 *11)}
\DoxyCodeLine{[    0.531000] ACPI: PCI Interrupt Link [LNKS] (IRQs *9)}
\DoxyCodeLine{[    0.536000] ACPI: Enabled 16 GPEs in block 00 to 0F}
\DoxyCodeLine{[    0.557000] SCSI subsystem initialized}
\DoxyCodeLine{[    0.577000] libata version 3.00 loaded.}
\DoxyCodeLine{[    0.581000] ACPI: bus type USB registered}
\DoxyCodeLine{[    0.583000] usbcore: registered new interface driver usbfs}
\DoxyCodeLine{[    0.583000] usbcore: registered new interface driver hub}
\DoxyCodeLine{[    0.585000] usbcore: registered new device driver usb}
\DoxyCodeLine{[    0.585000] pps\_core: LinuxPPS API ver. 1 registered}
\DoxyCodeLine{[    0.585000] pps\_core: Software ver. 5.3.6 - Copyright 2005-2007 Rodolfo Giometti <giometti@linux.it>}
\DoxyCodeLine{[    0.585000] PTP clock support registered}
\DoxyCodeLine{[    0.585000] EDAC MC: Ver: 3.0.0}
\DoxyCodeLine{[    0.587000] PCI: Using ACPI for IRQ routing}
\DoxyCodeLine{[    0.587000] PCI: pci\_cache\_line\_size set to 64 bytes}
\DoxyCodeLine{[    0.587000] e820: reserve RAM buffer [mem 0x0009fc00-0x0009ffff]}
\DoxyCodeLine{[    0.587000] e820: reserve RAM buffer [mem 0xbffde000-0xbfffffff]}
\DoxyCodeLine{[    0.602000] HPET: 3 timers in total, 0 timers will be used for per-cpu timer}
\DoxyCodeLine{[    0.607000] amd\_nb: Cannot enumerate AMD northbridges}
\DoxyCodeLine{[    0.609000] Switched to clocksource hpet}
\DoxyCodeLine{[    1.269281] pnp: PnP ACPI init}
\DoxyCodeLine{[    1.273332] pnp 00:00: Plug and Play ACPI device, IDs PNP0b00 (active)}
\DoxyCodeLine{[    1.274250] pnp 00:01: Plug and Play ACPI device, IDs PNP0303 (active)}
\DoxyCodeLine{[    1.275289] pnp 00:02: Plug and Play ACPI device, IDs PNP0f13 (active)}
\DoxyCodeLine{[    1.275289] pnp 00:03: [dma 2]}
\DoxyCodeLine{[    1.276252] pnp 00:03: Plug and Play ACPI device, IDs PNP0700 (active)}
\DoxyCodeLine{[    1.276252] pnp 00:04: Plug and Play ACPI device, IDs PNP0400 (active)}
\DoxyCodeLine{[    1.277251] pnp 00:05: Plug and Play ACPI device, IDs PNP0501 (active)}
\DoxyCodeLine{[    1.278251] pnp: PnP ACPI: found 6 devices}
\DoxyCodeLine{[    1.391255] clocksource acpi\_pm: mask: 0xffffff max\_cycles: 0xffffff, max\_idle\_ns: 2085701024 ns}
\DoxyCodeLine{[    1.392279] pci\_bus 0000:00: resource 4 [io  0x0000-0x0cf7 window]}
\DoxyCodeLine{[    1.392279] pci\_bus 0000:00: resource 5 [io  0x0d00-0xadff window]}
\DoxyCodeLine{[    1.392279] pci\_bus 0000:00: resource 6 [io  0xae0f-0xaeff window]}
\DoxyCodeLine{[    1.392279] pci\_bus 0000:00: resource 7 [io  0xaf20-0xafdf window]}
\DoxyCodeLine{[    1.392279] pci\_bus 0000:00: resource 8 [io  0xafe4-0xffff window]}
\DoxyCodeLine{[    1.392279] pci\_bus 0000:00: resource 9 [mem 0x000a0000-0x000bffff window]}
\DoxyCodeLine{[    1.392279] pci\_bus 0000:00: resource 10 [mem 0xc0000000-0xfebfffff window]}
\DoxyCodeLine{[    1.397292] NET: Registered protocol family 2}
\DoxyCodeLine{[    1.413327] TCP established hash table entries: 131072 (order: 8, 1048576 bytes)}
\DoxyCodeLine{[    1.457285] TCP bind hash table entries: 65536 (order: 8, 1048576 bytes)}
\DoxyCodeLine{[    1.482302] TCP: Hash tables configured (established 131072 bind 65536)}
\DoxyCodeLine{[    1.486319] UDP hash table entries: 8192 (order: 6, 262144 bytes)}
\DoxyCodeLine{[    1.492319] UDP-Lite hash table entries: 8192 (order: 6, 262144 bytes)}
\DoxyCodeLine{[    1.503329] NET: Registered protocol family 1}
\DoxyCodeLine{[    1.513346] RPC: Registered named UNIX socket transport module.}
\DoxyCodeLine{[    1.513346] RPC: Registered udp transport module.}
\DoxyCodeLine{[    1.513346] RPC: Registered tcp transport module.}
\DoxyCodeLine{[    1.513346] RPC: Registered tcp NFSv4.1 backchannel transport module.}
\DoxyCodeLine{[    1.513346] pci 0000:00:00.0: Limiting direct PCI/PCI transfers}
\DoxyCodeLine{[    1.514401] pci 0000:00:01.0: PIIX3: Enabling Passive Release}
\DoxyCodeLine{[    1.515347] pci 0000:00:01.0: Activating ISA DMA hang workarounds}
\DoxyCodeLine{[    1.516696] pci 0000:00:02.0: Video device with shadowed ROM}
\DoxyCodeLine{[    1.518477] PCI: CLS 0 bytes, default 64}
\DoxyCodeLine{[    1.540950] Trying to unpack rootfs image as initramfs...}
\DoxyCodeLine{[    3.898264] hrtimer: interrupt took 2606990 ns}
\DoxyCodeLine{[    4.378249] Freeing initrd memory: 5552K (ffff880037518000 - ffff880037a84000)}
\DoxyCodeLine{[    4.379361] PCI-DMA: Using software bounce buffering for IO (SWIOTLB)}
\DoxyCodeLine{[    4.379361] software IO TLB [mem 0xb7f00000-0xbbf00000] (64MB) mapped at [ffff8800b7f00000-ffff8800bbefffff]}
\DoxyCodeLine{[    4.392280] futex hash table entries: 512 (order: 3, 32768 bytes)}
\DoxyCodeLine{[    4.394243] audit: initializing netlink subsys (disabled)}
\DoxyCodeLine{[    4.395545] audit: type=2000 audit(1532990044.395:1): initialized}
\DoxyCodeLine{[    4.402327] HugeTLB registered 2 MB page size, pre-allocated 0 pages}
\DoxyCodeLine{[    4.413404] VFS: Disk quotas dquot\_6.6.0}
\DoxyCodeLine{[    4.418212] VFS: Dquot-cache hash table entries: 512 (order 0, 4096 bytes)}
\DoxyCodeLine{[    4.432314] squashfs: version 4.0 (2009/01/31) Phillip Lougher}
\DoxyCodeLine{[    4.445279] NFS: Registering the id\_resolver key type}
\DoxyCodeLine{[    4.446329] Key type id\_resolver registered}
\DoxyCodeLine{[    4.446329] Key type id\_legacy registered}
\DoxyCodeLine{[    4.446329] nfs4filelayout\_init: NFSv4 File Layout Driver Registering...}
\DoxyCodeLine{[    4.494033] alg: No test for stdrng (krng)}
\DoxyCodeLine{[    4.494948] bounce: pool size: 64 pages}
\DoxyCodeLine{[    4.495211] Block layer SCSI generic (bsg) driver version 0.4 loaded (major 250)}
\DoxyCodeLine{[    4.497336] io scheduler noop registered}
\DoxyCodeLine{[    4.497336] io scheduler deadline registered}
\DoxyCodeLine{[    4.497336] io scheduler cfq registered (default)}
\DoxyCodeLine{[    4.499240] pci\_hotplug: PCI Hot Plug PCI Core version: 0.5}
\DoxyCodeLine{[    4.499343] pciehp: PCI Express Hot Plug Controller Driver version: 0.4}
\DoxyCodeLine{[    4.500269] input: Power Button as /devices/LNXSYSTM:00/LNXPWRBN:00/input/input0}
\DoxyCodeLine{[    4.500269] ACPI: Power Button [PWRF]}
\DoxyCodeLine{[    4.502384] GHES: HEST is not enabled!}
\DoxyCodeLine{[    4.565258] ACPI: PCI Interrupt Link [LNKC] enabled at IRQ 11}
\DoxyCodeLine{[    4.566327] virtio-pci 0000:00:03.0: virtio\_pci: leaving for legacy driver}
\DoxyCodeLine{[    4.617298] ACPI: PCI Interrupt Link [LNKD] enabled at IRQ 10}
\DoxyCodeLine{[    4.618215] virtio-pci 0000:00:04.0: virtio\_pci: leaving for legacy driver}
\DoxyCodeLine{[    4.671307] ACPI: PCI Interrupt Link [LNKA] enabled at IRQ 10}
\DoxyCodeLine{[    4.671307] virtio-pci 0000:00:05.0: virtio\_pci: leaving for legacy driver}
\DoxyCodeLine{[    4.731300] ACPI: PCI Interrupt Link [LNKB] enabled at IRQ 11}
\DoxyCodeLine{[    4.732250] virtio-pci 0000:00:06.0: virtio\_pci: leaving for legacy driver}
\DoxyCodeLine{[    4.788310] virtio-pci 0000:00:07.0: virtio\_pci: leaving for legacy driver}
\DoxyCodeLine{[    4.792382] Serial: 8250/16550 driver, 4 ports, IRQ sharing enabled}
\DoxyCodeLine{[    4.857322] 00:05: ttyS0 at I/O 0x3f8 (irq = 4, base\_baud = 115200) is a 16550A}
\DoxyCodeLine{[    4.920304] brd: module loaded}
\DoxyCodeLine{[    4.939395] ata\_piix 0000:00:01.1: version 2.13}
\DoxyCodeLine{[    4.963072] scsi host0: ata\_piix}
\DoxyCodeLine{[    4.965288] scsi host1: ata\_piix}
\DoxyCodeLine{[    4.965288] ata1: PATA max MWDMA2 cmd 0x1f0 ctl 0x3f6 bmdma 0xc0c0 irq 14}
\DoxyCodeLine{[    4.965288] ata2: PATA max MWDMA2 cmd 0x170 ctl 0x376 bmdma 0xc0c8 irq 15}
\DoxyCodeLine{[    4.969294] ehci\_hcd: USB 2.0 'Enhanced' Host Controller (EHCI) Driver}
\DoxyCodeLine{[    4.973434] ehci-pci: EHCI PCI platform driver}
\DoxyCodeLine{[    4.974261] ohci\_hcd: USB 1.1 'Open' Host Controller (OHCI) Driver}
\DoxyCodeLine{[    4.974261] ohci-pci: OHCI PCI platform driver}
\DoxyCodeLine{[    4.975368] usbcore: registered new interface driver usb-storage}
\DoxyCodeLine{[    4.975368] usbcore: registered new interface driver usbserial}
\DoxyCodeLine{[    4.976263] usbcore: registered new interface driver usbserial\_generic}
\DoxyCodeLine{[    4.976263] usbserial: USB Serial support registered for generic}
\DoxyCodeLine{[    4.978100] i8042: PNP: PS/2 Controller [PNP0303:KBD,PNP0f13:MOU] at 0x60,0x64 irq 1,12}
\DoxyCodeLine{[    4.995630] serio: i8042 KBD port at 0x60,0x64 irq 1}
\DoxyCodeLine{[    4.996304] serio: i8042 AUX port at 0x60,0x64 irq 12}
\DoxyCodeLine{[    4.997293] mousedev: PS/2 mouse device common for all mice}
\DoxyCodeLine{[    5.008410] rtc\_cmos 00:00: RTC can wake from S4}
\DoxyCodeLine{[    5.020121] input: AT Translated Set 2 keyboard as /devices/platform/i8042/serio0/input/input1}
\DoxyCodeLine{[    5.022420] rtc\_cmos 00:00: rtc core: registered rtc\_cmos as rtc0}
\DoxyCodeLine{[    5.026530] rtc\_cmos 00:00: alarms up to one day, 114 bytes nvram, hpet irqs}
\DoxyCodeLine{[    5.027292] sdhci: Secure Digital Host Controller Interface driver}
\DoxyCodeLine{[    5.027292] sdhci: Copyright(c) Pierre Ossman}
\DoxyCodeLine{[    5.027292] sdhci-pltfm: SDHCI platform and OF driver helper}
\DoxyCodeLine{[    5.029331] usbcore: registered new interface driver usbhid}
\DoxyCodeLine{[    5.029331] usbhid: USB HID core driver}
\DoxyCodeLine{[    5.048328] ip\_tables: (C) 2000-2006 Netfilter Core Team}
\DoxyCodeLine{[    5.097290] NET: Registered protocol family 10}
\DoxyCodeLine{[    5.152437] ata1.01: NODEV after polling detection}
\DoxyCodeLine{[    5.155572] ata1.00: ATA-7: QEMU HARDDISK, 2.2.1, max UDMA/100}
\DoxyCodeLine{[    5.155572] ata1.00: 536870912 sectors, multi 16: LBA48 }
\DoxyCodeLine{[    5.158288] NET: Registered protocol family 17}
\DoxyCodeLine{[    5.159259] NET: Registered protocol family 15}
\DoxyCodeLine{[    5.160259] Key type dns\_resolver registered}
\DoxyCodeLine{[    5.162424] ata1.00: configured for MWDMA2}
\DoxyCodeLine{[    5.167289] scsi 0:0:0:0: Direct-Access     ATA      QEMU HARDDISK    1    PQ: 0 ANSI: 5}
\DoxyCodeLine{[    5.169352] registered taskstats version 1}
\DoxyCodeLine{[    5.171282] sd 0:0:0:0: [sda] 536870912 512-byte logical blocks: (275 GB/256 GiB)}
\DoxyCodeLine{[    5.173255] sd 0:0:0:0: [sda] Write Protect is off}
\DoxyCodeLine{[    5.173255] sd 0:0:0:0: [sda] Mode Sense: 00 3a 00 00}
\DoxyCodeLine{[    5.173721] sd 0:0:0:0: [sda] Write cache: enabled, read cache: enabled, doesn't support DPO or FUA}
\DoxyCodeLine{[    5.174324] sd 0:0:0:0: Attached scsi generic sg0 type 0}
\DoxyCodeLine{[    5.192279] rtc\_cmos 00:00: setting system clock to 2018-07-30 22:34:07 UTC (1532990047)}
\DoxyCodeLine{[    5.828280]  sda: sda1 sda2 sda3 sda4}
\DoxyCodeLine{[    5.880313] sd 0:0:0:0: [sda] Attached SCSI disk}
\DoxyCodeLine{[    5.928343] Freeing unused kernel memory: 1212K (ffffffff81d43000 - ffffffff81e72000)}
\DoxyCodeLine{[    5.928343] Write protecting the kernel read-only data: 12288k}
\DoxyCodeLine{[    5.957258] Freeing unused kernel memory: 612K (ffff880001767000 - ffff880001800000)}
\DoxyCodeLine{[    5.999209] Freeing unused kernel memory: 1652K (ffff880001a63000 - ffff880001c00000)}
\DoxyCodeLine{[    6.240209] systemd-udevd[73]: starting version 215}
\DoxyCodeLine{[    6.243245] random: systemd-udevd urandom read with 37 bits of entropy available}
\DoxyCodeLine{[    6.862251] piix4\_smbus 0000:00:01.3: SMBus Host Controller at 0x700, revision 0}
\DoxyCodeLine{[    7.251242] random: nonblocking pool is initialized}
\DoxyCodeLine{[    8.146300] device-mapper: uevent: version 1.0.3}
\DoxyCodeLine{[    8.147301] device-mapper: ioctl: 4.31.0-ioctl (2015-3-12) initialised: dm-devel@redhat.com}
\DoxyCodeLine{[    8.197314] raid6: sse2x1   gen()  2820 MB/s}
\DoxyCodeLine{[    8.214285] raid6: sse2x1   xor()  2257 MB/s}
\DoxyCodeLine{[    8.231751] raid6: sse2x2   gen()  3035 MB/s}
\DoxyCodeLine{[    8.248290] raid6: sse2x2   xor()  3027 MB/s}
\DoxyCodeLine{[    8.265245] raid6: sse2x4   gen()  4527 MB/s}
\DoxyCodeLine{[    8.282298] raid6: sse2x4   xor()  3650 MB/s}
\DoxyCodeLine{[    8.282298] raid6: using algorithm sse2x4 gen() 4527 MB/s}
\DoxyCodeLine{[    8.282298] raid6: .... xor() 3650 MB/s, rmw enabled}
\DoxyCodeLine{[    8.282298] raid6: using ssse3x2 recovery algorithm}
\DoxyCodeLine{[    8.286539] xor: measuring software checksum speed}
\DoxyCodeLine{[    8.296299]    prefetch64-sse:  8576.000 MB/sec}
\DoxyCodeLine{[    8.306220]    generic\_sse:  5860.000 MB/sec}
\DoxyCodeLine{[    8.306220] xor: using function: prefetch64-sse (8576.000 MB/sec)}
\DoxyCodeLine{[    8.426417] Btrfs loaded}
\DoxyCodeLine{[    8.894044] BTRFS: device label CL-SYSTEM devid 1 transid 5430 /dev/sda4}
\DoxyCodeLine{[    9.191691] BTRFS info (device sda4): allowing degraded mounts}
\DoxyCodeLine{[    9.191691] BTRFS info (device sda4): disk space caching is enabled}
\DoxyCodeLine{[    9.191691] BTRFS: has skinny extents}
\DoxyCodeLine{[   17.736281] systemd[1]: systemd 215 running in system mode. (+PAM +AUDIT +SELINUX +IMA +SYSVINIT +LIBCRYPTSETUP +GCRYPT +ACL +XZ -SECCOMP -APPARMOR)}
\DoxyCodeLine{[   17.736281] systemd[1]: Detected virtualization 'kvm'.}
\DoxyCodeLine{[   17.736281] systemd[1]: Detected architecture 'x86-64'.}
\DoxyCodeLine{[   19.054303] systemd[1]: Inserted module 'autofs4'}
\DoxyCodeLine{[   19.167317] systemd[1]: Set hostname to <oob-mgmt-server>.}
\DoxyCodeLine{[   29.337199] systemd[1]: [/lib/systemd/system/docker.service:25] Unknown lvalue 'Delegate' in section 'Service'}
\DoxyCodeLine{[   30.164315] systemd[1]: Starting Forward Password Requests to Wall Directory Watch.}
\DoxyCodeLine{[   30.165191] systemd[1]: Started Forward Password Requests to Wall Directory Watch.}
\DoxyCodeLine{[   30.165191] systemd[1]: Expecting device dev-ttyS0.device...}
\DoxyCodeLine{[   30.171193] systemd[1]: Starting Remote File Systems (Pre).}
\DoxyCodeLine{[   30.178450] systemd[1]: Reached target Remote File Systems (Pre).}
\DoxyCodeLine{[   30.179194] systemd[1]: Starting Dispatch Password Requests to Console Directory Watch.}
\DoxyCodeLine{[   30.179194] systemd[1]: Started Dispatch Password Requests to Console Directory Watch.}
\DoxyCodeLine{[   30.179194] systemd[1]: Starting Arbitrary Executable File Formats File System Automount Point.}
\DoxyCodeLine{[   30.188285] systemd[1]: Set up automount Arbitrary Executable File Formats File System Automount Point.}
\DoxyCodeLine{[   30.188285] systemd[1]: Starting Swap.}
\DoxyCodeLine{[   30.195179] systemd[1]: Reached target Swap.}
\DoxyCodeLine{[   30.195179] systemd[1]: Expecting device dev-disk-by\(\backslash\)x2duuid-77e5d5f6\(\backslash\)x2d9568\(\backslash\)x2d4559\(\backslash\)x2d8513\(\backslash\)x2daaf6e20d1b5f.device...}
\DoxyCodeLine{[   30.202413] systemd[1]: Starting Root Slice.}
\DoxyCodeLine{[   30.209345] systemd[1]: Created slice Root Slice.}
\DoxyCodeLine{[   30.209345] systemd[1]: Starting /dev/initctl Compatibility Named Pipe.}
\DoxyCodeLine{[   30.218190] systemd[1]: Listening on /dev/initctl Compatibility Named Pipe.}
\DoxyCodeLine{[   30.218190] systemd[1]: Starting Delayed Shutdown Socket.}
\DoxyCodeLine{[   30.225219] systemd[1]: Listening on Delayed Shutdown Socket.}
\DoxyCodeLine{[   30.225219] systemd[1]: Starting Journal Socket (/dev/log).}
\DoxyCodeLine{[   30.234203] systemd[1]: Listening on Journal Socket (/dev/log).}
\DoxyCodeLine{[   30.234203] systemd[1]: Starting LVM2 metadata daemon socket.}
\DoxyCodeLine{[   30.242393] systemd[1]: Listening on LVM2 metadata daemon socket.}
\DoxyCodeLine{[   30.242393] systemd[1]: Starting Device-mapper event daemon FIFOs.}
\DoxyCodeLine{[   30.251188] systemd[1]: Listening on Device-mapper event daemon FIFOs.}
\DoxyCodeLine{[   30.251188] systemd[1]: Starting User and Session Slice.}
\DoxyCodeLine{[   30.261368] systemd[1]: Created slice User and Session Slice.}
\DoxyCodeLine{[   30.261368] systemd[1]: Starting udev Control Socket.}
\DoxyCodeLine{[   30.269410] systemd[1]: Listening on udev Control Socket.}
\DoxyCodeLine{[   30.269410] systemd[1]: Starting udev Kernel Socket.}
\DoxyCodeLine{[   30.277546] systemd[1]: Listening on udev Kernel Socket.}
\DoxyCodeLine{[   30.277546] systemd[1]: Starting Journal Socket.}
\DoxyCodeLine{[   30.285295] systemd[1]: Listening on Journal Socket.}
\DoxyCodeLine{[   30.286193] systemd[1]: Starting System Slice.}
\DoxyCodeLine{[   30.294243] systemd[1]: Created slice System Slice.}
\DoxyCodeLine{[   30.294243] systemd[1]: Starting Increase datagram queue length...}
\DoxyCodeLine{[   30.304248] systemd[1]: Mounting Huge Pages File System...}
\DoxyCodeLine{[   30.315297] systemd[1]: Mounting POSIX Message Queue File System...}
\DoxyCodeLine{[   30.325215] systemd[1]: Starting Create list of required static device nodes for the current kernel...}
\DoxyCodeLine{[   30.340200] systemd[1]: Mounted Debug File System.}
\DoxyCodeLine{[   30.688308] systemd[1]: Started Set Up Additional Binary Formats.}
\DoxyCodeLine{[   30.688308] systemd[1]: Starting udev Coldplug all Devices...}
\DoxyCodeLine{[   30.708266] systemd[1]: Starting system-getty.slice.}
\DoxyCodeLine{[   30.723411] systemd[1]: Created slice system-getty.slice.}
\DoxyCodeLine{[   30.723411] systemd[1]: Starting system-serial\(\backslash\)x2dgetty.slice.}
\DoxyCodeLine{[   30.738183] systemd[1]: Created slice system-serial\(\backslash\)x2dgetty.slice.}
\DoxyCodeLine{[   30.894194] systemd[1]: Starting Load Kernel Modules...}
\DoxyCodeLine{[   30.909187] systemd[1]: Starting Slices.}
\DoxyCodeLine{[   30.921250] systemd[1]: Reached target Slices.}
\DoxyCodeLine{[   30.939200] systemd[1]: Started Create list of required static device nodes for the current kernel.}
\DoxyCodeLine{[   30.951175] systemd[1]: Starting Create Static Device Nodes in /dev...}
\DoxyCodeLine{[   31.251361] systemd[1]: Mounted POSIX Message Queue File System.}
\DoxyCodeLine{[   31.264473] systemd[1]: Mounted Huge Pages File System.}
\DoxyCodeLine{[   31.280247] systemd[1]: Started Increase datagram queue length.}
\DoxyCodeLine{[   31.301181] systemd[1]: Starting Syslog Socket.}
\DoxyCodeLine{[   31.314245] systemd[1]: Listening on Syslog Socket.}
\DoxyCodeLine{[   31.315200] systemd[1]: Starting Journal Service...}
\DoxyCodeLine{[   31.346290] systemd[1]: Started Journal Service.}
\DoxyCodeLine{[   32.338299] loop: module loaded}
\DoxyCodeLine{[   32.473313] Ethernet Channel Bonding Driver: v3.7.1 (April 27, 2011)}
\DoxyCodeLine{[   32.473313] MII link monitoring set to 100 ms}
\DoxyCodeLine{[   32.480320] systemd-udevd[263]: starting version 215}
\DoxyCodeLine{[   33.297175] bridge: automatic filtering via arp/ip/ip6tables has been deprecated. Update your scripts to load br\_netfilter if you need this.}
\DoxyCodeLine{[   33.420189] Bridge firewalling registered}
\DoxyCodeLine{[   33.429179] tun: Universal TUN/TAP device driver, 1.6}
\DoxyCodeLine{[   33.429179] tun: (C) 1999-2004 Max Krasnyansky <maxk@qualcomm.com>}
\DoxyCodeLine{[   34.913153] BTRFS info (device sda4): turning on flush-on-commit}
\DoxyCodeLine{[   34.913153] BTRFS info (device sda4): disk space caching is enabled}
\DoxyCodeLine{[   35.116207] cumulus\_vx\_platform: version 0.1 loaded}
\DoxyCodeLine{[   35.133816] softdog: Software Watchdog Timer: 0.08 initialized. soft\_noboot=0 soft\_margin=60 sec soft\_panic=0 (nowayout=0)}
\DoxyCodeLine{[   35.301366] disallow multiple vlans in bridge}
\DoxyCodeLine{[   35.302290] disable local fdb installation}
\DoxyCodeLine{[   35.302290] set stp state - user}
\DoxyCodeLine{[   35.315488] disable multicast ignore IPv6 solicited groups}
\DoxyCodeLine{[   38.336248] systemd-journald[257]: Received request to flush runtime journal from PID 1}
\DoxyCodeLine{[   40.799163] audit: type=1305 audit(1532990083.106:2): audit\_pid=349 old=0 auid=4294967295 ses=4294967295 res=1}
\DoxyCodeLine{[   47.180604] nf\_conntrack version 0.5.0 (65536 buckets, 262144 max)}
\DoxyCodeLine{[   59.866323] ip6\_tables: (C) 2000-2006 Netfilter Core Team}
\DoxyCodeLine{[   62.434205] Ebtables v2.0 registered}
\DoxyCodeLine{[   71.147494] ISO 9660 Extensions: Microsoft Joliet Level 3}
\DoxyCodeLine{[   71.150206] ISO 9660 Extensions: RRIP\_1991A}
\DoxyCodeLine{[   72.038292] BTRFS: new size for /dev/sda4 is 255728418816}
\DoxyCodeLine{[   95.182353] Initializing XFRM netlink socket}
\DoxyCodeLine{[   95.206528] Netfilter messages via NETLINK v0.30.}
\DoxyCodeLine{[   95.219279] ctnetlink v0.93: registering with nfnetlink.}
\DoxyCodeLine{[   95.417458] IPv6: ADDRCONF(NETDEV\_UP): docker0: link is not ready}
\end{DoxyCode}
 
\begin{DoxyCode}{0}
\DoxyCodeLine{Usage:}
\DoxyCodeLine{ dmesg [options]}
\DoxyCodeLine{}
\DoxyCodeLine{Options:}
\DoxyCodeLine{ -C, --clear                 clear the kernel ring buffer}
\DoxyCodeLine{ -c, --read-clear            read and clear all messages}
\DoxyCodeLine{ -D, --console-off           disable printing messages to console}
\DoxyCodeLine{ -E, --console-on            enable printing messages to console}
\DoxyCodeLine{ -F, --file <file>           use the file instead of the kernel log buffer}
\DoxyCodeLine{ -f, --facility <list>       restrict output to defined facilities}
\DoxyCodeLine{ -H, --human                 human readable output}
\DoxyCodeLine{ -k, --kernel                display kernel messages}
\DoxyCodeLine{ -L, --color[=<when>]        colorize messages (auto, always or never)}
\DoxyCodeLine{ -l, --level <list>          restrict output to defined levels}
\DoxyCodeLine{ -n, --console-level <level> set level of messages printed to console}
\DoxyCodeLine{ -P, --nopager               do not pipe output into a pager}
\DoxyCodeLine{ -r, --raw                   print the raw message buffer}
\DoxyCodeLine{ -S, --syslog                force to use syslog(2) rather than /dev/kmsg}
\DoxyCodeLine{ -s, --buffer-size <size>    buffer size to query the kernel ring buffer}
\DoxyCodeLine{ -u, --userspace             display userspace messages}
\DoxyCodeLine{ -w, --follow                wait for new messages}
\DoxyCodeLine{ -x, --decode                decode facility and level to readable string}
\DoxyCodeLine{ -d, --show-delta            show time delta between printed messages}
\DoxyCodeLine{ -e, --reltime               show local time and time delta in readable format}
\DoxyCodeLine{ -T, --ctime                 show human readable timestamp}
\DoxyCodeLine{ -t, --notime                don't print messages timestamp}
\DoxyCodeLine{     --time-format <format>  show time stamp using format:}
\DoxyCodeLine{                               [delta|reltime|ctime|notime|iso]}
\DoxyCodeLine{Suspending/resume will make ctime and iso timestamps inaccurate.}
\DoxyCodeLine{}
\DoxyCodeLine{ -h, --help     display this help and exit}
\DoxyCodeLine{ -V, --version  output version information and exit}
\DoxyCodeLine{}
\DoxyCodeLine{Supported log facilities:}
\DoxyCodeLine{    kern - kernel messages}
\DoxyCodeLine{    user - random user-level messages}
\DoxyCodeLine{    mail - mail system}
\DoxyCodeLine{  daemon - system daemons}
\DoxyCodeLine{    auth - security/authorization messages}
\DoxyCodeLine{  syslog - messages generated internally by syslogd}
\DoxyCodeLine{     lpr - line printer subsystem}
\DoxyCodeLine{    news - network news subsystem}
\DoxyCodeLine{}
\DoxyCodeLine{Supported log levels (priorities):}
\DoxyCodeLine{   emerg - system is unusable}
\DoxyCodeLine{   alert - action must be taken immediately}
\DoxyCodeLine{    crit - critical conditions}
\DoxyCodeLine{     err - error conditions}
\DoxyCodeLine{    warn - warning conditions}
\DoxyCodeLine{  notice - normal but significant condition}
\DoxyCodeLine{    info - informational}
\DoxyCodeLine{   debug - debug-level messages}
\DoxyCodeLine{}
\DoxyCodeLine{}
\DoxyCodeLine{For more details see dmesg(1).}
\end{DoxyCode}
 \subsection*{history}

\#\#\# most loved command of all time 
\begin{DoxyCode}{0}
\DoxyCodeLine{nohup: failed to run command ‘history’: No such file or directory}
\end{DoxyCode}
 
\begin{DoxyCode}{0}
\DoxyCodeLine{nohup: failed to run command ‘history’: No such file or directory}
\end{DoxyCode}
 \subsection*{hostname}

\#\#\# who is this system 
\begin{DoxyCode}{0}
\DoxyCodeLine{oob-mgmt-server}
\end{DoxyCode}
 
\begin{DoxyCode}{0}
\DoxyCodeLine{Usage: hostname [-b] \{hostname|-F file\}         set host name (from file)}
\DoxyCodeLine{       hostname [-a|-A|-d|-f|-i|-I|-s|-y]       display formatted name}
\DoxyCodeLine{       hostname                                 display host name}
\DoxyCodeLine{}
\DoxyCodeLine{       \{yp,nis,\}domainname \{nisdomain|-F file\}  set NIS domain name (from file)}
\DoxyCodeLine{       \{yp,nis,\}domainname                      display NIS domain name}
\DoxyCodeLine{}
\DoxyCodeLine{       dnsdomainname                            display dns domain name}
\DoxyCodeLine{}
\DoxyCodeLine{       hostname -V|--version|-h|--help          print info and exit}
\DoxyCodeLine{}
\DoxyCodeLine{Program name:}
\DoxyCodeLine{       \{yp,nis,\}domainname=hostname -y}
\DoxyCodeLine{       dnsdomainname=hostname -d}
\DoxyCodeLine{}
\DoxyCodeLine{Program options:}
\DoxyCodeLine{    -a, --alias            alias names}
\DoxyCodeLine{    -A, --all-fqdns        all long host names (FQDNs)}
\DoxyCodeLine{    -b, --boot             set default hostname if none available}
\DoxyCodeLine{    -d, --domain           DNS domain name}
\DoxyCodeLine{    -f, --fqdn, --long     long host name (FQDN)}
\DoxyCodeLine{    -F, --file             read host name or NIS domain name from given file}
\DoxyCodeLine{    -i, --ip-address       addresses for the host name}
\DoxyCodeLine{    -I, --all-ip-addresses all addresses for the host}
\DoxyCodeLine{    -s, --short            short host name}
\DoxyCodeLine{    -y, --yp, --nis        NIS/YP domain name}
\DoxyCodeLine{}
\DoxyCodeLine{Description:}
\DoxyCodeLine{   This command can get or set the host name or the NIS domain name. You can}
\DoxyCodeLine{   also get the DNS domain or the FQDN (fully qualified domain name).}
\DoxyCodeLine{   Unless you are using bind or NIS for host lookups you can change the}
\DoxyCodeLine{   FQDN (Fully Qualified Domain Name) and the DNS domain name (which is}
\DoxyCodeLine{   part of the FQDN) in the /etc/hosts file.}
\end{DoxyCode}
 \subsection*{date}

\#\#\# duh 
\begin{DoxyCode}{0}
\DoxyCodeLine{Mon Jul 30 23:57:57 UTC 2018}
\end{DoxyCode}
 
\begin{DoxyCode}{0}
\DoxyCodeLine{Usage: date [OPTION]... [+FORMAT]}
\DoxyCodeLine{  or:  date [-u|--utc|--universal] [MMDDhhmm[[CC]YY][.ss]]}
\DoxyCodeLine{Display the current time in the given FORMAT, or set the system date.}
\DoxyCodeLine{}
\DoxyCodeLine{Mandatory arguments to long options are mandatory for short options too.}
\DoxyCodeLine{  -d, --date=STRING         display time described by STRING, not 'now'}
\DoxyCodeLine{  -f, --file=DATEFILE       like --date once for each line of DATEFILE}
\DoxyCodeLine{  -I[TIMESPEC], --iso-8601[=TIMESPEC]  output date/time in ISO 8601 format.}
\DoxyCodeLine{                            TIMESPEC='date' for date only (the default),}
\DoxyCodeLine{                            'hours', 'minutes', 'seconds', or 'ns' for date}
\DoxyCodeLine{                            and time to the indicated precision.}
\DoxyCodeLine{  -r, --reference=FILE      display the last modification time of FILE}
\DoxyCodeLine{  -R, --rfc-2822            output date and time in RFC 2822 format.}
\DoxyCodeLine{                            Example: Mon, 07 Aug 2006 12:34:56 -0600}
\DoxyCodeLine{      --rfc-3339=TIMESPEC   output date and time in RFC 3339 format.}
\DoxyCodeLine{                            TIMESPEC='date', 'seconds', or 'ns' for}
\DoxyCodeLine{                            date and time to the indicated precision.}
\DoxyCodeLine{                            Date and time components are separated by}
\DoxyCodeLine{                            a single space: 2006-08-07 12:34:56-06:00}
\DoxyCodeLine{  -s, --set=STRING          set time described by STRING}
\DoxyCodeLine{  -u, --utc, --universal    print or set Coordinated Universal Time (UTC)}
\DoxyCodeLine{      --help     display this help and exit}
\DoxyCodeLine{      --version  output version information and exit}
\DoxyCodeLine{}
\DoxyCodeLine{FORMAT controls the output.  Interpreted sequences are:}
\DoxyCodeLine{}
\DoxyCodeLine{  \%\%   a literal \%}
\DoxyCodeLine{  \%a   locale's abbreviated weekday name (e.g., Sun)}
\DoxyCodeLine{  \%A   locale's full weekday name (e.g., Sunday)}
\DoxyCodeLine{  \%b   locale's abbreviated month name (e.g., Jan)}
\DoxyCodeLine{  \%B   locale's full month name (e.g., January)}
\DoxyCodeLine{  \%c   locale's date and time (e.g., Thu Mar  3 23:05:25 2005)}
\DoxyCodeLine{  \%C   century; like \%Y, except omit last two digits (e.g., 20)}
\DoxyCodeLine{  \%d   day of month (e.g., 01)}
\DoxyCodeLine{  \%D   date; same as \%m/\%d/\%y}
\DoxyCodeLine{  \%e   day of month, space padded; same as \%\_d}
\DoxyCodeLine{  \%F   full date; same as \%Y-\%m-\%d}
\DoxyCodeLine{  \%g   last two digits of year of ISO week number (see \%G)}
\DoxyCodeLine{  \%G   year of ISO week number (see \%V); normally useful only with \%V}
\DoxyCodeLine{  \%h   same as \%b}
\DoxyCodeLine{  \%H   hour (00..23)}
\DoxyCodeLine{  \%I   hour (01..12)}
\DoxyCodeLine{  \%j   day of year (001..366)}
\DoxyCodeLine{  \%k   hour, space padded ( 0..23); same as \%\_H}
\DoxyCodeLine{  \%l   hour, space padded ( 1..12); same as \%\_I}
\DoxyCodeLine{  \%m   month (01..12)}
\DoxyCodeLine{  \%M   minute (00..59)}
\DoxyCodeLine{  \%n   a newline}
\DoxyCodeLine{  \%N   nanoseconds (000000000..999999999)}
\DoxyCodeLine{  \%p   locale's equivalent of either AM or PM; blank if not known}
\DoxyCodeLine{  \%P   like \%p, but lower case}
\DoxyCodeLine{  \%r   locale's 12-hour clock time (e.g., 11:11:04 PM)}
\DoxyCodeLine{  \%R   24-hour hour and minute; same as \%H:\%M}
\DoxyCodeLine{  \%s   seconds since 1970-01-01 00:00:00 UTC}
\DoxyCodeLine{  \%S   second (00..60)}
\DoxyCodeLine{  \%t   a tab}
\DoxyCodeLine{  \%T   time; same as \%H:\%M:\%S}
\DoxyCodeLine{  \%u   day of week (1..7); 1 is Monday}
\DoxyCodeLine{  \%U   week number of year, with Sunday as first day of week (00..53)}
\DoxyCodeLine{  \%V   ISO week number, with Monday as first day of week (01..53)}
\DoxyCodeLine{  \%w   day of week (0..6); 0 is Sunday}
\DoxyCodeLine{  \%W   week number of year, with Monday as first day of week (00..53)}
\DoxyCodeLine{  \%x   locale's date representation (e.g., 12/31/99)}
\DoxyCodeLine{  \%X   locale's time representation (e.g., 23:13:48)}
\DoxyCodeLine{  \%y   last two digits of year (00..99)}
\DoxyCodeLine{  \%Y   year}
\DoxyCodeLine{  \%z   +hhmm numeric time zone (e.g., -0400)}
\DoxyCodeLine{  \%:z  +hh:mm numeric time zone (e.g., -04:00)}
\DoxyCodeLine{  \%::z  +hh:mm:ss numeric time zone (e.g., -04:00:00)}
\DoxyCodeLine{  \%:::z  numeric time zone with : to necessary precision (e.g., -04, +05:30)}
\DoxyCodeLine{  \%Z   alphabetic time zone abbreviation (e.g., EDT)}
\DoxyCodeLine{}
\DoxyCodeLine{By default, date pads numeric fields with zeroes.}
\DoxyCodeLine{The following optional flags may follow '\%':}
\DoxyCodeLine{}
\DoxyCodeLine{  -  (hyphen) do not pad the field}
\DoxyCodeLine{  \_  (underscore) pad with spaces}
\DoxyCodeLine{  0  (zero) pad with zeros}
\DoxyCodeLine{  \string^  use upper case if possible}
\DoxyCodeLine{  \#  use opposite case if possible}
\DoxyCodeLine{}
\DoxyCodeLine{After any flags comes an optional field width, as a decimal number;}
\DoxyCodeLine{then an optional modifier, which is either}
\DoxyCodeLine{E to use the locale's alternate representations if available, or}
\DoxyCodeLine{O to use the locale's alternate numeric symbols if available.}
\DoxyCodeLine{}
\DoxyCodeLine{Examples:}
\DoxyCodeLine{Convert seconds since the epoch (1970-01-01 UTC) to a date}
\DoxyCodeLine{  \$ date --date='@2147483647'}
\DoxyCodeLine{}
\DoxyCodeLine{Show the time on the west coast of the US (use tzselect(1) to find TZ)}
\DoxyCodeLine{  \$ TZ='America/Los\_Angeles' date}
\DoxyCodeLine{}
\DoxyCodeLine{Show the local time for 9AM next Friday on the west coast of the US}
\DoxyCodeLine{  \$ date --date='TZ="America/Los\_Angeles" 09:00 next Fri'}
\DoxyCodeLine{}
\DoxyCodeLine{GNU coreutils online help: <http://www.gnu.org/software/coreutils/>}
\DoxyCodeLine{Full documentation at: <http://www.gnu.org/software/coreutils/date>}
\DoxyCodeLine{or available locally via: info '(coreutils) date invocation'}
\DoxyCodeLine{}
\DoxyCodeLine{Mon Jul 30 23:57:57 UTC 2018}
\end{DoxyCode}
 \subsection*{uname -\/a}

\#\#\# duh 
\begin{DoxyCode}{0}
\DoxyCodeLine{Linux oob-mgmt-server 4.1.0-cl-7-amd64 \#1 SMP Debian 4.1.33-1+cl3u13 (2018-04-25) x86\_64 GNU/Linux}
\end{DoxyCode}
 
\begin{DoxyCode}{0}
\DoxyCodeLine{Usage: uname [OPTION]...}
\DoxyCodeLine{Print certain system information.  With no OPTION, same as -s.}
\DoxyCodeLine{}
\DoxyCodeLine{  -a, --all                print all information, in the following order,}
\DoxyCodeLine{                             except omit -p and -i if unknown:}
\DoxyCodeLine{  -s, --kernel-name        print the kernel name}
\DoxyCodeLine{  -n, --nodename           print the network node hostname}
\DoxyCodeLine{  -r, --kernel-release     print the kernel release}
\DoxyCodeLine{  -v, --kernel-version     print the kernel version}
\DoxyCodeLine{  -m, --machine            print the machine hardware name}
\DoxyCodeLine{  -p, --processor          print the processor type or "unknown"}
\DoxyCodeLine{  -i, --hardware-platform  print the hardware platform or "unknown"}
\DoxyCodeLine{  -o, --operating-system   print the operating system}
\DoxyCodeLine{      --help     display this help and exit}
\DoxyCodeLine{      --version  output version information and exit}
\DoxyCodeLine{}
\DoxyCodeLine{GNU coreutils online help: <http://www.gnu.org/software/coreutils/>}
\DoxyCodeLine{Full documentation at: <http://www.gnu.org/software/coreutils/uname>}
\DoxyCodeLine{or available locally via: info '(coreutils) uname invocation'}
\DoxyCodeLine{}
\DoxyCodeLine{Linux oob-mgmt-server 4.1.0-cl-7-amd64 \#1 SMP Debian 4.1.33-1+cl3u13 (2018-04-25) x86\_64 GNU/Linux}
\end{DoxyCode}
 \subsection*{net show version}

\#\#\# oh yeah 
\begin{DoxyCode}{0}
\DoxyCodeLine{NCLU\_VERSION=1.0}
\DoxyCodeLine{DISTRIB\_ID="Cumulus Linux"}
\DoxyCodeLine{DISTRIB\_RELEASE=3.6.1}
\DoxyCodeLine{DISTRIB\_DESCRIPTION="Cumulus Linux 3.6.1"}
\end{DoxyCode}
 
\begin{DoxyCode}{0}
\DoxyCodeLine{The following commands contain keyword(s) 'version', 'show'}
\DoxyCodeLine{}
\DoxyCodeLine{    net show version}
\end{DoxyCode}
 \subsection*{net show system}

\#\#\# 
\begin{DoxyCode}{0}
\DoxyCodeLine{Cumulus VX}
\DoxyCodeLine{Cumulus Linux 3.6.1}
\DoxyCodeLine{Build: Cumulus Linux 3.6.1}
\DoxyCodeLine{Uptime: 1:23:55.550000}
\end{DoxyCode}
 
\begin{DoxyCode}{0}
\DoxyCodeLine{The following commands contain keyword(s) 'system', 'show'}
\DoxyCodeLine{}
\DoxyCodeLine{    net show system}
\end{DoxyCode}
 