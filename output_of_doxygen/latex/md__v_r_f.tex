\subsection*{net example management-\/vrf}

\#\#\# 
\begin{DoxyCode}{0}
\DoxyCodeLine{When enabling management VRF the SSH daemon will be restarted as part of}
\DoxyCodeLine{the 'net commit'.  This will cause your SSH session to drop, this is expected}
\DoxyCodeLine{behavior.}
\DoxyCodeLine{}
\DoxyCodeLine{switch\# net add vrf mgmt}
\DoxyCodeLine{switch\# net commit}
\DoxyCodeLine{}
\DoxyCodeLine{}
\DoxyCodeLine{=======================================}
\DoxyCodeLine{Running a Service in the Management VRF}
\DoxyCodeLine{=======================================}
\DoxyCodeLine{By default services like SNMP, NTP, and installed 3rd party agents run in the}
\DoxyCodeLine{default VRF and thus can only provide service using the default VRF table via}
\DoxyCodeLine{interfaces also in the default VRF. If a service is desired to use the eth0}
\DoxyCodeLine{interface with management VRF applied the service must be restarted in the}
\DoxyCodeLine{management VRF.}
\end{DoxyCode}
 
\begin{DoxyCode}{0}
\DoxyCodeLine{The following commands contain keyword(s) 'management-vrf', 'example'}
\DoxyCodeLine{}
\DoxyCodeLine{    net example management-vrf}
\end{DoxyCode}
 \subsection*{vrf list}

\#\#\# 
\begin{DoxyCode}{0}
\DoxyCodeLine{VRF              Table}
\DoxyCodeLine{---------------- -----}
\end{DoxyCode}
 
\begin{DoxyCode}{0}
\DoxyCodeLine{vrf <OPTS>}
\DoxyCodeLine{}
\DoxyCodeLine{VRF domains:}
\DoxyCodeLine{    vrf list}
\DoxyCodeLine{}
\DoxyCodeLine{Links associated with VRF domains:}
\DoxyCodeLine{    vrf link list [<vrf-name>]}
\DoxyCodeLine{}
\DoxyCodeLine{Routes for a VRF domain:}
\DoxyCodeLine{    vrf route list [<vrf-name>]}
\DoxyCodeLine{}
\DoxyCodeLine{Tasks and VRF domain asociation:}
\DoxyCodeLine{    vrf task exec <vrf-name> <command>}
\DoxyCodeLine{    vrf task list [<vrf-name>]}
\DoxyCodeLine{    vrf task identify <pid>}
\DoxyCodeLine{}
\DoxyCodeLine{    NOTE: This command affects only AF\_INET and AF\_INET6 sockets opened by the}
\DoxyCodeLine{          command that gets exec'ed. Specifically, it has *no* impact on netlink}
\DoxyCodeLine{          sockets (e.g., ip command).}
\end{DoxyCode}
 \subsection*{vrf task list rocket}

\#\#\# 
\begin{DoxyCode}{0}
\DoxyCodeLine{VRF: rocket          }
\DoxyCodeLine{-----------------------}
\DoxyCodeLine{No cgroup for vrf}
\end{DoxyCode}
 
\begin{DoxyCode}{0}
\DoxyCodeLine{VRF: rocket          }
\DoxyCodeLine{-----------------------}
\DoxyCodeLine{No cgroup for vrf}
\DoxyCodeLine{}
\DoxyCodeLine{VRF: --help          }
\DoxyCodeLine{-----------------------}
\DoxyCodeLine{No cgroup for vrf}
\DoxyCodeLine{}
\DoxyCodeLine{}
\DoxyCodeLine{}
\DoxyCodeLine{VRF: rocket          }
\DoxyCodeLine{-----------------------}
\DoxyCodeLine{No cgroup for vrf}
\end{DoxyCode}
 \subsection*{vrf task identify 2829}

\#\#\# 
\begin{DoxyCode}{0}
\DoxyCodeLine{Process does not exist.}
\end{DoxyCode}
 
\begin{DoxyCode}{0}
\DoxyCodeLine{Process does not exist.}
\end{DoxyCode}
 \subsection*{net show bgp vrf turtle ipv4 unicast route-\/leak}

\#\#\# 
\begin{DoxyCode}{0}
\DoxyCodeLine{bgpd is not running}
\end{DoxyCode}
 
\begin{DoxyCode}{0}
\DoxyCodeLine{ERROR: There are no commands with keyword(s) 'route-leak', 'unicast', 'turtle', 'show', 'ipv4', 'bgp', 'vrf'}
\end{DoxyCode}
 \subsection*{net show route vrf turtle ipv4}

\#\#\# 
\begin{DoxyCode}{0}
\DoxyCodeLine{ERROR: Command not found.}
\DoxyCodeLine{}
\DoxyCodeLine{    net show route vrf turtle ipv4}
\DoxyCodeLine{                              \string^ Invalid value here.}
\DoxyCodeLine{}
\DoxyCodeLine{Use "net help KEYWORD(s)" to list all options that use KEYWORD(s).}
\end{DoxyCode}
 
\begin{DoxyCode}{0}
\DoxyCodeLine{ERROR: There are no commands with keyword(s) 'turtle', 'route', 'show', 'ipv4', 'vrf'}
\end{DoxyCode}
 \subsection*{cat /etc/frr/frr.conf}

\#\#\# 
\begin{DoxyCode}{0}
\DoxyCodeLine{frr version 4.0+cl3u2}
\DoxyCodeLine{frr defaults datacenter}
\DoxyCodeLine{hostname oob-mgmt-server}
\DoxyCodeLine{username cumulus nopassword}
\DoxyCodeLine{!}
\DoxyCodeLine{service integrated-vtysh-config}
\DoxyCodeLine{!}
\DoxyCodeLine{log syslog informational}
\DoxyCodeLine{!}
\DoxyCodeLine{interface eth2}
\DoxyCodeLine{ ipv6 nd ra-interval 10}
\DoxyCodeLine{ no ipv6 nd suppress-ra}
\DoxyCodeLine{!}
\DoxyCodeLine{interface eth3}
\DoxyCodeLine{ ipv6 nd ra-interval 10}
\DoxyCodeLine{ no ipv6 nd suppress-ra}
\DoxyCodeLine{!}
\DoxyCodeLine{line vty}
\DoxyCodeLine{!}
\end{DoxyCode}
 
\begin{DoxyCode}{0}
\DoxyCodeLine{Usage: cat [OPTION]... [FILE]...}
\DoxyCodeLine{Concatenate FILE(s), or standard input, to standard output.}
\DoxyCodeLine{}
\DoxyCodeLine{  -A, --show-all           equivalent to -vET}
\DoxyCodeLine{  -b, --number-nonblank    number nonempty output lines, overrides -n}
\DoxyCodeLine{  -e                       equivalent to -vE}
\DoxyCodeLine{  -E, --show-ends          display \$ at end of each line}
\DoxyCodeLine{  -n, --number             number all output lines}
\DoxyCodeLine{  -s, --squeeze-blank      suppress repeated empty output lines}
\DoxyCodeLine{  -t                       equivalent to -vT}
\DoxyCodeLine{  -T, --show-tabs          display TAB characters as \string^I}
\DoxyCodeLine{  -u                       (ignored)}
\DoxyCodeLine{  -v, --show-nonprinting   use \string^ and M- notation, except for LFD and TAB}
\DoxyCodeLine{      --help     display this help and exit}
\DoxyCodeLine{      --version  output version information and exit}
\DoxyCodeLine{}
\DoxyCodeLine{With no FILE, or when FILE is -, read standard input.}
\DoxyCodeLine{}
\DoxyCodeLine{Examples:}
\DoxyCodeLine{  cat f - g  Output f's contents, then standard input, then g's contents.}
\DoxyCodeLine{  cat        Copy standard input to standard output.}
\DoxyCodeLine{}
\DoxyCodeLine{GNU coreutils online help: <http://www.gnu.org/software/coreutils/>}
\DoxyCodeLine{Full documentation at: <http://www.gnu.org/software/coreutils/cat>}
\DoxyCodeLine{or available locally via: info '(coreutils) cat invocation'}
\DoxyCodeLine{}
\DoxyCodeLine{frr version 4.0+cl3u2}
\DoxyCodeLine{frr defaults datacenter}
\DoxyCodeLine{hostname oob-mgmt-server}
\DoxyCodeLine{username cumulus nopassword}
\DoxyCodeLine{!}
\DoxyCodeLine{service integrated-vtysh-config}
\DoxyCodeLine{!}
\DoxyCodeLine{log syslog informational}
\DoxyCodeLine{!}
\DoxyCodeLine{interface eth2}
\DoxyCodeLine{ ipv6 nd ra-interval 10}
\DoxyCodeLine{ no ipv6 nd suppress-ra}
\DoxyCodeLine{!}
\DoxyCodeLine{interface eth3}
\DoxyCodeLine{ ipv6 nd ra-interval 10}
\DoxyCodeLine{ no ipv6 nd suppress-ra}
\DoxyCodeLine{!}
\DoxyCodeLine{line vty}
\DoxyCodeLine{!}
\end{DoxyCode}
 \subsection*{net show bgp vrf rocket summary}

\#\#\# 
\begin{DoxyCode}{0}
\DoxyCodeLine{show bgp vrf rocket ipv4 unicast summary}
\DoxyCodeLine{========================================}
\DoxyCodeLine{bgpd is not running}
\DoxyCodeLine{}
\DoxyCodeLine{}
\DoxyCodeLine{show bgp vrf rocket ipv6 unicast summary}
\DoxyCodeLine{========================================}
\DoxyCodeLine{bgpd is not running}
\DoxyCodeLine{}
\DoxyCodeLine{}
\DoxyCodeLine{show bgp vrf rocket l2vpn evpn summary}
\DoxyCodeLine{======================================}
\DoxyCodeLine{bgpd is not running}
\end{DoxyCode}
 
\begin{DoxyCode}{0}
\DoxyCodeLine{ERROR: There are no commands with keyword(s) 'summary', 'rocket', 'show', 'bgp', 'vrf'}
\end{DoxyCode}
 \subsection*{net show bgp vrf vrf1012}

\#\#\# 
\begin{DoxyCode}{0}
\DoxyCodeLine{show bgp vrf vrf1012 ipv4 unicast}
\DoxyCodeLine{=================================}
\DoxyCodeLine{bgpd is not running}
\DoxyCodeLine{}
\DoxyCodeLine{}
\DoxyCodeLine{show bgp vrf vrf1012 ipv6 unicast}
\DoxyCodeLine{=================================}
\DoxyCodeLine{bgpd is not running}
\end{DoxyCode}
 
\begin{DoxyCode}{0}
\DoxyCodeLine{ERROR: There are no commands with keyword(s) 'vrf1012', 'show', 'bgp', 'vrf'}
\end{DoxyCode}
 \subsection*{net show ospf vrf all}

\#\#\# 
\begin{DoxyCode}{0}
\DoxyCodeLine{ospfd is not running}
\end{DoxyCode}
 
\begin{DoxyCode}{0}
\DoxyCodeLine{The following commands contain keyword(s) 'show', 'ospf', 'vrf', 'all'}
\DoxyCodeLine{}
\DoxyCodeLine{    net show ospf vrf <text> neighbor [all|<interface>|<ipv4>|detail] [json]}
\end{DoxyCode}
 \subsection*{net show ospf vrf vrf1012 route}

\#\#\# 
\begin{DoxyCode}{0}
\DoxyCodeLine{ospfd is not running}
\end{DoxyCode}
 
\begin{DoxyCode}{0}
\DoxyCodeLine{ERROR: There are no commands with keyword(s) 'vrf1012', 'route', 'show', 'ospf', 'vrf'}
\end{DoxyCode}
 \subsection*{ip -\/d link show type vrf}

\#\#\# 
\begin{DoxyCode}{0}
\end{DoxyCode}
 
\begin{DoxyCode}{0}
\DoxyCodeLine{Device "--help" does not exist.}
\end{DoxyCode}
 \subsection*{ip -\/d link show vrf vrf1012}

\#\#\# 
\begin{DoxyCode}{0}
\DoxyCodeLine{Error: argument "vrf1012" is wrong: Not a valid VRF name}
\end{DoxyCode}
 
\begin{DoxyCode}{0}
\DoxyCodeLine{Error: argument "vrf1012" is wrong: Not a valid VRF name}
\end{DoxyCode}
 \subsection*{ip route show table vrf1012}

\#\#\# 
\begin{DoxyCode}{0}
\DoxyCodeLine{Error: argument "vrf1012" is wrong: table id value is invalid}
\end{DoxyCode}
 
\begin{DoxyCode}{0}
\DoxyCodeLine{Error: argument "vrf1012" is wrong: table id value is invalid}
\end{DoxyCode}
 \subsection*{ip -\/6 route show table vrf1012}

\#\#\# 
\begin{DoxyCode}{0}
\DoxyCodeLine{Error: argument "vrf1012" is wrong: table id value is invalid}
\end{DoxyCode}
 
\begin{DoxyCode}{0}
\DoxyCodeLine{Error: argument "vrf1012" is wrong: table id value is invalid}
\end{DoxyCode}
 \subsection*{ip link list rocket}

\#\#\# 
\begin{DoxyCode}{0}
\DoxyCodeLine{Device "rocket" does not exist.}
\end{DoxyCode}
 
\begin{DoxyCode}{0}
\DoxyCodeLine{Error: either "dev" is duplicate, or "--help" is a garbage.}
\DoxyCodeLine{}
\DoxyCodeLine{Device "rocket" does not exist.}
\end{DoxyCode}
 \subsection*{net show mroute vrf blue}

\#\#\# 
\begin{DoxyCode}{0}
\DoxyCodeLine{pimd is not running}
\end{DoxyCode}
 
\begin{DoxyCode}{0}
\DoxyCodeLine{ERROR: There are no commands with keyword(s) 'mroute', 'show', 'blue', 'vrf'}
\end{DoxyCode}
 \subsection*{sudo systemctl status \href{mailto:dhcrelay@rocket.service}\texttt{ dhcrelay@rocket.\+service}}

\#\#\# 
\begin{DoxyCode}{0}
\DoxyCodeLine{● dhcrelay@rocket.service - DHCPv4 Relay Agent Daemon rocket}
\DoxyCodeLine{   Loaded: loaded (/lib/systemd/system/dhcrelay@.service; disabled)}
\DoxyCodeLine{  Drop-In: /run/systemd/generator/dhcrelay@.service.d}
\DoxyCodeLine{           └─vrf.conf}
\DoxyCodeLine{   Active: inactive (dead)}
\DoxyCodeLine{     Docs: man:dhcrelay(8)}
\end{DoxyCode}
 
\begin{DoxyCode}{0}
\DoxyCodeLine{systemctl [OPTIONS...] \{COMMAND\} ...}
\DoxyCodeLine{}
\DoxyCodeLine{Query or send control commands to the systemd manager.}
\DoxyCodeLine{}
\DoxyCodeLine{  -h --help           Show this help}
\DoxyCodeLine{     --version        Show package version}
\DoxyCodeLine{     --system         Connect to system manager}
\DoxyCodeLine{     --user           Connect to user service manager}
\DoxyCodeLine{  -H --host=[USER@]HOST}
\DoxyCodeLine{                      Operate on remote host}
\DoxyCodeLine{  -M --machine=CONTAINER}
\DoxyCodeLine{                      Operate on local container}
\DoxyCodeLine{  -t --type=TYPE      List only units of a particular type}
\DoxyCodeLine{     --state=STATE    List only units with particular LOAD or SUB or ACTIVE state}
\DoxyCodeLine{  -p --property=NAME  Show only properties by this name}
\DoxyCodeLine{  -a --all            Show all loaded units/properties, including dead/empty}
\DoxyCodeLine{                      ones. To list all units installed on the system, use}
\DoxyCodeLine{                      the 'list-unit-files' command instead.}
\DoxyCodeLine{  -l --full           Don't ellipsize unit names on output}
\DoxyCodeLine{  -r --recursive      Show unit list of host and local containers}
\DoxyCodeLine{     --reverse        Show reverse dependencies with 'list-dependencies'}
\DoxyCodeLine{     --job-mode=MODE  Specify how to deal with already queued jobs, when}
\DoxyCodeLine{                      queueing a new job}
\DoxyCodeLine{     --show-types     When showing sockets, explicitly show their type}
\DoxyCodeLine{  -i --ignore-inhibitors}
\DoxyCodeLine{                      When shutting down or sleeping, ignore inhibitors}
\DoxyCodeLine{     --kill-who=WHO   Who to send signal to}
\DoxyCodeLine{  -s --signal=SIGNAL  Which signal to send}
\DoxyCodeLine{  -q --quiet          Suppress output}
\DoxyCodeLine{     --no-block       Do not wait until operation finished}
\DoxyCodeLine{     --no-wall        Don't send wall message before halt/power-off/reboot}
\DoxyCodeLine{     --no-reload      When enabling/disabling unit files, don't reload daemon}
\DoxyCodeLine{                      configuration}
\DoxyCodeLine{     --no-legend      Do not print a legend (column headers and hints)}
\DoxyCodeLine{     --no-pager       Do not pipe output into a pager}
\DoxyCodeLine{     --no-ask-password}
\DoxyCodeLine{                      Do not ask for system passwords}
\DoxyCodeLine{     --global         Enable/disable unit files globally}
\DoxyCodeLine{     --runtime        Enable unit files only temporarily until next reboot}
\DoxyCodeLine{  -f --force          When enabling unit files, override existing symlinks}
\DoxyCodeLine{                      When shutting down, execute action immediately}
\DoxyCodeLine{     --preset-mode=   Specifies whether fully apply presets, or only enable,}
\DoxyCodeLine{                      or only disable}
\DoxyCodeLine{     --root=PATH      Enable unit files in the specified root directory}
\DoxyCodeLine{  -n --lines=INTEGER  Number of journal entries to show}
\DoxyCodeLine{  -o --output=STRING  Change journal output mode (short, short-monotonic,}
\DoxyCodeLine{                      verbose, export, json, json-pretty, json-sse, cat)}
\DoxyCodeLine{     --plain          Print unit dependencies as a list instead of a tree}
\DoxyCodeLine{}
\DoxyCodeLine{Unit Commands:}
\DoxyCodeLine{  list-units [PATTERN...]         List loaded units}
\DoxyCodeLine{  list-sockets [PATTERN...]       List loaded sockets ordered by address}
\DoxyCodeLine{  list-timers [PATTERN...]        List loaded timers ordered by next elapse}
\DoxyCodeLine{  start NAME...                   Start (activate) one or more units}
\DoxyCodeLine{  stop NAME...                    Stop (deactivate) one or more units}
\DoxyCodeLine{  reload NAME...                  Reload one or more units}
\DoxyCodeLine{  restart NAME...                 Start or restart one or more units}
\DoxyCodeLine{  try-restart NAME...             Restart one or more units if active}
\DoxyCodeLine{  reload-or-restart NAME...       Reload one or more units if possible,}
\DoxyCodeLine{                                  otherwise start or restart}
\DoxyCodeLine{  reload-or-try-restart NAME...   Reload one or more units if possible,}
\DoxyCodeLine{                                  otherwise restart if active}
\DoxyCodeLine{  isolate NAME                    Start one unit and stop all others}
\DoxyCodeLine{  kill NAME...                    Send signal to processes of a unit}
\DoxyCodeLine{  is-active PATTERN...            Check whether units are active}
\DoxyCodeLine{  is-failed PATTERN...            Check whether units are failed}
\DoxyCodeLine{  status [PATTERN...|PID...]      Show runtime status of one or more units}
\DoxyCodeLine{  show [PATTERN...|JOB...]        Show properties of one or more}
\DoxyCodeLine{                                  units/jobs or the manager}
\DoxyCodeLine{  cat PATTERN...                  Show files and drop-ins of one or more units}
\DoxyCodeLine{  set-property NAME ASSIGNMENT... Sets one or more properties of a unit}
\DoxyCodeLine{  help PATTERN...|PID...          Show manual for one or more units}
\DoxyCodeLine{  reset-failed [PATTERN...]       Reset failed state for all, one, or more}
\DoxyCodeLine{                                  units}
\DoxyCodeLine{  list-dependencies [NAME]        Recursively show units which are required}
\DoxyCodeLine{                                  or wanted by this unit or by which this}
\DoxyCodeLine{                                  unit is required or wanted}
\DoxyCodeLine{}
\DoxyCodeLine{Unit File Commands:}
\DoxyCodeLine{  list-unit-files [PATTERN...]    List installed unit files}
\DoxyCodeLine{  enable NAME...                  Enable one or more unit files}
\DoxyCodeLine{  disable NAME...                 Disable one or more unit files}
\DoxyCodeLine{  reenable NAME...                Reenable one or more unit files}
\DoxyCodeLine{  preset NAME...                  Enable/disable one or more unit files}
\DoxyCodeLine{                                  based on preset configuration}
\DoxyCodeLine{  preset-all                      Enable/disable all unit files based on}
\DoxyCodeLine{                                  preset configuration}
\DoxyCodeLine{  is-enabled NAME...              Check whether unit files are enabled}
\DoxyCodeLine{}
\DoxyCodeLine{  mask NAME...                    Mask one or more units}
\DoxyCodeLine{  unmask NAME...                  Unmask one or more units}
\DoxyCodeLine{  link PATH...                    Link one or more units files into}
\DoxyCodeLine{                                  the search path}
\DoxyCodeLine{  get-default                     Get the name of the default target}
\DoxyCodeLine{  set-default NAME                Set the default target}
\DoxyCodeLine{}
\DoxyCodeLine{Machine Commands:}
\DoxyCodeLine{  list-machines [PATTERN...]      List local containers and host}
\DoxyCodeLine{}
\DoxyCodeLine{Job Commands:}
\DoxyCodeLine{  list-jobs [PATTERN...]          List jobs}
\DoxyCodeLine{  cancel [JOB...]                 Cancel all, one, or more jobs}
\DoxyCodeLine{}
\DoxyCodeLine{Snapshot Commands:}
\DoxyCodeLine{  snapshot [NAME]                 Create a snapshot}
\DoxyCodeLine{  delete NAME...                  Remove one or more snapshots}
\DoxyCodeLine{}
\DoxyCodeLine{Environment Commands:}
\DoxyCodeLine{  show-environment                Dump environment}
\DoxyCodeLine{  set-environment NAME=VALUE...   Set one or more environment variables}
\DoxyCodeLine{  unset-environment NAME...       Unset one or more environment variables}
\DoxyCodeLine{  import-environment NAME...      Import all, one or more environment variables}
\DoxyCodeLine{}
\DoxyCodeLine{Manager Lifecycle Commands:}
\DoxyCodeLine{  daemon-reload                   Reload systemd manager configuration}
\DoxyCodeLine{  daemon-reexec                   Reexecute systemd manager}
\DoxyCodeLine{}
\DoxyCodeLine{System Commands:}
\DoxyCodeLine{  is-system-running               Check whether system is fully running}
\DoxyCodeLine{  default                         Enter system default mode}
\DoxyCodeLine{  rescue                          Enter system rescue mode}
\DoxyCodeLine{  emergency                       Enter system emergency mode}
\DoxyCodeLine{  halt                            Shut down and halt the system}
\DoxyCodeLine{  poweroff                        Shut down and power-off the system}
\DoxyCodeLine{  reboot [ARG]                    Shut down and reboot the system}
\DoxyCodeLine{  kexec                           Shut down and reboot the system with kexec}
\DoxyCodeLine{  exit                            Request user instance exit}
\DoxyCodeLine{  switch-root ROOT [INIT]         Change to a different root file system}
\DoxyCodeLine{  suspend                         Suspend the system}
\DoxyCodeLine{  hibernate                       Hibernate the system}
\DoxyCodeLine{  hybrid-sleep                    Hibernate and suspend the system}
\end{DoxyCode}
 \subsection*{sudo vrf task exec rocket /usr/sbin/dhcrelay -\/d -\/q -\/i DP -\/i DP 102.\+0.\+0.\+2}

\#\#\# 
\begin{DoxyCode}{0}
\DoxyCodeLine{ERROR: VRF does not exist}
\end{DoxyCode}
 
\begin{DoxyCode}{0}
\DoxyCodeLine{ERROR: VRF does not exist}
\end{DoxyCode}
 